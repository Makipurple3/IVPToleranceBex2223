% Options for packages loaded elsewhere
\PassOptionsToPackage{unicode}{hyperref}
\PassOptionsToPackage{hyphens}{url}
%
\documentclass[
]{article}
\title{Testing for Tolerance in a Box Experiment in Wild Vervet Monkeys
- M.AungKyaw - 2022-2023}
\author{}
\date{\vspace{-2.5em}janvier 02, 2025 17:55}

\usepackage{amsmath,amssymb}
\usepackage{lmodern}
\usepackage{iftex}
\ifPDFTeX
  \usepackage[T1]{fontenc}
  \usepackage[utf8]{inputenc}
  \usepackage{textcomp} % provide euro and other symbols
\else % if luatex or xetex
  \usepackage{unicode-math}
  \defaultfontfeatures{Scale=MatchLowercase}
  \defaultfontfeatures[\rmfamily]{Ligatures=TeX,Scale=1}
\fi
% Use upquote if available, for straight quotes in verbatim environments
\IfFileExists{upquote.sty}{\usepackage{upquote}}{}
\IfFileExists{microtype.sty}{% use microtype if available
  \usepackage[]{microtype}
  \UseMicrotypeSet[protrusion]{basicmath} % disable protrusion for tt fonts
}{}
\makeatletter
\@ifundefined{KOMAClassName}{% if non-KOMA class
  \IfFileExists{parskip.sty}{%
    \usepackage{parskip}
  }{% else
    \setlength{\parindent}{0pt}
    \setlength{\parskip}{6pt plus 2pt minus 1pt}}
}{% if KOMA class
  \KOMAoptions{parskip=half}}
\makeatother
\usepackage{xcolor}
\IfFileExists{xurl.sty}{\usepackage{xurl}}{} % add URL line breaks if available
\IfFileExists{bookmark.sty}{\usepackage{bookmark}}{\usepackage{hyperref}}
\hypersetup{
  pdftitle={Testing for Tolerance in a Box Experiment in Wild Vervet Monkeys - M.AungKyaw - 2022-2023},
  hidelinks,
  pdfcreator={LaTeX via pandoc}}
\urlstyle{same} % disable monospaced font for URLs
\usepackage[margin=1in]{geometry}
\usepackage{longtable,booktabs,array}
\usepackage{calc} % for calculating minipage widths
% Correct order of tables after \paragraph or \subparagraph
\usepackage{etoolbox}
\makeatletter
\patchcmd\longtable{\par}{\if@noskipsec\mbox{}\fi\par}{}{}
\makeatother
% Allow footnotes in longtable head/foot
\IfFileExists{footnotehyper.sty}{\usepackage{footnotehyper}}{\usepackage{footnote}}
\makesavenoteenv{longtable}
\usepackage{graphicx}
\makeatletter
\def\maxwidth{\ifdim\Gin@nat@width>\linewidth\linewidth\else\Gin@nat@width\fi}
\def\maxheight{\ifdim\Gin@nat@height>\textheight\textheight\else\Gin@nat@height\fi}
\makeatother
% Scale images if necessary, so that they will not overflow the page
% margins by default, and it is still possible to overwrite the defaults
% using explicit options in \includegraphics[width, height, ...]{}
\setkeys{Gin}{width=\maxwidth,height=\maxheight,keepaspectratio}
% Set default figure placement to htbp
\makeatletter
\def\fps@figure{htbp}
\makeatother
\setlength{\emergencystretch}{3em} % prevent overfull lines
\providecommand{\tightlist}{%
  \setlength{\itemsep}{0pt}\setlength{\parskip}{0pt}}
\setcounter{secnumdepth}{-\maxdimen} % remove section numbering
\usepackage[utf8]{inputenc}
\usepackage{amssymb}
\usepackage{textcomp}
\usepackage[T1]{fontenc}
\usepackage{underscore}
\usepackage{newunicodechar}
\newunicodechar{≠}{$\neq$}
\ifLuaTeX
  \usepackage{selnolig}  % disable illegal ligatures
\fi

\begin{document}
\maketitle

{
\setcounter{tocdepth}{6}
\tableofcontents
}
\hypertarget{introduction}{%
\section{Introduction}\label{introduction}}

\hypertarget{opening-the-data}{%
\section{0.Opening the data}\label{opening-the-data}}

\hypertarget{loading-data}{%
\subsubsection{Loading data}\label{loading-data}}

\begin{itemize}
\tightlist
\item
  First I downloaded the \textbf{knitr package} to create outputs as
  html, pdf or word files when knitting my r markdown file. I also
  loaded the \textbf{pander} package for better presentation
\item
  The \textbf{dplyr} package was installed for better manipulation of
  the data as filtering or creating new variables and \textbf{lubridate}
  for a better manipulation of dates and time
\item
  Then, I installed the \textbf{readxl package} to import my dataset
  which is called \textbf{Box Experiments.xls}
\item
  This dataset contains information related to my master thesis project.
  I used cyber tracker in order to record the behaviors of dyads of
  Vervet monkeys in a box experiment on tolerance from September 2022 to
  September 2023
\end{itemize}

\hypertarget{explore-the-data}{%
\section{1.Explore the data}\label{explore-the-data}}

\hypertarget{description-of-the-initial-datset---boxex}{%
\subsubsection{Description of the initial datset -
``Boxex''}\label{description-of-the-initial-datset---boxex}}

\begin{verbatim}
## Glimpse of the Box Experiment dataset:
\end{verbatim}

\begin{verbatim}
## Rows: 2,795
## Columns: 20
## $ Date                  <dttm> 2022-09-27, 2022-09-27, 2022-09-27, 2022-09-27,~
## $ Time                  <dttm> 1899-12-31 09:47:50, 1899-12-31 09:50:07, 1899-~
## $ Data                  <chr> "Box Experiment", "Box Experiment", "Box Experim~
## $ Group                 <chr> "Baie Dankie", "Baie Dankie", "Baie Dankie", "Ba~
## $ GPSS                  <chr> "-28.010549999999999", "-28.010549999999999", "-~
## $ GPSE                  <chr> "31.191050000000001", "31.191050000000001", "31.~
## $ MaleID                <chr> "Nge", "Nge", "Nge", "Nge", "Nge", "Nge", "Nge",~
## $ FemaleID              <chr> "Oerw", "Oerw", "Oerw", "Oerw", "Oerw", "Oerw", ~
## $ `Male placement corn` <dbl> NA, NA, NA, NA, NA, NA, NA, NA, NA, NA, NA, NA, ~
## $ MaleCorn              <dbl> 3, 3, 3, 3, 3, 3, 3, 3, 3, 3, 3, 3, 3, 3, 3, 3, ~
## $ FemaleCorn            <dbl> NA, NA, NA, NA, NA, NA, NA, NA, NA, NA, NA, NA, ~
## $ DyadDistance          <chr> "2m", "2m", "1m", "1m", "0m", "0m", "0m", "0m", ~
## $ DyadResponse          <chr> "Tolerance", "Tolerance", "Tolerance", "Toleranc~
## $ OtherResponse         <chr> NA, NA, NA, NA, NA, NA, NA, NA, NA, NA, NA, NA, ~
## $ Audience              <chr> "Obse; Oup; Sirk", "Obse; Oup; Sirk", "Oup; Sirk~
## $ IDIndividual1         <chr> NA, NA, NA, NA, NA, NA, NA, NA, NA, NA, NA, NA, ~
## $ IntruderID            <chr> NA, NA, NA, NA, NA, NA, NA, NA, NA, NA, NA, "Sey~
## $ Remarks               <chr> NA, NA, "Nge box did not open because of the bat~
## $ Observers             <chr> "Josefien; Michael; Ona; Zonke", "Josefien; Mich~
## $ DeviceId              <chr> "{7A4E6639-7387-7648-88EC-7FD27A0F258A}", "{7A4E~
\end{verbatim}

\begin{verbatim}
## # A tibble: 6 x 20
##   Date                Time                Data          Group GPSS  GPSE  MaleID
##   <dttm>              <dttm>              <chr>         <chr> <chr> <chr> <chr> 
## 1 2022-09-27 00:00:00 1899-12-31 09:47:50 Box Experime~ Baie~ -28.~ 31.1~ Nge   
## 2 2022-09-27 00:00:00 1899-12-31 09:50:07 Box Experime~ Baie~ -28.~ 31.1~ Nge   
## 3 2022-09-27 00:00:00 1899-12-31 09:53:11 Box Experime~ Baie~ -28.~ 31.1~ Nge   
## 4 2022-09-27 00:00:00 1899-12-31 09:54:28 Box Experime~ Baie~ -28.~ 31.1~ Nge   
## 5 2022-09-27 00:00:00 1899-12-31 09:55:19 Box Experime~ Baie~ -28.~ 31.1~ Nge   
## 6 2022-09-27 00:00:00 1899-12-31 09:56:56 Box Experime~ Baie~ -28.~ 31.1~ Nge   
## # i 13 more variables: FemaleID <chr>, `Male placement corn` <dbl>,
## #   MaleCorn <dbl>, FemaleCorn <dbl>, DyadDistance <chr>, DyadResponse <chr>,
## #   OtherResponse <chr>, Audience <chr>, IDIndividual1 <chr>, IntruderID <chr>,
## #   Remarks <chr>, Observers <chr>, DeviceId <chr>
\end{verbatim}

\begin{itemize}
\item
  I am now using the \textbf{View} function to have a sight on the
  entire dataset and \textbf{glimpse} to display a summary of my dataset
\item
  I have \textbf{20 variables} (here columns) and \textbf{2795 trials}
  (here rows)
\item
  I will now make a brief summary of each variables and their use before
  creating a new dataframe (df) with my variables of interest that I
  will call \textbf{Bex}
\item
  The highlighted variables are the ones I will use for \textbf{Bex}. I
  will then \textbf{clean the data} before heading to the
  \textbf{statistical analysis} and the \textbf{interpretation of the
  results}
\end{itemize}

\hypertarget{variables-of-boxex}{%
\paragraph{Variables of Boxex}\label{variables-of-boxex}}

\begin{itemize}
\item
  \textbf{Date} : ``Date'' is in a \textbf{POSIXct} format which is
  appropriate for the display of time

  \begin{itemize}
  \tightlist
  \item
    I want to use the date to know \textbf{how many sessions} have been
    done with each dyads in my experiment.
  \item
    I will create a variable called \textbf{Session} where \textbf{1
    session = 1 day}
  \item
    The data has values from the \textbf{14th of September 2022} until
    the \textbf{13th of September 2023}
  \item
    I may consider separating the \textbf{12 months} of data in
    \textbf{4 seasons} to make a preliminary check of a potential effect
    of seasonality. Nevertheless since we did not use any tools to
    measure the weather, temperature, humidity or food availability
    (also related to seasonality and weather). Categorizing my data in 4
    without having further data would then be quite arbitrary. If I end
    up doing it in my report, it will be done without any intention to
    include it in my scientific analysis nor my scientific report .
  \end{itemize}
\item
  \textbf{Time} : ``Time is coded'' in a \textbf{POSIXct} format

  \begin{itemize}
  \tightlist
  \item
    I do not plan to use this variable but we can see that ``Time'' has
    the correct hours displayed with a date which is incorrect.
  \item
    (In the case I wanted to observe \textbf{when the trials occurred
    during the day} as time may have an influence on their behavior
    (\textbf{Isbell \& Young 1993})) I would need to correct the
    incorrect display of the date in the dataset.
  \item
    This variable could also be useful to see when the \textbf{seasonal
    effect} took place as we only went in the morning during summer
    because of the heat while we went later and for longer times in the
    field to do the box experiment in winter
  \item
    For now, the values in ``Time'' are all on the same (wrong) day
    which is the \textbf{31st of December}
  \item
    Note: I first did not intend to keep \textbf{Time} in Bex but I
    needed this variable to see the order of the trails within a day. I
    finally decided to keep it.
  \end{itemize}
\item
  Data : chr ``Data'' is coded as \textbf{character}

  \begin{itemize}
  \tightlist
  \item
    It describes \textbf{the type of data} being recorded in the
    software \textbf{cybertracker}. We installed the software on tablets
    to record the different behaviors of vervet monkeys in our research
    center
  \item
    In our case, my data was recorded in cybertracker as \textbf{Box
    Experiment} as we created a form specifically for this experiment
  \item
    For this reason we can remove this column since the information it
    contains is unecessary and redundant
  \end{itemize}
\item
  Group : chr The data is coded in r as a \textbf{character}

  \begin{itemize}
  \tightlist
  \item
    It describes the \textbf{group of monkey} in which we did the trial
  \item
    I will keep this column to see the amount of trials that we did in
    the 3 group of monkeys which are Baie-Dankie \textbf{(BD)}, Ankhase
    \textbf{(AK)}, and Noha \textbf{(NH)}
  \end{itemize}
\item
  GPSS : num ``GPSS'' is coded as \textbf{numerical}

  \begin{itemize}
  \tightlist
  \item
    It gives the \textbf{south coordinates} in which we started the
    experiment
  \item
    I do not plan to use coordinates nor look at locations so I will
    remove this acolumn
  \end{itemize}
\item
  GPSE : num ``GPSE'' is coded in as \textbf{numerical}

  \begin{itemize}
  \tightlist
  \item
    It gives the \textbf{east coordinates} in which we started the
    experiment
  \item
    I do not plan to use coordinates nor look at locations so I will
    remove this column
  \end{itemize}
\item
  \textbf{MaleID} : chr ``MaleID'' is coded as \textbf{character}

  \begin{itemize}
  \tightlist
  \item
    It indicates the \textbf{name of the male involved in the trial}
  \item
    I plan to use this to see how factors related to the individual may
    influence the experiment (age, sex, rank)
  \item
    It will also help me see which behaviour was displayed by each
    individuals (here males)
  \end{itemize}
\item
  \textbf{FemaleID} : chr ``FemaleID'' is coded as \textbf{character}

  \begin{itemize}
  \tightlist
  \item
    It indicates the \textbf{name of the female involved in the trial}
  \item
    I plan to use this variable in the same way as ``Male ID''
  \item
    It will also help me see which behaviour was displayed by each
    individuals (here females)
  \end{itemize}
\item
  \textbf{Male placement corn}: dbl ``Male placement corn is coded in r
  as \textbf{double}

  \begin{itemize}
  \item
    It gives the \textbf{amount of corn given to the male of the dyad
    before the trials}
  \item
    Within a session it happened that we gave more placement corn to
    attract the monkeys again to the boxes. This lead to an update of
    the number in the same session. The number found at the end of the
    session is the total placement corn an individual has received
  \item
    I will fuse this column with \textbf{male corn} as the data has been
    separated between these two variables. This is due to a mistake when
    creating the original box experiment form in cybertracker
  \item
    This variable could be related to the level of motivation of a
    monkey but as it is not directly related to my hypothesis I may not
    use this column. I will re-consider the use of this column later on
  \item
    In regards of this possibility I will change the format of the
    variable to numerical
  \end{itemize}
\item
  \textbf{MaleCorn} : dbl ``MaleCorn'' is coded in r as \textbf{double}

  \begin{itemize}
  \tightlist
  \item
    It gives the same information as in \textbf{male placement corn}
  \item
    I will import the values from ``male placement corn'' into this one
  \item
    I will change the format of the variable to numerical
  \end{itemize}
\item
  \textbf{FemaleCorn} : dbl The data is coded in r as \textbf{double}

  \begin{itemize}
  \tightlist
  \item
    It gives the \textbf{amount of corn given to the female of the dyad
    before the trials}
  \item
    It works in the same way as ``male placement corn''/``MaleCORN''
  \item
    I will change the format of the variable to numerical
  \end{itemize}
\item
  \textbf{DyadDistance} : chr The data is coded in r as
  \textbf{character}

  \begin{itemize}
  \tightlist
  \item
    It gives the \textbf{distance for each trial} that we have done with
    the dyads.
  \item
    The trial number 1 for each dyad was at 5 meters.
  \item
    The maximum was around 10 m while the minimum is 0
  \item
    We will have to remove the ``m'' for meters in order to have a
    numerical variable instead of character
  \item
    Also, since the very first trials per dyad can be considered as a
    kind of learning phase, i may remove the \textbf{15 first trials}
    that were made for each dyad
  \end{itemize}
\item
  \textbf{DyadResponse} : chr The data is coded in r as
  \textbf{character}

  \begin{itemize}
  \tightlist
  \item
    It indicated which \textbf{behaviour was produced by the dyad's
    during each trial}
  \item
    The different behaviours were: \textbf{Distracted}, \textbf{Female
    aggress male}, \textbf{Male aggress female}, \textbf{Intrusion},
    \textbf{Loosing interest}, \textbf{Not approaching},
    \textbf{Tolerance} and \textbf{Other}
  \item
    I will change the columns associated to each behavior
    (i.e.~Response) of \textbf{DyadResponse} into dichotomic variables
    in order to see the frequency of each behaviour
  \item
    This will allow me to see which behavior occurred more ,and
    behavioural differences could be found between dyads
  \item
    As multiple response could occur within the same trial, multiple
    behaviors can be found in a single cell. I will create a hierarchy
    to reduce the amount of behaviors assigned to each trial (if there
    is more than one). This will also be complemented with the
    information found in the column \textbf{remarks}

    \begin{enumerate}
    \def\labelenumi{\arabic{enumi}.}
    \tightlist
    \item
      correct any mistakes (ex. if tolerance and aggression are together
      aggression\textgreater tolerance)
    \item
      assign as few labels per trial
    \item
      get a better View and understanding of the data and the most
      common behaviours produced by each dyad
    \item
      create variables that can complement the behaviour found (ex. not
      approaching + looks at partner would be looks at partner + a new
      variable called hesistant to see when the did not come but look at
      the other individual / )
    \end{enumerate}
  \item
    Projection of the hierarchy (changes will be made)

    \begin{itemize}
    \item
      Create a table with each combination existing
    \item
      Decide what is more important
    \item
      Ex:

      \begin{itemize}
      \tightlist
      \item
        Aggression \textgreater{} Tolerance
      \item
        Tolerance \textgreater{} Not approaching -\textgreater{} Create
        a variable called hesistant in addtion to the tolerance count to
        see frequency of tolerance behaviour that happened after
        \textgreater{} 1min
      \item
        Tolerance \textgreater{} Loosing interest
      \item
        Tolerance \textgreater{} Intrusion
      \item
        Not approaching = looking box but not coming while Loosing
        interest = not paying attention to the box
      \item
        Intrusion \textgreater{} Loosing interest
      \item
        Intrusion \textgreater{} Not approaching
      \item
        Not approaching \textgreater{} Looks at partner
      \item
        We can code every look at partner as no approaching and keep the
        count of looks at partner as additional information
      \item
        Not approaching \textgreater?\textgreater{} Loosing interest ?
        !!
      \item
        Define distracted
      \item
        Not approaching \textgreater{} Distracted
      \item
        Aggression \textgreater{} Not approaching
      \item
        Other \textgreater{} Look case by case and categorize depending
        of behavior
      \item
        Remarks may be used for the same reason
      \end{itemize}
    \end{itemize}
  \end{itemize}
\item
  \textbf{OtherResponse} : chr ``The data''OtherResponse'' is coded as
  \textbf{character}

  \begin{itemize}
  \tightlist
  \item
    It describes \textbf{any behaviour that is different from the ones
    found in Dyad Response} (meaning ≠ tolerance, aggression, intrusion,
    loosing interest, not approaching, distracted, looks at partner that
    where categorized as \textbf{other})
  \item
    I will have to look at every \textbf{OtherResponse} and rename each
    entry in one of the response already if existing. I will proceed
    case by case.
  \item
    If I want to do an intermediate manipulation I may rename every NA
    in ``OtherResponse'' into \textbf{Response} to see the amount of
    case to treat and how many occurrences seem to not fit in the
    categories of ``DyadResponse''
  \end{itemize}
\item
  \textbf{Audience} : chr ``Audience'' is in r as \textbf{character}

  \begin{itemize}
  \tightlist
  \item
    It gives the \textbf{names of the individuals in the audience}
  \item
    I would like to use it to see the \textbf{amount of audience (big vs
    small)} and the \textbf{dominance level of the audience (high vs
    low)}
  \item
    I will create a variable called \textbf{NAudience} to see hoy many
    individuals are in the audience for each trial
  \item
    After calculating the elo ratings of the individuals using another
    dataset (Life history), I will create a dichotomic variable called
    \textbf{RankAudience} to see effects related to rank with the effect
    of audience
  \end{itemize}
\item
  \textbf{IDIndividual1} : chr ``IDIndividual1'' is coded in r as
  \textbf{character}

  \begin{itemize}
  \tightlist
  \item
    It gives the \textbf{names of the individuals that did not approach,
    showed aggression, got distracted or lost interest} during a trial
  \item
    I will have to look at it to see how often these behaviors occurred
  \item
    I will consider how to use this variable during the cleaning of the
    data
  \end{itemize}
\item
  \textbf{IntruderID} : chr ``IndtruderID'' is coded as
  \textbf{character}

  \begin{itemize}
  \tightlist
  \item
    It gives the \textbf{name of the individual that intruded the
    experiment during a trial}
  \item
    Intrusion could mean, invade the space of the experiment and
    interact with one of our individual, steal the food, show agnostic
    behavior, stand in very close proximity of the dyad's individuals
  \end{itemize}
\item
  \textbf{Remarks} : chr The data is coded in r as \textbf{character}

  \begin{itemize}
  \tightlist
  \item
    It gives either additional information concerning the experiment
    when unusual behaviors occurred , mistakes that needed to be
    corrected or details that we wanted to record in case we would need
    them
  \end{itemize}
\item
  Observers :chr The data is coded in r as \textbf{character}

  \begin{itemize}
  \tightlist
  \item
    It gives the \textbf{names of the observers during the experiment}
  \item
    We will not use this data as we do not look at the effect that an
    experimenter would have on the monkeys
  \item
    (Should I still look at an effect of the amount of
    experimenter?\ldots maybe better for detailled analysis of our
    study)
  \end{itemize}
\item
  DeviceID :chr ``The data''DeviceID'' is coded in r as
  \textbf{character}

  \begin{itemize}
  \tightlist
  \item
    It gives the \textbf{name of the device/tablet} used to record the
    data during the experiment
  \item
    We will not use this data either
  \end{itemize}
\end{itemize}

\hypertarget{treating-missing-data}{%
\section{2. Treating missing data}\label{treating-missing-data}}

\hypertarget{creating-a-new-dataframe---bex}{%
\subsection{2.1. Creating a new dataframe -
Bex}\label{creating-a-new-dataframe---bex}}

\begin{itemize}
\item
  Since I do not want to work with the whole dataset, I'm gonna select
  the variables of interest using the function \textbf{select}
\item
  I will keep Time, Date, Group, MaleID, FemaleID, MaleCorn, Male
  placement corn, FemaleCorn, DyadDistance, DyadResponse, OtherResponse,
  Audience, IDIndividual1, IntruderID, Remarks
\end{itemize}

\begin{verbatim}
## Rows: 2,795
## Columns: 15
## $ Time                  <dttm> 1899-12-31 09:47:50, 1899-12-31 09:50:07, 1899-~
## $ Date                  <dttm> 2022-09-27, 2022-09-27, 2022-09-27, 2022-09-27,~
## $ Group                 <chr> "Baie Dankie", "Baie Dankie", "Baie Dankie", "Ba~
## $ MaleID                <chr> "Nge", "Nge", "Nge", "Nge", "Nge", "Nge", "Nge",~
## $ FemaleID              <chr> "Oerw", "Oerw", "Oerw", "Oerw", "Oerw", "Oerw", ~
## $ MaleCorn              <dbl> 3, 3, 3, 3, 3, 3, 3, 3, 3, 3, 3, 3, 3, 3, 3, 3, ~
## $ `Male placement corn` <dbl> NA, NA, NA, NA, NA, NA, NA, NA, NA, NA, NA, NA, ~
## $ FemaleCorn            <dbl> NA, NA, NA, NA, NA, NA, NA, NA, NA, NA, NA, NA, ~
## $ DyadDistance          <chr> "2m", "2m", "1m", "1m", "0m", "0m", "0m", "0m", ~
## $ DyadResponse          <chr> "Tolerance", "Tolerance", "Tolerance", "Toleranc~
## $ OtherResponse         <chr> NA, NA, NA, NA, NA, NA, NA, NA, NA, NA, NA, NA, ~
## $ Audience              <chr> "Obse; Oup; Sirk", "Obse; Oup; Sirk", "Oup; Sirk~
## $ IDIndividual1         <chr> NA, NA, NA, NA, NA, NA, NA, NA, NA, NA, NA, NA, ~
## $ IntruderID            <chr> NA, NA, NA, NA, NA, NA, NA, NA, NA, NA, NA, "Sey~
## $ Remarks               <chr> NA, NA, "Nge box did not open because of the bat~
\end{verbatim}

\hypertarget{merging-male-placement-corn-and-malecorn}{%
\subsubsection{2.1.1 Merging Male placement corn and
MaleCorn}\label{merging-male-placement-corn-and-malecorn}}

\begin{itemize}
\tightlist
\item
  I want to process all the missing data in Bex. But before, I will
  merge the column \textbf{MaleCorn} and \textbf{Male placement corn} as
  the data of both columns is supposed to be together under ``MaleCorn''
\item
  Looking manually in the Bex table it seems that very few data is in
  \textbf{MaleCorn} while most of it seems to be in \textbf{Male
  placement corn}
\item
  Every time there is a missing value in Male placement corn we can see
  a value in Male Corn, I will then create a new variable MaleCorn where
  every time that there is NA in male placement corn the value will be
  taken in MaleCornOld (previous malecorn). If there is no NA it will
  take the value of ´Male placement corn´
\item
  I will first \textbf{rename MaleCorn to MaleCornOld}, then
  \textbf{check the amount of NA's} and then \textbf{merge
  ``MaleCornOld'' and ``male placement corn''} into the \textbf{new
  variable ``MaleCorn''}
\end{itemize}

\begin{verbatim}
## Number of rows with common NAs in MaleCornOld and 'Male placement corn': 1499 
## Number of occurrences of 0 in MaleCorn: 1499 
## Number of remaining NA values in MaleCorn: 0
\end{verbatim}

\begin{itemize}
\item
  I have found \textbf{1499 NA in common} between MaleCornOld and `male
  placement corn', \textbf{1609 NA in Male placement corn} and
  \textbf{2685 in MaleCorn old}
\item
  For the \textbf{merge of MaleCornOld and Male placement corn}, I used
  different conditions: 1.In this code, a new variable MaleCorn is
  created. If there is a missing value in Male placement corn, it takes
  the corresponding value from MaleCornOld; otherwise, it takes the
  value from Male placementcorn. 2.If there are no value in both
  MaleCornOld and Male placement corn (NA,NA) for a given row, I would
  like the code to display 0 as it means that no placement was given
\item
  In this way, I should not loose any data, minimize the mistakes and
  already transform the NA's of this variable into a number which will
  remove the remaining NA's which are meant to be 0
\item
  After the merge I found that there were \textbf{no NA's remaining} in
  the \textbf{``New'' Male Corn} and that \textbf{1499 0's} where found
  in the column which \textbf{corresponds to the amount of common NA's
  found previously} between the \textbf{``Old'' Male Corn} and
  \textbf{male placement corn}
\end{itemize}

\hypertarget{cleaning-femalecorn}{%
\subsubsection{2.1.2 Cleaning FemaleCorn}\label{cleaning-femalecorn}}

\begin{verbatim}
## Number of remaining NA values in FemaleCorn: 0
\end{verbatim}

\hypertarget{cleaning-variables-with-missing-data}{%
\subsubsection{2.2 Cleaning variables with missing
data}\label{cleaning-variables-with-missing-data}}

\begin{itemize}
\item
  Now in order to see where are located the missing points in the data,
  I'm going to \textbf{print} the variables \textbf{with and without
  NA's}
\item
  The function \textbf{sapply} is used to apply the function
  \textbf{sum} for NA's to each column of the data frame, so each
  variable
\end{itemize}

\begin{verbatim}
## Variables with Missing Data:
\end{verbatim}

\begin{longtable}[]{@{}lr@{}}
\toprule
& x \\
\midrule
\endhead
MaleID & 19 \\
FemaleID & 60 \\
DyadDistance & 33 \\
DyadResponse & 47 \\
OtherResponse & 2758 \\
Audience & 924 \\
IDIndividual1 & 2143 \\
IntruderID & 2737 \\
Remarks & 2181 \\
\bottomrule
\end{longtable}

\begin{verbatim}
## Variables with No Missing Data:
\end{verbatim}

\begin{longtable}[]{@{}lr@{}}
\toprule
& x \\
\midrule
\endhead
Time & 0 \\
Date & 0 \\
Group & 0 \\
FemaleCorn & 0 \\
MaleCorn & 0 \\
\bottomrule
\end{longtable}

\begin{itemize}
\item
  We can see that out of the 14 variables we have in \textbf{Bex} we
  have \textbf{9 variables with missing data} which are \textbf{Male ID,
  Female ID, DyadDistance, DyadResponse, OtherResponse, Audience,
  IDIndividual1, IntruderID, Remarks}: I will proceed to clean these
  variables one by one
\item
  MaleID 19
\item
  FemaleID 60
\item
  DyadDistance 33
\item
  DyadResponse 47
\item
  OtherResponse 2758
\item
  Audience 924
\item
  ID Individual1 2143
\item
  IntruderID 2737
\item
  Remarks 2181
\item
  Before making treating the NA's in the dataset I will make a backup of
  the data at this point:
\end{itemize}

\hypertarget{treating-variables-with-missing-data}{%
\subsubsection{2.3 Treating variables with missing
data}\label{treating-variables-with-missing-data}}

\hypertarget{cleaning-remarks---2181-nas}{%
\paragraph{2.3.1 Cleaning ``Remarks'' - (2181
NA's)}\label{cleaning-remarks---2181-nas}}

\begin{itemize}
\item
  Since most of the time we did not have any remarks it is
  understandable that this variable contains 2181 NA's out of 2795 rows
\item
  I will first transform every missing data in the column Remark into
  \textbf{No Remarks} and then check that the amount of ``No remarks''
  found
\item
  After the changes we can effectively see that we have \textbf{2181
  ``No Remarks''} and we have no missing data left in that column, I
  will treat this column by hand once all the NA's have been removed
  from the dataset
\end{itemize}

\begin{verbatim}
## Number of 'No Remarks' in the 'Remarks' column: 2181
\end{verbatim}

\begin{verbatim}
## 
## No Remarks    Remarks 
##       2181        614
\end{verbatim}

\hypertarget{cleaning-intruder-id---2737-nas}{%
\subsubsection{2.3.2 Cleaning ``Intruder ID'' - (2737
NA's)}\label{cleaning-intruder-id---2737-nas}}

\begin{itemize}
\tightlist
\item
  \textbf{Intruder ID} is a variable that contains the \textbf{name of
  the individuals that made and intrusion during a trial}.
\item
  If more than one individual intruded, his name may be in the comments,
  which I will check when treating the data from this column
\item
  Because nothing was entered when there was no intrusion, I will
  replace every NA's by \textbf{No Intrusion}
\item
  Also, I will use a function to create a new dichotomic variable called
  \textbf{Intrusion}. Every time there is a value in IntruderID, it
  should display 1 (Yes), if not a 0 (No intrusion)
\end{itemize}

\begin{verbatim}
## Number of 'No Intrusion' in the 'Intruder ID' column after replacement: 2737
\end{verbatim}

\begin{itemize}
\tightlist
\item
  We previously had 2737 NA's in IntruderID while now we have the same
  amount of occurrences of IntruderID which shis that the transformation
  went as intended
\end{itemize}

\hypertarget{cleaning-idindividual1---2143-nas}{%
\paragraph{2.3.3 Cleaning ``IDIndividual1'' - (2143
NA's)}\label{cleaning-idindividual1---2143-nas}}

\begin{itemize}
\tightlist
\item
  IDIndividual1 is meant to report the name of the individual that did a
  behavior such as not approach, show aggression or loose interest
  during a trial
\item
  I will now replace every NA in this column by \textbf{No individual}
  and print the amount of NA's left and the amount of changes made
\end{itemize}

\begin{verbatim}
## Number of NAs replaced in IDIndividual1: 2143 
## Number of remaining NA values in IDIndividual1: 0
\end{verbatim}

\hypertarget{cleaning-audience---924-nas}{%
\subsubsection{2.6 Cleaning ``Audience'' - (924
NA'S)}\label{cleaning-audience---924-nas}}

\begin{itemize}
\tightlist
\item
  Audience is made to report every name of individuals around our dyad
  during a given trial
\item
  I will replace every NA by \textbf{No audience} as no entry means the
  absence of other individuals around
\item
  I will also create a new variable called ``Amount audience'' that will
  have to tell me how many individuals are found in the column Audience
\end{itemize}

\begin{verbatim}
## Number of changes made in 'Audience': 924
\end{verbatim}

\begin{verbatim}
## Remaining NA values in 'Audience': 0
\end{verbatim}

\hypertarget{cleaning-otherresponse---2758-nas}{%
\subsubsection{2.7 Cleaning ``OtherResponse'' - (2758
NA'S)}\label{cleaning-otherresponse---2758-nas}}

\begin{verbatim}
## Number of changes made in 'OtherResponse': 2758
\end{verbatim}

\begin{verbatim}
## Remaining NA values in 'OtherResponse': 0
\end{verbatim}

\hypertarget{cleaning-of-time}{%
\subsubsection{2.8 Cleaning of ``Time''}\label{cleaning-of-time}}

\begin{itemize}
\tightlist
\item
  Since the reading of the data is more complicated without the time,
  which was usefull to know which trial was before or after, I changed
  the code made for Bex and added \textbf{Time} in the dataframe. Since
  I will need it for the cleaning of Dyaddistance, I will now extract
  the time from the date. Even if the date is wrong as seen in the first
  output, the time is correct. As in the second output, only the time
  has been kept
\end{itemize}

\begin{verbatim}
## [1] "1899-12-31 09:47:50 UTC" "1899-12-31 09:50:07 UTC"
## [3] "1899-12-31 09:53:11 UTC" "1899-12-31 09:54:28 UTC"
## [5] "1899-12-31 09:55:19 UTC" "1899-12-31 09:56:56 UTC"
\end{verbatim}

\begin{verbatim}
## [1] "09:47:50" "09:50:07" "09:53:11" "09:54:28" "09:55:19" "09:56:56"
\end{verbatim}

\hypertarget{cleaning-dyaddistance}{%
\subsubsection{2.9 Cleaning DyadDistance}\label{cleaning-dyaddistance}}

\begin{itemize}
\tightlist
\item
  Before looking at the NA's of Dyaddistance I will remove the ``m''
  that is in front of every number to have a numerical variable
\item
  Then I will look at the location of the NA's in the data to treat them
  case by case.
\end{itemize}

\begin{verbatim}
## Warning: NAs introduits lors de la conversion automatique
\end{verbatim}

\begin{verbatim}
## # A tibble: 69 x 16
##    Time     Date                Group    MaleID FemaleID FemaleCorn DyadDistance
##    <chr>    <dttm>              <chr>    <chr>  <chr>         <dbl>        <dbl>
##  1 12:09:34 2022-09-27 00:00:00 Baie Da~ Xia    Piep              7           NA
##  2 12:13:28 2022-09-27 00:00:00 Baie Da~ Xia    Piep              7           NA
##  3 16:02:32 2022-09-15 00:00:00 Ankhase  Sho    Ginq              6           NA
##  4 10:46:33 2023-08-17 00:00:00 Baie Da~ Xia    Piep              0           NA
##  5 09:30:17 2023-07-29 00:00:00 Baie Da~ Xin    Ouli              0           NA
##  6 12:08:51 2023-07-11 00:00:00 Baie Da~ Xia    Piep              0           NA
##  7 13:30:07 2023-06-29 00:00:00 Baie Da~ Sey    Sirk              0           NA
##  8 09:54:24 2023-06-27 00:00:00 Ankhase  Sho    Ginq              0           NA
##  9 10:13:56 2023-06-23 00:00:00 Ankhase  Sho    Ginq              0           NA
## 10 09:39:04 2023-06-15 00:00:00 Ankhase  Sho    Ginq              2           NA
## # i 59 more rows
## # i 9 more variables: DyadResponse <chr>, OtherResponse <chr>, Audience <chr>,
## #   IDIndividual1 <chr>, IntruderID <chr>, Remarks <chr>, MaleCorn <dbl>,
## #   Intrusion <dbl>, AmountAudience <dbl>
## Number of NA values in DyadDistance column (using second approach): 69 
## Rows with NA values in DyadDistance column: 24, 27, 95, 492, 744, 971, 1113, 1130, 1164, 1261, 1341, 1396, 1491, 1583, 1683, 1693, 1717, 1718, 1719, 1724, 1725, 1739, 1755, 1756, 1757, 1764, 1779, 1782, 1792, 1799, 1800, 1840, 1841, 1868, 1869, 1888, 1891, 1892, 1896, 1911, 1912, 1915, 1918, 1919, 1952, 1953, 1958, 1980, 1981, 1984, 1986, 1996, 2000, 2009, 2054, 2104, 2105, 2191, 2233, 2234, 2287, 2437, 2569, 2579, 2580, 2643, 2676, 2709, 2729
\end{verbatim}

\begin{itemize}
\item
  We have 69 missing values in DyadDistance. I will look at each row in
  it's context as the actual distance of the box was always dependent of
  the previous trials. I will start with the bigger number as for now
  the oldest trial is at the last row while the closest one is in row 1.

  \begin{itemize}
  \tightlist
  \item
    If \textbf{tolerance} was achieved \textbf{twice in a row} =
    \textless1m
  \item
    If \textbf{aggression} (male agress female or female agress male),
    not approaching,or \textbf{loosing interest} occured =
    \textgreater1m
  \item
    If \textbf{distracted} or \textbf{intrusion} occured = same distance
  \end{itemize}

  \begin{enumerate}
  \def\labelenumi{\arabic{enumi}.}
  \tightlist
  \item
    \textbf{24} - In trial23 (0m) there was aggression then at trial24
    (1m) there was tolerance. The 24th trial is supposed to be at
    \textbf{1m}
  \item
    \textbf{27} - In trial25 (1m) there was not approaching then at
    trial26 (2m) there was tolerance.The 25th trial is supposed to be at
    \textbf{2m}
  \item
    \textbf{95} - In trial93 (2m ) there was male agress female then at
    trial 94 (3m) there was not approaching. The 95th trial is supposed
    to be at \textbf{4m}
  \item
    \textbf{492} - In trial490 (0m) there was tolerance then at trial491
    (0m) there was tolerance. The 492nd trial is supposed to be at
    \textbf{0m}
  \item
    \textbf{744} - In trial742 (3m) there was aggression then at
    trial743 (4m) there was tolerance. The 744th trial is supposed to be
    at \textbf{4m}\\
  \item
    \textbf{971} - In trial969 (0m) there was tolerance then at trial970
    (0m) there was tolerance. The 971st trial is supposed to be at
    \textbf{0m}
  \item
    \textbf{1113} - In trial1111 (2m) there was tolerance then at
    trial1112 (0m) there was tolerance. The 1113th trial is supposed to
    be at \textbf{0m}
  \item
    \textbf{1130} - In trial1128 there was another dyad so we can not
    use this cell. Then at trial1129 (3m) there was not approaching.
    Nevertheless, we don't have any DyadResponse, i will thus
    \textbf{delete this row}
  \item
    \textbf{1164} - In trial1162 (3m) there was not approaching then at
    trial1163 (3m) there was not approaching. The 1164th trial is
    supposed to be at \textbf{4m}
  \item
    \textbf{1261} - The two preivous trials were made with another Dyad.
    Also DyadResponse is not available. I will thus \textbf{delete this
    row}
  \item
    \textbf{1341} - In trial1339 (0m) there was tolerance then at
    trial1340 (0m) there was tolerance. The 1341st trial is supposed to
    be at \textbf{0m}
  \item
    \textbf{1396} - The two previous trials were made with another Dyad.
    Also DyadResponse is not available. I will thus \textbf{delete this
    row}
  \item
    \textbf{1491} - The two previous trials were made with another Dyad.
    Also DyadResponse is not available. I will thus \textbf{delete this
    row}
  \item
    \textbf{1583} - In trial1581 (2m) there was not approaching and
    intrusion then at trial1582 (2m) there was not approaching. The
    1583rd trial is supposed to be at \textbf{3m}
  \item
    \textbf{1683} - One trial only was made with tolerance (2m) but
    since there are no DyadResponse I will \textbf{delete this row}
  \item
    \textbf{1693} - The two previous trials were made with another Dyad.
    Also DyadResponse is not available. I will thus \textbf{delete this
    row}
  \item
    \textbf{1717} - The two previous trials were made with another Dyad.
    Also DyadResponse is not available. I will thus \textbf{delete this
    row}
  \item
    \textbf{1718} - The two previous trials were made with another Dyad.
    Also DyadResponse is not available. I will thus \textbf{delete this
    row}
  \item
    \textbf{1719} - The two previous trials were made with another Dyad.
    Also DyadResponse is not available. I will thus \textbf{delete this
    row}
  \item
    \textbf{1724} - The two previous trials were made with another Dyad.
    Also DyadResponse is not available. I will thus \textbf{delete this
    row}
  \item
    \textbf{1725} - The two previous trials were made with another Dyad.
    Also DyadResponse is not available. I will thus \textbf{delete this
    row}
  \item
    \textbf{1739} - The two previous trials were made with another Dyad.
    Also DyadResponse is not available. I will thus \textbf{delete this
    row}
  \item
    \textbf{1755} - since there are no DyadResponse I will
    \textbf{delete this row}
  \item
    \textbf{1756} - The two previous trials were made with another Dyad.
    Also DyadResponse is not available. I will thus \textbf{delete this
    row}
  \item
    \textbf{1757} - The two previous trials were made with another Dyad.
    Also DyadResponse is not available. I will thus \textbf{delete this
    row}
  \item
    \textbf{1764} - Since there are no DyadResponse I will
    \textbf{delete this row}
  \item
    \textbf{1779} - It seems like it was the first trial of the Dyad Pom
    Xian, if so, the distance has to be \textbf{5m}
  \item
    \textbf{1782} - The two previous trials were made with another Dyad.
    Also DyadResponse is not available. I will thus \textbf{delete this
    row}
  \item
    \textbf{1792} - Trial1791 was intrusion (4m) so this trial should be
    at \textbf{4m}
  \item
    \textbf{1799} - The two previous trials were made with another Dyad.
    Also DyadResponse is not available. I will thus \textbf{delete this
    row}
  \item
    \textbf{1800} - The two previous trials were made with another Dyad.
    Also DyadResponse is not available. I will thus \textbf{delete this
    row}
  \item
    \textbf{1840} - The two previous trials were made with another Dyad.
    Also DyadResponse is not available. I will thus \textbf{delete this
    row}
  \item
    \textbf{1841} - The two previous trials were made with another Dyad.
    Also DyadResponse is not available. I will thus \textbf{delete this
    row}
  \item
    \textbf{1868} - The two previous trials were made with another Dyad.
    Also DyadResponse is not available. I will thus \textbf{delete this
    row}
  \item
    \textbf{1869} - The two previous trials were made with another Dyad.
    Also DyadResponse is not available. I will thus \textbf{delete this
    row}
  \item
    \textbf{1888} - he two previous trials were made with another Dyad.
    Also DyadResponse is not available. I will thus \textbf{delete this
    row}
  \item
    \textbf{1891} - Since there are no DyadResponse I will
    \textbf{delete this row}
  \item
    \textbf{1892} - The two previous trials were made with another Dyad.
    Also DyadResponse is not available. I will thus \textbf{delete this
    row}
  \item
    \textbf{1896} - Since there are no DyadResponse I will
    \textbf{delete this row}
  \item
    \textbf{1911} - The two previous trials were made with another Dyad.
    Also DyadResponse is not available. I will thus \textbf{delete this
    row}
  \item
    \textbf{1912} - The two previous trials were made with another Dyad.
    Also DyadResponse is not available. I will thus \textbf{delete this
    row}
  \item
    \textbf{1915} - The two previous trials were made with another Dyad.
    Also DyadResponse is not available. I will thus \textbf{delete this
    row}
  \item
    \textbf{1918} - In trial1916 (4m) there was tolerance then at
    trial1917 (4m) there was not loosing interest The 1918th trial is
    supposed to be at \textbf{4m}
  \item
    \textbf{1919} - The two previous trials were made with another Dyad.
    Also DyadResponse is not available. I will thus \textbf{delete this
    row}
  \item
    \textbf{1952} - The two previous trials were made with another Dyad.
    Also DyadResponse is not available. I will thus \textbf{delete this
    row}
  \item
    \textbf{1953} - The two previous trials were made with another Dyad.
    Also DyadResponse is not available. I will thus \textbf{delete this
    row}
  \item
    \textbf{1958} - In trial1956 (2m) there was tolerance then at
    trial1957 (2m) there was distracted. The 1958th trial is supposed to
    be at \textbf{2m}
  \item
    \textbf{1980} - The two previous trials were made with another Dyad.
    Also DyadResponse is not available. I will thus \textbf{delete this
    row}
  \item
    \textbf{1981} - The two previous trials were made with another Dyad.
    Also DyadResponse is not available. I will thus \textbf{delete this
    row}
  \item
    \textbf{1984} - The two previous trials were made with another Dyad.
    Also DyadResponse is not available. I will thus \textbf{delete this
    row}
  \item
    \textbf{1986} - The two previous trials were made with another Dyad.
    Also DyadResponse is not available. I will thus \textbf{delete this
    row}
  \item
    \textbf{1996} - The two previous trials were made with another Dyad.
    Also DyadResponse is not available. I will thus \textbf{delete this
    row}
  \item
    \textbf{2000} - In trial1997 and 1999 (5m) there was tolerance then
    at trial1999 (5m) there was intrusion. The 2000th trial is supposed
    to be at \textbf{4m}
  \item
    \textbf{2009} - Since there are no DyadResponse I will
    \textbf{delete this row}
  \item
    \textbf{2054} - The two previous trials were made with another Dyad.
    Also DyadResponse is not available. I will thus \textbf{delete this
    row}
  \item
    \textbf{2104} - The two previous trials were made with another Dyad.
    Also DyadResponse is not available. I will thus \textbf{delete this
    row}
  \item
    \textbf{2105} - The two previous trials were made with another Dyad.
    Also DyadResponse is not available. I will thus \textbf{delete this
    row}
  \item
    \textbf{2191} - In trial2189 (1m) there was not approaching then at
    trial2190 (2m) there was not approaching. The 2191st trial is
    supposed to be at \textbf{3m}
  \item
    \textbf{2233} - In trial2231 (3m) there was not approaching then at
    trial2232 (4m) there was not approaching. The 2233rd trial is
    supposed to be at \textbf{5m}
  \item
    \textbf{2234} - The trial did not happen because they where not at
    the right distance. I will thus \textbf{delete this row}
  \item
    \textbf{2287} - Since there are no DyadResponse I will
    \textbf{delete this row}
  \item
    \textbf{2437} - Since there are no DyadResponse I will
    \textbf{delete this row}
  \item
    \textbf{2569} - In trial2567 (1m) there was tolerance then at
    trial2568 (1m) there was tolerance. The 2569th trial is supposed to
    be at (0m)
  \item
    \textbf{2579} - In trial2577 (1m) there was tolerance then at
    trial2578 (0m) there was not approaching. The 2579th trial is
    supposed to be at \textbf{1m}
  \item
    \textbf{2580} - The two previous trials were made with another Dyad.
    Also DyadResponse is not available. I will thus \textbf{delete this
    row}
  \item
    \textbf{2643} - Since there are no DyadResponse I will
    \textbf{delete this row}
  \item
    \textbf{2676} - In trial2674 (1m) there was tolerance then at
    trial2675 (0m) there was tolerance. The 2676th trial is supposed to
    be at \textbf{0m}
  \item
    \textbf{2709} - In trial2707 (2m) there was tolerance then at
    trial2708 (2m) there was tolerance. The 2709th trial is supposed to
    be at \textbf{1m}
  \item
    \textbf{2729} - In trial2727 (3m) there wastolerance then at
    trial2728 (2m) there was tolerance. The 2729th trial is supposed to
    be at \textbf{2m}
  \end{enumerate}
\item
  Now that I have looked at each missing line and saw which ones to
  keep, I decided to create a new variable called \textbf{Distance}. I
  will also to create a new variable called \textbf{No trial}.
\item
  For the variable \textbf{Distance} I will replace each row where there
  was missing data with a value and I will delete the ones where no
  values could be assigned. This will allow me to have no missing data
  and find a number to each trial that has been done
\item
  Before making the changes i'm gonna make a backup called
  \textbf{BackupbeforeDistanceNA}
\end{itemize}

\begin{verbatim}
## Number of NA's in DyadDistance after replacements and deletions: 1 
## Data size after deletions: 2748
\end{verbatim}

\begin{verbatim}
## Row index with NA in DyadDistance: 1925
\end{verbatim}

*It seems that there is still the row 1925 with an NA in DyadDistance

\begin{enumerate}
\def\labelenumi{\arabic{enumi}.}
\setcounter{enumi}{69}
\tightlist
\item
  \textbf{1925} - In trial1923 (2m) there was distracted then at
  trial1924 (2m) there was tolerance. The 2725th trial is supposed to be
  at \textbf{2m}
\end{enumerate}

\begin{verbatim}
## Row index with NA in DyadDistance:
\end{verbatim}

*In this modification, I added a check to see if the columns Dyadistance
and Distance already exist in your dataframe (Bex). If they do, it
prints a message saying that the modification has already been applied,
and no changes are made. If they don't exist, it proceeds with the
modifications. This way, running the code multiple times won't cause
redundant changes.

\begin{itemize}
\tightlist
\item
  \textbf{24 values} were inserted in \textbf{Distance} to replace the
  NA's where the distance could be found by looking at the previous
  rows. The \textbf{46 remaining NA's} were then \textbf{removed} from
  Distance \textbf{leaving 0 remaining NA in the variable ``Distance''}
\end{itemize}

\hypertarget{cleaning-female-and-male-id}{%
\subsubsection{2.10 Cleaning Female and Male
ID}\label{cleaning-female-and-male-id}}

\begin{itemize}
\item
  Before cleaning Female and Male ID, here is a list of every dyad of
  the box experiment and their respective groups. This will help us find
  the missing names when only one individual is missing out of the duo
  (either male or female):

  \begin{enumerate}
  \def\labelenumi{\alph{enumi}.}
  \item
    Sirk \& Sey - BD
  \item
    Ouli \& Xin - BD
  \item
    Piep \& Xia - BD
  \item
    Oerw \& Nge - BD
  \item
    Oort \& Kom - BD
  \item
    Ginq \& Sho - AK
  \item
    Ndaw \& Buk - Ak
  \item
    Xian \& Pom - AK
  \item
    Guat \& Pom - Ak
  \end{enumerate}
\item
  Note that the 4 letter codes correspond to the femaleID, the 3 letter
  codes to the males ID and the 2 letter codes to the group name of the
  monkeys
\item
  I need to check where are the NA's in both FemaleID and Male ID by
  looking at the rows where data is missing. Since every trial was made
  with a Dyad and never with an single individual, treating these two
  columns together makes more sense. If both individuals are missing I
  may have to delete the row.
\end{itemize}

\begin{verbatim}
## Row numbers with missing values in FemaleID:  865 866 867 868 869 870 871 872 873 874 875 876 877 878 879 1693 1694 1695 1696 1697 1698 1699 1700 1701 1702 1703 1704 1705 1706 1707 1708 1709 1710 1808 1809 1810 1811 1812 1813 1814 1815 1816 1817 1818 1819 1820 1884 1885 2619 2620 2621 2622 2623 2624 2625 2626 2627 2628 2629
\end{verbatim}

\begin{verbatim}
## Number of missing values in FemaleID:  59
\end{verbatim}

\begin{verbatim}
## Row numbers with missing values in MaleID:  1693 1694 1695 1696 1697 1698 1699 1700 1701 1702 1703 1704 1705 1706 1707 1708 1709 1710
\end{verbatim}

\begin{verbatim}
## Number of missing values in MaleID:  18
\end{verbatim}

\begin{verbatim}
## Number of rows with missing values in both FemaleID and MaleID:  18
\end{verbatim}

\begin{verbatim}
## Row numbers with missing values in both FemaleID and MaleID:  1693, 1694, 1695, 1696, 1697, 1698, 1699, 1700, 1701, 1702, 1703, 1704, 1705, 1706, 1707, 1708, 1709, 1710
\end{verbatim}

\begin{verbatim}
## Number of missing values in FemaleID not in MaleID:  41
\end{verbatim}

\begin{verbatim}
## Row numbers with missing values in FemaleID not in MaleID:  865, 866, 867, 868, 869, 870, 871, 872, 873, 874, 875, 876, 877, 878, 879, 1808, 1809, 1810, 1811, 1812, 1813, 1814, 1815, 1816, 1817, 1818, 1819, 1820, 1884, 1885, 2619, 2620, 2621, 2622, 2623, 2624, 2625, 2626, 2627, 2628, 2629
\end{verbatim}

\begin{itemize}
\item
  \textbf{FemaleID} has \textbf{41 NA's} while they are \textbf{18 NA's}
  in \textbf{Male ID}
\item
  In these missing data, we have \textbf{18 NA's} that are in common
  between FemaleID and MaleID which represents the totality of the
  missing values in MaleID
\item
  All the missing data in MaleID are found in consecutive rows, from row
  \textbf{1693} to row \textbf{1710} and are from the group Noha (NH) on
  the 19th of april 2023. We can also see that trials had bee made in
  the same day, and looking at the time of the experiment, the previous
  trials made and the audience we can see that these NA's in female and
  male ID we can asses that the individuals involved were \textbf{Xian}
  for the female ID and \textbf{Pom} for the \textbf{MaleID}. I will
  thus replace these values using a condtion. These NA's in Noha (Trial
  1693 to 1710) are the only NA's that MaleID has and are the only NA's
  of female ID in Noha. I will thus replace every NA of \textbf{MaleID
  NA in Noha} with \textbf{Pom} and every \textbf{Female ID NA in Noha}
  with \textbf{Xian}
\end{itemize}

\begin{verbatim}
## Number of remaining NA values in MaleID after replacement:  0
\end{verbatim}

\begin{verbatim}
## Number of remaining NA values in FemaleID after replacement:  41
\end{verbatim}

\begin{verbatim}
## Number of rows with missing values in both MaleID and FemaleID after replacement:  0
\end{verbatim}

\begin{itemize}
\item
  In order to clean FemaleID, I will use the data from the now complete
  MaleID. I will use conditions stating that depending which name is
  found in MaleID when there is an NA in FemaleID, a certain name will
  have to replace the NA in female ID
\item
  Before automating the process I will check manually the data to see if
  they are any exceptions or mistakes
\end{itemize}

\begin{verbatim}
## Rows with missing values in FemaleID:
\end{verbatim}

\begin{verbatim}
## # A tibble: 41 x 16
##    Time     Date                Group   MaleID FemaleID FemaleCorn DyadDistance
##    <chr>    <dttm>              <chr>   <chr>  <chr>         <dbl>        <dbl>
##  1 09:31:55 2023-07-22 00:00:00 Ankhase Buk    <NA>              7            1
##  2 09:33:14 2023-07-22 00:00:00 Ankhase Buk    <NA>              7            1
##  3 09:34:07 2023-07-22 00:00:00 Ankhase Buk    <NA>              7            0
##  4 09:34:51 2023-07-22 00:00:00 Ankhase Buk    <NA>              7            0
##  5 09:36:59 2023-07-22 00:00:00 Ankhase Buk    <NA>              7            0
##  6 09:38:13 2023-07-22 00:00:00 Ankhase Buk    <NA>              7            1
##  7 09:39:26 2023-07-22 00:00:00 Ankhase Buk    <NA>              7            0
##  8 09:41:11 2023-07-22 00:00:00 Ankhase Buk    <NA>              0            0
##  9 09:42:17 2023-07-22 00:00:00 Ankhase Buk    <NA>              0            0
## 10 09:44:06 2023-07-22 00:00:00 Ankhase Buk    <NA>              0            1
## # i 31 more rows
## # i 9 more variables: DyadResponse <chr>, OtherResponse <chr>, Audience <chr>,
## #   IDIndividual1 <chr>, IntruderID <chr>, Remarks <chr>, MaleCorn <dbl>,
## #   Intrusion <dbl>, AmountAudience <dbl>
\end{verbatim}

If there is NA in femaleID, we will replace the value with - Sirk if
MaleID is Sey - Ouli if MaleID is Xin - Piep if MaleID is Xia - Oerw if
MaleID is Nge - Oort if MaleID is Kom - Ginq if MaleID is Sho - Ndaw if
MaleID is Buk

\begin{verbatim}
## # A tibble: 20 x 3
##    MaleID FemaleID Count
##    <chr>  <chr>    <int>
##  1 Xia    Piep       576
##  2 Sey    Sirk       557
##  3 Kom    Oort       338
##  4 Sho    Ginq       278
##  5 Pom    Xian       259
##  6 Buk    Ndaw       245
##  7 Xin    Ouli       159
##  8 Nge    Oerw       153
##  9 Piep   Xia         35
## 10 Oort   Kom         29
## 11 Ouli   Xin         27
## 12 Oerw   Nge         19
## 13 Sirk   Sey         17
## 14 Buk    <NA>        15
## 15 Sey    <NA>        13
## 16 Nge    <NA>        11
## 17 Buk    Ginq         6
## 18 Pom    Guat         5
## 19 Xin    Oort         4
## 20 Kom    <NA>         2
\end{verbatim}

\begin{verbatim}
## Number of NA values in MaleID:  0
\end{verbatim}

\begin{verbatim}
## Number of NA values in FemaleID:  0
\end{verbatim}

\begin{itemize}
\tightlist
\item
  After the use of the conditions in FemaleID I could see the changes
  where successfully done and that 0 NA's are remaining in both FemaleID
  and MaleID
\end{itemize}

\hypertarget{dyad-response-7}{%
\subsubsection{2.12 Dyad Response (7)}\label{dyad-response-7}}

\begin{itemize}
\tightlist
\item
  The last variable I still have to treat for NA's is DyadResponse.
  Before treating the NA's we had 47 NA's we know that we treated most
  of them we have only 7 remaining. These NA's can be found at the rows
  \textbf{871, 1163, 1219, 1339, 1579,1888 and 1962}
\end{itemize}

\begin{verbatim}
## Rows with missing values in DyadResponse:  871, 1163, 1219, 1339, 1579, 1888, 1962
\end{verbatim}

\begin{verbatim}
## Lines with missing values in DyadResponse:
\end{verbatim}

\begin{verbatim}
## # A tibble: 7 x 16
##   Time     Date                Group     MaleID FemaleID FemaleCorn DyadDistance
##   <chr>    <dttm>              <chr>     <chr>  <chr>         <dbl>        <dbl>
## 1 09:39:26 2023-07-22 00:00:00 Ankhase   Buk    Ndaw              7            0
## 2 10:13:56 2023-06-23 00:00:00 Ankhase   Sho    Ginq              0            4
## 3 08:34:45 2023-06-17 00:00:00 Baie Dan~ Kom    Oort              3            2
## 4 08:54:12 2023-06-09 00:00:00 Baie Dan~ Xia    Piep              1            0
## 5 13:35:08 2023-05-03 00:00:00 Baie Dan~ Kom    Oort              5            3
## 6 13:27:30 2023-01-18 00:00:00 Ankhase   Buk    Ndaw              5            4
## 7 08:36:49 2022-12-13 00:00:00 Baie Dan~ Kom    Oort              8            4
## # i 9 more variables: DyadResponse <chr>, OtherResponse <chr>, Audience <chr>,
## #   IDIndividual1 <chr>, IntruderID <chr>, Remarks <chr>, MaleCorn <dbl>,
## #   Intrusion <dbl>, AmountAudience <dbl>
\end{verbatim}

\begin{itemize}
\item
  Row 871: The previous row was tolerance at 1m and the next tolerance
  at 0 which means that the row 871 should be \textbf{Tolerance} for
  DyadResponse
\item
  Row 1163: The value can not be found from the other rows so I will
  \textbf{delete} row 1163
\item
  Row 1219: The previous row was not approaching at 2m and the next is
  tolerance at 2m and tolerance at 1m, which means that the row 1219
  should be \textbf{Tolerance} for DyadResponse
\item
  Row 1339: The previous row was tolerance at 0m while the next one was
  tolerance at 0m, which means that the tow 1339 should be
  \textbf{Tolerance} for DyadResponse
\item
  Row 1579: The value can not be found from the other rows so I will
  \textbf{delete} row 1579
\item
  Row 1888: The value can not be found from the other rows so I will
  \textbf{delete} row 1888
\item
  Row 1962: The value can not be found from the other rows so I will
  \textbf{delete} row 1962
\end{itemize}

\begin{verbatim}
## Number of remaining NA values in DyadResponse:  0
\end{verbatim}

\hypertarget{final-check-remaing-nas-in-bex}{%
\subsubsection{2.13 Final check: remaing NA's in
Bex?}\label{final-check-remaing-nas-in-bex}}

\begin{verbatim}
## Final check of NA values in Bex:
\end{verbatim}

\begin{verbatim}
##           Time           Date          Group         MaleID       FemaleID 
##              0              0              0              0              0 
##     FemaleCorn   DyadDistance   DyadResponse  OtherResponse       Audience 
##              0              0              0              0              0 
##  IDIndividual1     IntruderID        Remarks       MaleCorn      Intrusion 
##              0              0              0              0              0 
## AmountAudience 
##              0
\end{verbatim}

\hypertarget{correction-and-creation-of-new-variables}{%
\section{3. Correction and creation of New
Variables}\label{correction-and-creation-of-new-variables}}

\hypertarget{making-a-backup-of-bex}{%
\subsubsection{3.1 Making a backup of
Bex}\label{making-a-backup-of-bex}}

\begin{itemize}
\tightlist
\item
  Before making new changes I will make a backup of my dataset at this
  point
\end{itemize}

\hypertarget{treating-remarks-before-processing-with-new-modifications}{%
\subsubsection{3.2 Treating Remarks before processing with new
modifications}\label{treating-remarks-before-processing-with-new-modifications}}

\begin{itemize}
\item
  Since I have removed all the missing data from the different columns,
  I now have to correct potential mistakes that can be found and create
  new variables to be able to manipulate better my data.
\item
  Since the column remarks contains corrections and additional
  information, I will treat it now
\item
  Before that lets check how many remarks we have in our dataset, how
  many of the main keywords we can find and make a visual representation
  of it
\end{itemize}

\hypertarget{vizualization-of-the-remarks-keywords}{%
\paragraph{3.2.2 Vizualization of the Remarks
Keywords}\label{vizualization-of-the-remarks-keywords}}

\begin{itemize}
\tightlist
\item
  Before making any changes I will make a count of the total amount of
  remarks and a count and barplot of the main keywords in the column to
  see in which proportion they are found. It has to be noted that some
  of the words are used in different contexts and have different
  meaning. This is why I will clean them manually
\end{itemize}

\begin{verbatim}
## Number of 'No Remarks' entries:  2139
\end{verbatim}

\begin{verbatim}
## Number of actual remarks entries:  603
\end{verbatim}

\begin{itemize}
\tightlist
\item
  There will be 599 entries I will have to treat manually in the Excel
  Spreadsheet for the Remarks
\end{itemize}

\includegraphics{BEX2223_files/figure-latex/Remarks Exploration-1.pdf}

\begin{verbatim}
## Total number of keyword occurrences in the Barplot:  822
\end{verbatim}

\hypertarget{exporting-of-bex}{%
\paragraph{3.2.3 Exporting of Bex}\label{exporting-of-bex}}

\begin{itemize}
\tightlist
\item
  I will now export the dataset and treat manually treat the remarks in
  an Excel spreadsheet before uploading it again and creating a new
  dataframe. I will also print a Glimpse of Bex to have information
  before the manual changes
\end{itemize}

\begin{verbatim}
## Glimpse of the Bex Before treating Remarks:
\end{verbatim}

\begin{verbatim}
## Rows: 2,742
## Columns: 16
## $ Time           <chr> "09:47:50", "09:50:07", "09:53:11", "09:54:28", "09:55:~
## $ Date           <dttm> 2022-09-27, 2022-09-27, 2022-09-27, 2022-09-27, 2022-0~
## $ Group          <chr> "Baie Dankie", "Baie Dankie", "Baie Dankie", "Baie Dank~
## $ MaleID         <chr> "Nge", "Nge", "Nge", "Nge", "Nge", "Nge", "Nge", "Nge",~
## $ FemaleID       <chr> "Oerw", "Oerw", "Oerw", "Oerw", "Oerw", "Oerw", "Oerw",~
## $ FemaleCorn     <dbl> 0, 0, 0, 0, 0, 0, 0, 0, 0, 0, 0, 0, 0, 0, 0, 7, 7, 7, 7~
## $ DyadDistance   <dbl> 2, 2, 1, 1, 0, 0, 0, 0, 0, 0, 0, 0, 1, 2, 2, 1, 1, 0, 0~
## $ DyadResponse   <chr> "Tolerance", "Tolerance", "Tolerance", "Tolerance", "To~
## $ OtherResponse  <chr> "No Response", "No Response", "No Response", "No Respon~
## $ Audience       <chr> "Obse; Oup; Sirk", "Obse; Oup; Sirk", "Oup; Sirk", "Sir~
## $ IDIndividual1  <chr> "No individual", "No individual", "No individual", "No ~
## $ IntruderID     <chr> "No Intrusion", "No Intrusion", "No Intrusion", "No Int~
## $ Remarks        <chr> "No Remarks", "No Remarks", "Nge box did not open becau~
## $ MaleCorn       <dbl> 3, 3, 3, 3, 3, 3, 3, 3, 3, 3, 3, 3, 3, 3, 3, 3, 3, 3, 3~
## $ Intrusion      <dbl> 0, 0, 0, 0, 0, 0, 0, 0, 0, 0, 0, 1, 0, 0, 0, 0, 0, 0, 0~
## $ AmountAudience <dbl> 3, 3, 2, 1, 2, 2, 2, 1, 1, 2, 6, 6, 3, 2, 2, 2, 2, 2, 2~
\end{verbatim}

\begin{verbatim}
## [1] "/Users/maki/Desktop/Master Thesis/BEX 2223 Master Thesis Maung Kyaw/IVPToleranceBex2223"
\end{verbatim}

\hypertarget{journal-of-manual-changes-in-bex-excel-spreadsheet}{%
\paragraph{3.2.4 Journal of manual changes in Bex excel
spreadsheet}\label{journal-of-manual-changes-in-bex-excel-spreadsheet}}

\begin{itemize}
\item
  Before treating all the data in the Remarks I will create a few
  columns to redistribute information

  \begin{enumerate}
  \def\labelenumi{\arabic{enumi}.}
  \tightlist
  \item
    \textbf{Context}: To add contextual information
  \item
    \textbf{SpecialBehaviour} : To report any particular behaviour an
    individual could have done during a trial
  \item
    \textbf{Got corn}, to see if the individual got the corn or not
  \end{enumerate}
\item
  Also whenever i will have treated a remark, i will replace it with
  ``Treated''. And if I have to delete the row I'll write ``Delete''.
  After re importing the data I will make a count of these changes to
  see if I still have the correct amount of cells and changes that have
  been done
\item
  \begin{enumerate}
  \def\labelenumi{\arabic{enumi}.}
  \tightlist
  \item
    Creation of the columns \textbf{Context}, \textbf{SpecialBehaviour}
    and \textbf{GotCorn}
  \end{enumerate}
\item
  \begin{enumerate}
  \def\labelenumi{\arabic{enumi}.}
  \setcounter{enumi}{1}
  \tightlist
  \item
    Default values for the new columns are \textbf{NoContext},
    \textbf{NoSpecialBehaviour} \& \textbf{Yes}
  \end{enumerate}

  a.Context: \textbf{BoxMalfunction}, \textbf{BoxOpenedBefore},
  \textbf{NoExperiment}, \textbf{Agonistic}, \textbf{Guat;Ap;Xian},
  \textbf{CornLeak}, \textbf{BetweenGroupEncounter},
  \textbf{ContactCalling},

  b.SpecialBehaviour \textbf{Oerw;Vo;Exp}, \textbf{Sey;Ap;AfterOpen},
  \textbf{Oerw;Vo;Exp,Nge;Vo;Exp}, \textbf{Sirk;ApAfter30},
  \textbf{Sirk;Av;Oerw},
  \textbf{Oerw;Lo},\textbf{Sey;Sf;Oort,Oort;At;Kom},
  \textbf{Kom;Ap;AfterOpen}, \textbf{Sey;Ch,Sirk},
  \textbf{Xin;Hesitation}. \textbf{Xia;Sf;Piep},
  \textbf{Pom;Sf;Xian},\textbf{Kom;Sf;Oort}, \textbf{Sey;Sf;Sirk},
  \textbf{Xia;Sf;Piep,Piep;Sf,XIa}, \textbf{Oort;At;Kom},
  \textbf{Sey;Rt;Sho;Ap}, \textbf{Sho;Rt;Ginq;Ap}, \textbf{Buk;Sf;Ndaw},
  \textbf{Sho;Rt;Ndaw;Ap}, \textbf{Oort;Sf;Kom},
  \textbf{Ginq;Sho;Ap;After30}, \textbf{Ndaw;Sc,Buk;Sf},
  \textbf{Ndaw;Ap;After30}, \textbf{Kom;Ap;After30},
  \textbf{Xia;Piep;Ap;After30}, \textbf{Pom;Bi;Xian},
  \textbf{Sho;Ndaw;Av;Buk}, \textbf{Kom;Sf;Oort},
  \textbf{Kom;St;Oort,Oort;St;Kom}, \textbf{Sey;Hi;Sirk},
  \textbf{Obse;Ap;Piep;Av},\textbf{Piep;Sf;Xia},
  \textbf{Sirk;ApWhenPartnerLeft}, \textbf{Sey;Hh;Sirk},
  \textbf{Xia;Sf;Piep;Sc}, \textbf{Xia;Piep;ShareFood},
  \textbf{Piep;Ap;After30,Xia;Mu;Piep}, \textbf{Oort;St;Sirk;Ja,Sey;Sf},
  \textbf{Pom;Sf;Xian}, \textbf{Ndaw;ApWhenPartnerLeft},
  \textbf{Xian;At;Pom,Gaya;Su}, \textbf{Xian;Sf;Pom},
  \textbf{Xian;Hesitation},
  \textbf{Xia;ApWhenPartnerLeft},\textbf{Sirk;Hesitation},
  \textbf{Ginq;Hesitation}, \textbf{Sey;Ap;Kom;Av},
  \textbf{Oort;Sc;Kom}, \textbf{Xian; Pom}, \textbf{Pom;Ap;Xian},
  \textbf{Pom;Ap;Xian,Xian;Rt}, \textbf{Sey;Ap;Sirk;Rt},
  \textbf{Sey;St;Sirk;Ig}, \textbf{Xia;Asf;Piep},
  \textbf{Piep;ApWhenPartnerLeft}, \textbf{Sho;Ap;After30},
  \textbf{Ginq;ApWhenPartnerLeft}, \textbf{Pom;Sf;Xian;Sf;Pom},
  \textbf{Xian;ApWhenPartnerLeft}, \textbf{Piep;Ch;Sirk},
  \textbf{Sey;St;Sirk}, \textbf{Ndaw;Ap;After30},
  \textbf{Xian;Ap;After30}, \textbf{Xian;St;Prai},
  \textbf{Pom;Sf;Xian;Vc}, \textbf{Kom;Ap;After30},
  \textbf{Kom;ApproachWithPartner}, \textbf{Oort;ApWhenPartnerLeft},
  \textbf{Sho;Ap;After30},\textbf{Ginq;Ap;After30},
  \textbf{Ginq;ApproachWithPartner}, \textbf{Ndaw;Hesitation},
  \textbf{Oerw;Hesitation}, \textbf{Oerw;ApWhenPartnerLeft}.
  \textbf{Piep;Ap;After30}, \textbf{Sirk;Ap;After30},
  \textbf{Xia;Ap;After30}, \textbf{Ouli;Gr;BBOuli},
  \textbf{Oerw;Ap;After30}, \textbf{Sirk;Hesitation},
  \textbf{Sey;Ap;Sirk;Av}, \textbf{Ouli;Ap;Xia;Av},
  \textbf{Xin;Ap;After30}, \textbf{Sho;Sf;Ginq;Sc},
  \textbf{Xia;ApWhenPartnerLeft}, \textbf{Sey;Ap;Sirk;Ja},
  \textbf{Nge;Oerw;ShareFood}, \textbf{Nge;Ap;Oerw;Oerw;At,Obse;At;Nge},

  c.GotCorn: \textbf{No;Nge}, \textbf{No;Piep}, \textbf{No;Xian},
  \textbf{No;Oort}, \textbf{No;Sirk}, \textbf{No;Kom}, \textbf{No;Ndaw},
  \textbf{No;Kom}, \textbf{No;Oort}, \textbf{No;Xia}, \textbf{No;Buk},
  \textbf{No;Sho}, \textbf{No;Sey,No;Piep}, \textbf{No;Ginq}

  \begin{enumerate}
  \def\labelenumi{\alph{enumi}.}
  \setcounter{enumi}{3}
  \tightlist
  \item
    Remarks: \textbf{Treated}, \textbf{TODelete}
  \end{enumerate}
\item
  \begin{enumerate}
  \def\labelenumi{\arabic{enumi}.}
  \setcounter{enumi}{3}
  \tightlist
  \item
    Values set for exsting columns
  \end{enumerate}

  \begin{enumerate}
  \def\labelenumi{\alph{enumi}.}
  \item
    IntruderID: \textbf{Sey}, \textbf{Oerw}, \textbf{Guat},
    \textbf{Kom}, \textbf{Gris}, \textbf{Sho}, \textbf{Oerw; Ouli},
    \textbf{Guat; Gri}, \textbf{Xop}, \textbf{Obse}, \textbf{Oort},
    \textbf{Obse; Sey}, \textbf{Ginq; Ghid}, \textbf{Xia},
    \textbf{Grif}, \textbf{Sey}, \textbf{Gree; Gran}, \textbf{Godu;
    Gub}, \textbf{Gran}, \textbf{Oerw; Nak}, \textbf{Ghid},
    \textbf{Buk}, \textbf{Oup}
  \item
    DyadDistance: \textbf{6}, \textbf{7}, \textbf{8}, \textbf{9} ,
    \textbf{1}
  \item
    Audience: \textbf{UnidentifiedAudience}, \textbf{Ouli; Riss},
    \textbf{Gris}, \textbf{Sey}, \textbf{Sey; Piep; Sirk, Oup Ome}
  \item
    IDIndividual1: \textbf{Piep}, \textbf{Oort; Kom}, \textbf{Ndaw;
    Buk}, \textbf{Sho; Ginq}, \textbf{Ndaw, Buk}, \textbf{Xian, Pom},
    \textbf{Oort; Kom}, \textbf{Buk; Ndaw}, \textbf{Sirk; Sey},
    \textbf{Xin; Ouli}, \textbf{Oerw; Nge}
  \item
    DyadResponse: \textbf{Tolerance}, \textbf{Not approaching; Losing
    interest}, \textbf{Losing interest; Intrusion}
  \end{enumerate}
\end{itemize}

\hypertarget{re-uploading-the-dataset-after-the-treatment-of-the-remarks}{%
\subsubsection{3.3 Re Uploading the dataset after the treatment of the
Remarks}\label{re-uploading-the-dataset-after-the-treatment-of-the-remarks}}

\begin{itemize}
\tightlist
\item
  Now that I have manually treated the remarks in a spreadsheet I will
  re import the dataset
\end{itemize}

\begin{verbatim}
## Rows: 2,742
## Columns: 19
## $ Time             <chr> "09:47:50", "09:50:07", "09:53:11", "09:54:28", "09:5~
## $ Date             <dttm> 2022-09-27, 2022-09-27, 2022-09-27, 2022-09-27, 2022~
## $ Group            <chr> "Baie Dankie", "Baie Dankie", "Baie Dankie", "Baie Da~
## $ MaleID           <chr> "Nge", "Nge", "Nge", "Nge", "Nge", "Nge", "Nge", "Nge~
## $ FemaleID         <chr> "Oerw", "Oerw", "Oerw", "Oerw", "Oerw", "Oerw", "Oerw~
## $ FemaleCorn       <dbl> 0, 0, 0, 0, 0, 0, 0, 0, 0, 0, 0, 0, 0, 0, 0, 7, 7, 7,~
## $ DyadDistance     <dbl> 2, 2, 1, 1, 0, 0, 0, 0, 0, 0, 0, 0, 1, 2, 2, 1, 1, 0,~
## $ DyadResponse     <chr> "Tolerance", "Tolerance", "Tolerance", "Tolerance", "~
## $ OtherResponse    <chr> "No Response", "No Response", "No Response", "No Resp~
## $ Audience         <chr> "Obse; Oup; Sirk", "Obse; Oup; Sirk", "Oup; Sirk", "S~
## $ IDIndividual1    <chr> "No individual", "No individual", "No individual", "N~
## $ IntruderID       <chr> "No Intrusion", "No Intrusion", "No Intrusion", "No I~
## $ Remarks          <chr> "No Remarks", "No Remarks", "Treated", "Treated", "No~
## $ MaleCorn         <dbl> 3, 3, 3, 3, 3, 3, 3, 3, 3, 3, 3, 3, 3, 3, 3, 3, 3, 3,~
## $ Intrusion        <dbl> 0, 0, 0, 0, 0, 0, 0, 0, 0, 0, 0, 1, 0, 0, 0, 0, 0, 0,~
## $ AmountAudience   <dbl> 3, 3, 2, 1, 2, 2, 2, 1, 1, 2, 6, 6, 3, 2, 2, 2, 2, 2,~
## $ Context          <chr> "NoContext", "NoContext", "BoxMalfunction", "BoxOpene~
## $ SpecialBehaviour <chr> "NoSpecialBehaviour", "NoSpecialBehaviour", "Oerw;Vo;~
## $ GotCorn          <chr> "Yes", "Yes", "No;Nge", "Yes", "Yes", "Yes", "Yes", "~
\end{verbatim}

\begin{verbatim}
## # A tibble: 6 x 19
##   Time     Date                Group     MaleID FemaleID FemaleCorn DyadDistance
##   <chr>    <dttm>              <chr>     <chr>  <chr>         <dbl>        <dbl>
## 1 09:47:50 2022-09-27 00:00:00 Baie Dan~ Nge    Oerw              0            2
## 2 09:50:07 2022-09-27 00:00:00 Baie Dan~ Nge    Oerw              0            2
## 3 09:53:11 2022-09-27 00:00:00 Baie Dan~ Nge    Oerw              0            1
## 4 09:54:28 2022-09-27 00:00:00 Baie Dan~ Nge    Oerw              0            1
## 5 09:55:19 2022-09-27 00:00:00 Baie Dan~ Nge    Oerw              0            0
## 6 09:56:56 2022-09-27 00:00:00 Baie Dan~ Nge    Oerw              0            0
## # i 12 more variables: DyadResponse <chr>, OtherResponse <chr>, Audience <chr>,
## #   IDIndividual1 <chr>, IntruderID <chr>, Remarks <chr>, MaleCorn <dbl>,
## #   Intrusion <dbl>, AmountAudience <dbl>, Context <chr>,
## #   SpecialBehaviour <chr>, GotCorn <chr>
\end{verbatim}

\begin{itemize}
\tightlist
\item
  And Check for any remaining NA's in BexClean
\end{itemize}

\begin{verbatim}
## Number of NA entries in Context:  0
\end{verbatim}

\begin{verbatim}
## Number of NA entries in SpecialBehaviour:  0
\end{verbatim}

\begin{verbatim}
## Number of NA entries in GotCorn:  0
\end{verbatim}

\begin{verbatim}
## Number of NA entries in BexClean:  0
\end{verbatim}

\hypertarget{time---creation-of-period-and-hour}{%
\subsubsection{3.4 Time - Creation of Period and
Hour}\label{time---creation-of-period-and-hour}}

\begin{itemize}
\item
  Time : I considered looking at the time sections in which we did the
  expermiment. I will thus look at the time ranges (max and min in the
  day / latest and earliest time) before separating the day in different
  sections to have an idea in which part of the day most of the
  experiments occured. This will not be used in my analysis, but if I
  wanted to, I could interesting to compare the amount of
  experimentations made per day and have a line indicating the time of
  sunrise.
\item
  The \textbf{Minimum Time} in the dataset is \textbf{06:03:26}* while
  the \textbf{Maximum Time} is at \textbf{16:36:59}
\item
  In my box experiment I have this variable called time that tells me
  when the experiment was done. I don't think I need this information
  per se. I was wondering if it could be easy and interesting to see
  from when to when the time occurs and then separate this time in a few
  sections like early, monring, morning, miday, afternoon, end of the
  day
\item
  a.6 to 8 : Early morning b.8 to 10: Morning c.10 to 12: Noon d.12 to
  14: Afternoon e.14 to 17: End of the day
\item
  Last, I want to create a variable called Hour that will take the value
  in Time and round it to the hour in which it is ex: from 06:00 to
  06:59 -\textgreater{} 6, from 07:00 to 07:59 -\textgreater{} 7
  etc\ldots{}
\item
  This will allow me to see when most of the trials occured with more
  detail and I will be to see in which hour most of the trial happened.
  Nevertheless Period will be better for an improved readability
\end{itemize}

\hypertarget{date---creation-of-month-and-day}{%
\subsubsection{3.5 Date - Creation of Month and
Day}\label{date---creation-of-month-and-day}}

\begin{itemize}
\tightlist
\item
  i want to Create a variable called month to see the month of the
  experiment and day so I know which day of the experiment it was (1st,
  10th, 1000th..)
\end{itemize}

\hypertarget{male-and-female-id---creation-of-dyad-trial-and-session}{%
\subsubsection{3.6 Male and Female ID - Creation of Dyad, Trial and
Session}\label{male-and-female-id---creation-of-dyad-trial-and-session}}

\begin{itemize}
\tightlist
\item
  I will use Female and Male ID to create different variables

  \begin{enumerate}
  \def\labelenumi{\arabic{enumi}.}
  \tightlist
  \item
    While checking if there are still any mistakes in \textbf{FemaleID
    and MaleID} using \textbf{unique}, I saw that some of the names are
    in the wrong rows. I want the \textbf{3 letter male codes} whether
    they are in the column FemaleID or MaleID to be in the \textbf{new
    column Male} while I want the \textbf{4 letter female codes} whether
    they are in FemaleID or MaleID to be in the \textbf{new column
    Female} before checking again with unique that the transformation
    worked. I will use mutate
  \end{enumerate}
\end{itemize}

\begin{verbatim}
## Unique Female IDs: Sirk Ginq Piep Oerw Xin Ndaw Xia Sey Ouli Nge Oort Xian Guat Kom
\end{verbatim}

\begin{verbatim}
## Unique Male IDs: Sey Sho Xia Nge Ouli Buk Piep Sirk Xin Oerw Kom Pom Oort
\end{verbatim}

\begin{longtable}[]{@{}lrrrrrrrrrrrrr@{}}
\toprule
& Buk & Kom & Nge & Oerw & Oort & Ouli & Piep & Pom & Sey & Sho & Sirk &
Xia & Xin \\
\midrule
\endhead
Ginq & 6 & 0 & 0 & 0 & 0 & 0 & 0 & 0 & 0 & 277 & 0 & 0 & 0 \\
Guat & 0 & 0 & 0 & 0 & 0 & 0 & 0 & 5 & 0 & 0 & 0 & 0 & 0 \\
Kom & 0 & 0 & 0 & 0 & 29 & 0 & 0 & 0 & 0 & 0 & 0 & 0 & 0 \\
Ndaw & 259 & 0 & 0 & 0 & 0 & 0 & 0 & 0 & 0 & 0 & 0 & 0 & 0 \\
Nge & 0 & 0 & 0 & 19 & 0 & 0 & 0 & 0 & 0 & 0 & 0 & 0 & 0 \\
Oerw & 0 & 0 & 164 & 0 & 0 & 0 & 0 & 0 & 0 & 0 & 0 & 0 & 0 \\
Oort & 0 & 337 & 0 & 0 & 0 & 0 & 0 & 0 & 0 & 0 & 0 & 0 & 4 \\
Ouli & 0 & 0 & 0 & 0 & 0 & 0 & 0 & 0 & 0 & 0 & 0 & 0 & 159 \\
Piep & 0 & 0 & 0 & 0 & 0 & 0 & 0 & 0 & 0 & 0 & 0 & 575 & 0 \\
Sey & 0 & 0 & 0 & 0 & 0 & 0 & 0 & 0 & 0 & 0 & 17 & 0 & 0 \\
Sirk & 0 & 0 & 0 & 0 & 0 & 0 & 0 & 0 & 570 & 0 & 0 & 0 & 0 \\
Xia & 0 & 0 & 0 & 0 & 0 & 0 & 35 & 0 & 0 & 0 & 0 & 0 & 0 \\
Xian & 0 & 0 & 0 & 0 & 0 & 0 & 0 & 259 & 0 & 0 & 0 & 0 & 0 \\
Xin & 0 & 0 & 0 & 0 & 0 & 27 & 0 & 0 & 0 & 0 & 0 & 0 & 0 \\
\bottomrule
\end{longtable}

\begin{enumerate}
\def\labelenumi{\arabic{enumi}.}
\tightlist
\item
  Create a variable called Male that in each row will take the name of
  the \textbf{3 letter code} that is either in MaleID or Female ID and a
  variable called Female that in each row will take the name of the
  \textbf{4 letter code} that is either in MaleID or FemaleID
\end{enumerate}

\begin{verbatim}
## Unique Dyads: Sey Sirk Sho Ginq Xia Piep Nge Oerw Xin Ouli Buk Ndaw Buk Ginq Kom Oort Pom Xian Pom Guat Xin Oort
\end{verbatim}

\begin{longtable}[]{@{}lr@{}}
\toprule
Var1 & Freq \\
\midrule
\endhead
Buk Ginq & 6 \\
Buk Ndaw & 259 \\
Kom Oort & 366 \\
Nge Oerw & 183 \\
Pom Guat & 5 \\
Pom Xian & 259 \\
Sey Sirk & 587 \\
Sho Ginq & 277 \\
Xia Piep & 610 \\
Xin Oort & 4 \\
Xin Ouli & 186 \\
\bottomrule
\end{longtable}

\begin{verbatim}
## Unique Male-Female Combinations:
\end{verbatim}

\begin{verbatim}
## # A tibble: 11 x 2
##    Male  Female
##    <chr> <chr> 
##  1 Sey   Sirk  
##  2 Sho   Ginq  
##  3 Xia   Piep  
##  4 Nge   Oerw  
##  5 Xin   Ouli  
##  6 Buk   Ndaw  
##  7 Buk   Ginq  
##  8 Kom   Oort  
##  9 Pom   Xian  
## 10 Pom   Guat  
## 11 Xin   Oort
\end{verbatim}

\begin{enumerate}
\def\labelenumi{\arabic{enumi}.}
\setcounter{enumi}{1}
\tightlist
\item
  Create the variable called \textbf{Dyad} created by combining the name
  of FemaleID and MaleID into one name with a space between the two
  codes. For information the 3 letter code is the name of the female
  while the 4 letter code is the name of the male like displayed here
  with the correct dyads;
\end{enumerate}

\begin{itemize}
\tightlist
\item
  The correct dyads are:

  \begin{itemize}
  \tightlist
  \item
    Buk Ndaw\\
  \item
    Kom Oort\\
  \item
    Nge Oerw\\
  \item
    Pom Guat\\
  \item
    Pom Xian\\
  \item
    Sey Sirk\\
  \item
    Sho Ginq\\
  \item
    Xia Piep\\
  \item
    Xin Ouli
  \end{itemize}
\item
  While the dyads we have now in the datset are;

  \begin{itemize}
  \tightlist
  \item
    Buk Ginq 6
  \item
    Buk Ndaw 259
  \item
    Kom Oort 366
  \item
    Nge Oerw 183
  \item
    Pom Guat 5
  \item
    Pom Xian 259
  \item
    Sey Sirk 587
  \item
    Sho Ginq 277
  \item
    Xia Piep 610
  \item
    Xin Oort 4
  \item
    Xin Ouli 186
  \end{itemize}
\end{itemize}

\begin{verbatim}
## [1] "Wrong Rows:"
\end{verbatim}

\begin{verbatim}
##  [1]  613  614  615  616  617  931 2710 2711 2712 2713
\end{verbatim}

\begin{verbatim}
## [1] "Wrong Dyads:"
\end{verbatim}

\begin{verbatim}
##  [1] "Buk Ginq" "Buk Ginq" "Buk Ginq" "Buk Ginq" "Buk Ginq" "Buk Ginq"
##  [7] "Xin Oort" "Xin Oort" "Xin Oort" "Xin Oort"
\end{verbatim}

\begin{itemize}
\item
  They are a 10 \textbf{wrong dyads} that I will have to identify in the
  dataset and manually correct, those wrong dyads to change and identify
  are: -Buk Ginq - 6 occurences -Xin Oort - 4 occurences
\item
  I will change the occurences of \textbf{Buk Ginq} to \textbf{Sho Ginq}
  for \textbf{row 613 to 617} and \textbf{row 931}. I know these trials
  are with Sho Ginq because the comments mentioned Sho in them while
  Male(ID) gave Buk which was a mistake
\item
  For the \textbf{rows from 2710 to 2713} since, Ouli is in the audience
  it is unlikely that we had trials with the dyad \textbf{Xin Ouli}.
  Also I think they are little chances that the names of both
  individuals were entered wrong. I will replace these occurences where
  we had \textbf{Xin Oort} by \textbf{Kom Oort}
\item
  I thus want \textbf{Buk} to be replaced in male ID in rows 613 to 617
  and row 913 with \textbf{Sho} and, \textbf{Xin} to be replaced by
  \textbf{Kom} in rows 2710 to 2713 in Male ID before updating Dyad
\item
  If Rows 613 to 617 and 931 are coded with Buk for MaleId and Ginq for
  FemaleId replace Male by Sho
\end{itemize}

\begin{verbatim}
## Rows to correct Sho Ginq:
\end{verbatim}

\begin{verbatim}
## [1] 613 614 615 616 617 931
\end{verbatim}

\begin{verbatim}
## Rows to correct Kom Oort:
\end{verbatim}

\begin{verbatim}
## [1] 2710 2711 2712 2713
\end{verbatim}

\begin{verbatim}
## Wrong Rows After Correction:
\end{verbatim}

\begin{verbatim}
## integer(0)
\end{verbatim}

\begin{verbatim}
## Wrong Dyads After Correction:
\end{verbatim}

\begin{verbatim}
## character(0)
\end{verbatim}

\begin{verbatim}
## All dyads are now correct.
\end{verbatim}

\begin{verbatim}
## Unique Dyads after correction: Sey Sirk Sho Ginq Xia Piep Nge Oerw Xin Ouli Buk Ndaw Kom Oort Pom Xian Pom Guat
\end{verbatim}

\begin{longtable}[]{@{}lr@{}}
\toprule
Var1 & Freq \\
\midrule
\endhead
Buk Ndaw & 259 \\
Kom Oort & 370 \\
Nge Oerw & 183 \\
Pom Guat & 5 \\
Pom Xian & 259 \\
Sey Sirk & 587 \\
Sho Ginq & 283 \\
Xia Piep & 610 \\
Xin Ouli & 186 \\
\bottomrule
\end{longtable}

\begin{verbatim}
## Number of rows changed to Sho Ginq: 6
\end{verbatim}

\begin{verbatim}
## Number of rows changed to Kom Oort: 4
\end{verbatim}

\begin{enumerate}
\def\labelenumi{\arabic{enumi}.}
\setcounter{enumi}{2}
\tightlist
\item
  Create the variable called \textbf{Trial} where the data will be
  \textbf{sorted by date and dyad} in order to see how many trials have
  been done with each individual: \textbf{One row (per dyad) = one
  trial} and the variable called \textbf{Day} where the data will be
  \textbf{sorted by date and dyad and day} in order to see how many
  sessions have been done with each individual: \textbf{One day (per
  dyad) = one session} Now, let's proceed with creating the Dyad
  variable, Trial, and Day:
\end{enumerate}

\begin{itemize}
\item
  **Experiment Day (across all dyads): This variable should count the
  number of unique experiment days across all dyads. For instance, if
  multiple dyads have trials on the same date, that date should be
  considered a single experiment day.
\item
  \textbf{Dyad Day} (within each dyad): This variable should count the
  number of unique experiment days for each dyad separately. The first
  day of trials for a dyad should be Day 1, the second distinct date of
  trials should be Day 2, and so on.
\item
  DaysSinceStart: Tracks the total number of days since the first
  experiment, counting every calendar day, including gaps between
  experiments.
\item
  ExperimentDay: Counts unique experiment dates, with each distinct date
  assigned a consecutive day number.
\item
  Trial: Numbers the trials sequentially within each dyad, starting from
  1 for each dyad.
\item
  \begin{itemize}
  \tightlist
  \item
    DyadDay: Counts unique experiment days for each dyad separately,
    assigning Day 1 to the first distinct date, Day 2 to the next, etc.
  \end{itemize}
\end{itemize}

\begin{verbatim}
## # A tibble: 2,742 x 5
##    Date       Dyad     Trial DyadDay TrialDay
##    <date>     <chr>    <int>   <int>    <int>
##  1 2022-09-29 Buk Ndaw     1       1        1
##  2 2022-09-29 Buk Ndaw     2       1        2
##  3 2022-09-29 Buk Ndaw     3       1        3
##  4 2022-10-04 Buk Ndaw     4       2        1
##  5 2022-10-13 Buk Ndaw     5       3        1
##  6 2022-10-13 Buk Ndaw     6       3        2
##  7 2022-10-13 Buk Ndaw     7       3        3
##  8 2022-10-13 Buk Ndaw     8       3        4
##  9 2022-10-13 Buk Ndaw     9       3        5
## 10 2022-10-13 Buk Ndaw    10       3        6
## # i 2,732 more rows
\end{verbatim}

\begin{enumerate}
\def\labelenumi{\arabic{enumi}.}
\setcounter{enumi}{3}
\tightlist
\item
  Make a summary of trial and session so I can see see how many trials
  and sessions have been done with the individuals
\end{enumerate}

\begin{verbatim}
## Total Number of Unique Experiment Days: 92
\end{verbatim}

\begin{verbatim}
## 
## Combined Summary:
\end{verbatim}

\begin{longtable}[]{@{}lrr@{}}
\caption{Summary of Trials and Days}\tabularnewline
\toprule
Dyad & Amount of Trials & Number of Days \\
\midrule
\endfirsthead
\toprule
Dyad & Amount of Trials & Number of Days \\
\midrule
\endhead
Xia Piep & 610 & 49 \\
Sey Sirk & 587 & 53 \\
Kom Oort & 370 & 31 \\
Sho Ginq & 283 & 35 \\
Buk Ndaw & 259 & 39 \\
Pom Xian & 259 & 19 \\
Xin Ouli & 186 & 27 \\
Nge Oerw & 183 & 22 \\
Pom Guat & 5 & 1 \\
\bottomrule
\end{longtable}

\includegraphics{BEX2223_files/figure-latex/Dyad and Day Summary-1.pdf}
\includegraphics{BEX2223_files/figure-latex/Dyad and Day Summary-2.pdf}

\begin{enumerate}
\def\labelenumi{\arabic{enumi}.}
\setcounter{enumi}{4}
\tightlist
\item
  After relfection I decided to \textbf{remove every column} that is
  with PomGuat since they are not enough trials for this Dyad and since
  we then changed PomGuat for PomXian. I have 5 occurnces to change. For
  easier manipulation I will remove every row where there is
  \textbf{Guat}
\end{enumerate}

\begin{verbatim}
## Change in Rows:  -5
\end{verbatim}

\hypertarget{female-and-male-placement-corn}{%
\subsubsection{3.7 Female and Male Placement
Corn}\label{female-and-male-placement-corn}}

\begin{itemize}
\tightlist
\item
  The idea is that for each Dyad, we gave an amount of corn to attract
  the monkey of a dyad to the right distance of his partner for a trial
  by putting corn in experiment box that he could get by approaching. We
  repeated this step as much as needed to have our Dyad at the desired
  distance to continue the trials from the previous day of
  experimentation. This means I will only Keep the last number per dyad
  and day for each day.
\item
  I decided to create \textbf{two variables}, that are called
  \textbf{PlacementMale} and \textbf{PlacementFemale} that will only
  keep the final amount of corn given to each individual within a day of
  experiment
\end{itemize}

\begin{verbatim}
## Placement columns were created successfully.
\end{verbatim}

\includegraphics{BEX2223_files/figure-latex/PlacementMaleFemale Plot-1.pdf}
\includegraphics{BEX2223_files/figure-latex/PlacementMaleFemale Plot-2.pdf}

\hypertarget{dyaddistance---creation-of-proximity}{%
\subsubsection{3.8 DyadDistance - Creation of
proximity}\label{dyaddistance---creation-of-proximity}}

\begin{itemize}
\tightlist
\item
  I would like to create a variable called proximity to have another
  measure of the proximity of the individuals.
\item
  First lets look at the maximum and minimum distance found in Dyad
  Distance
\end{itemize}

\begin{verbatim}
## Maximum Distance: 10
\end{verbatim}

\begin{verbatim}
## Minimum Distance: 0
\end{verbatim}

\begin{itemize}
\tightlist
\item
  The \textbf{minumum Distance} is \textbf{0m} while the \textbf{maximum
  Distance} is \textbf{10m}
\item
  Then lets make a graph of the frequency for each distances with a line
  for the median to show the most frquent distance
\end{itemize}

\begin{verbatim}
## # A tibble: 11 x 2
##    DyadDistance Frequency
##           <dbl>     <int>
##  1            0       904
##  2            1       644
##  3            2       471
##  4            3       303
##  5            4       217
##  6            5       133
##  7            6        41
##  8            7        13
##  9            8         9
## 10            9         1
## 11           10         1
\end{verbatim}

\includegraphics{BEX2223_files/figure-latex/DyadDistance Frequency Plot-1.pdf}

\begin{itemize}
\tightlist
\item
  Now lets create a new variable called \textbf{Proximity} using the
  distances found in \textbf{DyadDistance} on the following model:

  \begin{enumerate}
  \def\labelenumi{\alph{enumi}.}
  \tightlist
  \item
    0 = Contact
  \item
    1 - 2 = 1-2
  \item
    2 - 3 = 2-3
  \item
    4 - 5 = 4-5
  \item
    5 - 10 = +5
  \end{enumerate}
\end{itemize}

\includegraphics{BEX2223_files/figure-latex/Proximity plot-1.pdf}

\hypertarget{dyad-response---corrections-and-detailed-cleaning}{%
\subsubsection{3.9 Dyad Response - Corrections and detailed
cleaning}\label{dyad-response---corrections-and-detailed-cleaning}}

\hypertarget{section}{%
\paragraph{3.9.1}\label{section}}

\begin{itemize}
\item
  Reminder: The different behaviors that are coded in
  \textbf{DyadResponse} are: \textbf{Distracted}, \textbf{Female aggress
  male}, \textbf{Male aggress female}, \textbf{Intrusion},
  \textbf{Loosing interest}, \textbf{Not approaching},
  \textbf{Tolerance} and \textbf{Other}

  \begin{itemize}
  \tightlist
  \item
    I will change the columns associated to each behavior
    (i.e.~Response) of \textbf{DyadResponse} into dichotomic variables
    in order to see the frequency of each behaviour
  \item
    This will allow me to see which behavior occurred more than others,
    and what differences are between dyads
  \item
    As multiple behaviours could occur within the same trial, multiple
    responses (data entries) can be found in a single cell. I will
    create a hierarchy to reduce the amount of behaviors assigned to
    each trial (if there is more than one).

    \begin{enumerate}
    \def\labelenumi{\arabic{enumi}.}
    \tightlist
    \item
      correct any potential discrepancies (ex. if tolerance and
      aggression occured within the same trial
      aggression\textgreater tolerance)
    \item
      assign as few labels per trial, ideally one using a hierarchy
      among the occuring behaviours to choose which response to keep
    \item
      get a better view and understanding of the data and the most
      common behaviours produced by each dyad producing plots and tables
    \item
      if necessary create new variables that can complement redistribute
      the information initally found in the column
    \end{enumerate}
  \end{itemize}
\item
  I will create some tables to have a better understanding of the state
  of the column dyadresponse and the different existing combinations at
  this point
\item
  Also I will create the hierarchy before implementing in the dataset
\end{itemize}

\hypertarget{dyad-response-description}{%
\paragraph{3.9.2 Dyad Response
description}\label{dyad-response-description}}

\begin{itemize}
\tightlist
\item
  I need to know how many different combinations exist within
  DyadResponse and how many occurrences/cells have more than one
  response per cell. I need the code to not take in account the order
  but rather the responses in themselves
\item
  I want one table for the Different combinations and one showing the
  differernt combinations for only ONE response per cell, in a second
  table
\end{itemize}

\begin{verbatim}
## Total number of rows in DyadResponse:  2737
\end{verbatim}

\begin{verbatim}
## Number of rows with a single entry in DyadResponse:  2466
\end{verbatim}

\begin{verbatim}
## Number of rows with multiple entries in DyadResponse:  271
\end{verbatim}

\begin{verbatim}
## Number of rows with exactly 2 entries in DyadResponse:  258
\end{verbatim}

\begin{verbatim}
## Number of rows with more than 2 entries in DyadResponse:  13
\end{verbatim}

\begin{verbatim}
## 
## 
## Table: Rows with Single Responses in DyadResponse
## 
## |DyadResponse        | Frequency|
## |:-------------------|---------:|
## |Tolerance           |      1809|
## |Not approaching     |       465|
## |Male aggress female |        73|
## |Intrusion           |        51|
## |Losing interest     |        37|
## |Female aggress male |        18|
## |Other               |         9|
## |Distracted          |         4|
\end{verbatim}

\includegraphics{BEX2223_files/figure-latex/DyadResponse unique and multiple combinations and frequencies and plots-1.pdf}

\begin{verbatim}
## 
## 
## Table: Rows with Multiple Responses in DyadResponse
## 
## |Combination                                       | Frequency|
## |:-------------------------------------------------|---------:|
## |Intrusion;Tolerance                               |        49|
## |Losing interest;Not approaching                   |        47|
## |Intrusion;Not approaching                         |        30|
## |Male aggress female;Tolerance                     |        27|
## |Looks at partner;Tolerance                        |        23|
## |Losing interest;Tolerance                         |        22|
## |Female aggress male;Tolerance                     |        19|
## |Looks at partner;Not approaching                  |         9|
## |Other;Tolerance                                   |         7|
## |Distracted;Not approaching                        |         6|
## |Not approaching;Tolerance                         |         4|
## |Distracted;Losing interest                        |         3|
## |Female aggress male;Male aggress female;Tolerance |         3|
## |Female aggress male;Not approaching               |         3|
## |Distracted;Tolerance                              |         2|
## |Female aggress male;Intrusion;Tolerance           |         2|
## |Male aggress female;Not approaching               |         2|
## |Distracted;Intrusion;Not approaching              |         1|
## |Distracted;Intrusion;Tolerance                    |         1|
## |Female aggress male;Intrusion                     |         1|
## |Intrusion;Losing interest                         |         1|
## |Intrusion;Losing interest;Not approaching         |         1|
## |Intrusion;Male aggress female                     |         1|
## |Intrusion;Not approaching;Tolerance               |         1|
## |Looks at partner;Losing interest;Not approaching  |         1|
## |Looks at partner;Male aggress female              |         1|
## |Looks at partner;Male aggress female;Tolerance    |         1|
## |Looks at partner;Other;Tolerance                  |         1|
## |Losing interest;Not approaching;Tolerance         |         1|
## |Not approaching;Other                             |         1|
\end{verbatim}

\includegraphics{BEX2223_files/figure-latex/DyadResponse unique and multiple combinations and frequencies and plots-2.pdf}

\begin{verbatim}
## Unique Combinations and Counts for More than 2 Entries:
## Female aggress male & Male aggress female & Tolerance 3
## Female aggress male & Intrusion & Tolerance 2
## Distracted & Intrusion & Not approaching 1
## Distracted & Intrusion & Tolerance 1
## Intrusion & Losing interest & Not approaching 1
## Intrusion & Not approaching & Tolerance 1
## Looks at partner & Losing interest & Not approaching 1
## Looks at partner & Male aggress female & Tolerance 1
## Looks at partner & Other & Tolerance 1
## Losing interest & Not approaching & Tolerance 1
\end{verbatim}

\begin{verbatim}
## # A tibble: 271 x 36
##    Time    Date       Group MaleID FemaleID FemaleCorn DyadDistance DyadResponse
##    <chr>   <date>     <chr> <chr>  <chr>         <dbl>        <dbl> <chr>       
##  1 08:37:~ 2022-09-29 Ankh~ Buk    Ndaw              0            5 Not approac~
##  2 08:16:~ 2022-10-04 Ankh~ Buk    Ndaw              3            5 Losing inte~
##  3 08:12:~ 2022-10-13 Ankh~ Buk    Ndaw              3            1 Tolerance; ~
##  4 09:02:~ 2022-11-10 Ankh~ Buk    Ndaw              3            3 Not approac~
##  5 14:05:~ 2022-11-16 Ankh~ Buk    Ndaw              0            3 Not approac~
##  6 14:23:~ 2022-11-16 Ankh~ Buk    Ndaw              1            2 Not approac~
##  7 14:25:~ 2022-11-17 Ankh~ Buk    Ndaw              4            4 Not approac~
##  8 14:27:~ 2022-11-18 Ankh~ Buk    Ndaw              0            6 Not approac~
##  9 08:58:~ 2022-11-22 Ankh~ Buk    Ndaw              6            6 Not approac~
## 10 12:47:~ 2022-11-24 Ankh~ Buk    Ndaw              4            4 Not approac~
## # i 261 more rows
## # i 28 more variables: OtherResponse <chr>, Audience <chr>,
## #   IDIndividual1 <chr>, IntruderID <chr>, Remarks <chr>, MaleCorn <dbl>,
## #   Intrusion <dbl>, AmountAudience <dbl>, Context <chr>,
## #   SpecialBehaviour <chr>, GotCorn <chr>, Period <chr>, Hour <chr>, Day <dbl>,
## #   Month <chr>, Male <chr>, Female <chr>, Dyad <chr>, DaysSinceStart <dbl>,
## #   ExperimentDay <dbl>, ExperimentDay_Verified <dbl>, Trial <int>, ...
\end{verbatim}

\begin{verbatim}
## Rows with multiple entries in DyadResponse:  3 4 13 53 58 64 68 71 72 84 88 89 97 98 113 133 137 192 194 244 275 277 278 284 289 298 302 304 318 322 326 327 328 331 341 346 347 367 368 387 388 391 397 409 454 468 469 482 497 510 511 522 529 536 597 600 626 629 687 696 706 711 713 726 740 748 749 757 760 761 763 768 769 780 786 818 829 841 843 845 861 866 877 888 892 898 912 919 934 937 938 954 955 962 973 997 1004 1015 1027 1042 1048 1060 1104 1141 1143 1164 1165 1190 1209 1213 1217 1225 1229 1236 1240 1243 1245 1247 1254 1267 1271 1284 1288 1293 1308 1310 1311 1326 1327 1330 1336 1342 1344 1353 1378 1398 1399 1407 1416 1461 1464 1473 1478 1484 1488 1489 1495 1510 1520 1521 1528 1558 1595 1598 1602 1607 1616 1629 1630 1633 1636 1639 1642 1645 1654 1658 1679 1708 1714 1718 1724 1741 1750 1752 1784 1785 1788 1795 1800 1801 1802 1804 1805 1806 1807 1808 1815 1816 1819 1820 1821 1822 1824 1825 1828 1829 1830 1831 1832 1837 1843 1856 1857 1858 1889 1897 1968 1984 2023 2066 2088 2090 2091 2100 2101 2102 2103 2104 2105 2109 2112 2113 2147 2148 2149 2152 2153 2166 2175 2184 2185 2192 2193 2194 2199 2201 2202 2205 2218 2226 2237 2248 2257 2290 2294 2343 2353 2354 2359 2360 2362 2363 2370 2388 2395 2396 2400 2465 2466 2467 2493 2534 2601 2609 2626 2629 2630 2643 2651 2657 2730
\end{verbatim}

\begin{verbatim}
## Unique Responses and Counts (Sorted Alphabetically):
\end{verbatim}

\begin{verbatim}
## Distracted                4
## Distracted Losing interest 3
## Female aggress male       18
## Female aggress male Intrusion 1
## Female aggress male Not approaching 3
## Intrusion                 51
## Losing interest           37
## Losing interest Intrusion 1
## Male aggress female       73
## Male aggress female Intrusion 1
## Male aggress female Looks at partner 1
## Male aggress female Not approaching 2
## Not approaching           465
## Not approaching Distracted 6
## Not approaching Distracted Intrusion 1
## Not approaching Intrusion 30
## Not approaching Looks at partner 9
## Not approaching Losing interest 47
## Not approaching Losing interest Intrusion 1
## Not approaching Losing interest Looks at partner 1
## Not approaching Other     1
## Other                     9
## Tolerance                 1809
## Tolerance Distracted      2
## Tolerance Distracted Intrusion 1
## Tolerance Female aggress male 19
## Tolerance Female aggress male Intrusion 2
## Tolerance Intrusion       49
## Tolerance Looks at partner 23
## Tolerance Looks at partner Other 1
## Tolerance Losing interest 22
## Tolerance Male aggress female 27
## Tolerance Male aggress female Female aggress male 3
## Tolerance Male aggress female Looks at partner 1
## Tolerance Not approaching 4
## Tolerance Not approaching Intrusion 1
## Tolerance Not approaching Losing interest 1
## Tolerance Other           7
\end{verbatim}

\includegraphics{BEX2223_files/figure-latex/Dyad Response Unique-1.pdf}

\hypertarget{dyad-response-hierarchy}{%
\paragraph{3.9.3 Dyad Response
Hierarchy}\label{dyad-response-hierarchy}}

\hypertarget{a-dyadresponse-hierarchy---looks-at-partner}{%
\subparagraph{3.9.3.a DyadResponse Hierarchy - Looks at
partner}\label{a-dyadresponse-hierarchy---looks-at-partner}}

\begin{itemize}
\tightlist
\item
  I will delete all the occurences of \textbf{Looks at partner} in
  **DyadResponse as this data was not collected during the whole period
  of the experiment and will not be usable for my analysis
\end{itemize}

\begin{verbatim}
## Number of occurrences of 'Looks at partner' in DyadResponse:  36
\end{verbatim}

\begin{verbatim}
## Number of rows that have been changed:  36
\end{verbatim}

\hypertarget{b-dyadresponse-hierarchy---other}{%
\subparagraph{3.9.3.b DyadResponse Hierarchy -
Other}\label{b-dyadresponse-hierarchy---other}}

\begin{itemize}
\tightlist
\item
  I will print the amount of \textbf{Other} occurrencess in DyadResponse
  and treat them to remove some multiple combinations
\end{itemize}

\begin{verbatim}
## Number of 'Other' occurrences in DyadResponse:  18
\end{verbatim}

\begin{longtable}[]{@{}
  >{\raggedleft\arraybackslash}p{(\columnwidth - 6\tabcolsep) * \real{0.05}}
  >{\raggedright\arraybackslash}p{(\columnwidth - 6\tabcolsep) * \real{0.08}}
  >{\raggedright\arraybackslash}p{(\columnwidth - 6\tabcolsep) * \real{0.10}}
  >{\raggedright\arraybackslash}p{(\columnwidth - 6\tabcolsep) * \real{0.77}}@{}}
\caption{Rows with `Other' in DyadResponse}\tabularnewline
\toprule
\begin{minipage}[b]{\linewidth}\raggedleft
Line
\end{minipage} & \begin{minipage}[b]{\linewidth}\raggedright
MaleID
\end{minipage} & \begin{minipage}[b]{\linewidth}\raggedright
FemaleID
\end{minipage} & \begin{minipage}[b]{\linewidth}\raggedright
OtherResponse
\end{minipage} \\
\midrule
\endfirsthead
\toprule
\begin{minipage}[b]{\linewidth}\raggedleft
Line
\end{minipage} & \begin{minipage}[b]{\linewidth}\raggedright
MaleID
\end{minipage} & \begin{minipage}[b]{\linewidth}\raggedright
FemaleID
\end{minipage} & \begin{minipage}[b]{\linewidth}\raggedright
OtherResponse
\end{minipage} \\
\midrule
\endhead
39 & Buk & Ndaw & No Response \\
286 & Kom & Oort & Ooet scream while at the box and Kom get the corm of
both \\
300 & Kom & Oort & Both at boxes; Kom touching the box of Oort \\
304 & Kom & Oort & Kom touching Oort's box. Oort came after 30 sec to
her own box. \\
700 & Nge & Oerw & No Response \\
815 & Pom & Xian & No Response \\
826 & Pom & Xian & No Response \\
1484 & Sey & Sirk & wait for her. moved back and waited for sirk to
approach and came back \\
1659 & Sho & Ginq & No Response \\
1741 & Sho & Ginq & Sho took one of her corn \\
2101 & Xia & Piep & aggression \\
2102 & Xia & Piep & Xia ate corn from both boxes \\
2103 & Xia & Piep & Xia stolen some pieces from piep box \\
2104 & Xia & Piep & Xia immediately go for piep box and eat her corn
then eat his. \\
2105 & Xia & Piep & piep got 2 pieces out of 3. Xia took one corn \\
2170 & Piep & Xia & fem ag male \\
2184 & Xia & Piep & Neither ID approach \\
2729 & Xin & Ouli & opened for oerw \\
\bottomrule
\end{longtable}

\begin{itemize}
\item
  There are 18 cases where was \textbf{Other} as a Response in the
  experiment. Here are the modifications I will make for them:

  -First lets remember that all of these \textbf{DyadResponse} contain
  \textbf{Other} which will make their identification easier

  \begin{itemize}
  \tightlist
  \item
    Row 39: Since there were No Response as ``Other'' I will
    \textbf{Delete Row 39} (Buk Ndaw No Response)
  \item
    Row 286: In \textbf{DyadResponse} instead of \textbf{Other} I will
    put \textbf{Tolerance} and in \textbf{SpecialBehaviour} I will put
    \textbf{Kom;Ap;Sf;Oort,Oort;Rt;Sc} and in \textbf{GotCorn} I will
    put \textbf{No;Oort} (Kom Oort Ooet scream while at the box and Kom
    get the corm of both)
  \item
    Row 300: In \textbf{DyadResponse} I will put \textbf{Tolerance} as
    both individual were at their box at the same time (300 Kom Oort
    Both at boxes; Kom touching the box of Oort)
  \item
    Row 304: In \textbf{DyadResponse} I will put
    \textbf{Oort;Ap;After30} in \textbf{SpecialBehaviour} (Kom Oort Kom
    touching Oort's box. Oort came after 30 sec to her own box.)
  \item
    Row 700: Since ther were No Response as ``Other'' I will
    \textbf{Delete Row 700} ( Nge Oerw No Response)
  \item
    Row 815: Since ther were No Response as ``Other'' I will
    \textbf{Delete Row 815} (Pom Xian No Response)
  \item
    Row 826: Since ther were No Response as ``Other'' I will
    \textbf{Delete Row 826} (Pom Xian No Response)
  \item
    Row 1484: In \textbf{Row 1484} I will replace the
    \textbf{DyadResponse} occurence \textbf{Other by }Tolerance** and
    put \textbf{Sirk;ApproachWithPartner} (Sey Sirk wait for her. moved
    back and waited for sirk to approach and came back)
  \item
    Row 1659: Since there were No Response as ``Other'' I will
    \textbf{Delete Row 1659} (1659 Sho Ginq No Response)
  \item
    Row 1741: In \textbf{Row 1174} Sho stole corn to Ginq. I will put
    \textbf{Sho;Sf;Ginq} in \textbf{SpecialBehaviour} and
    \textbf{No;Ginq} in \textbf{GotCorn} (Sho Ginq Sho took one of her
    corn)
  \item
    Row 2101: The information in \textbf{Row 2101} stating
    \textbf{aggression from Pix} is not clear to whom. I will
    \textbf{delete Row 2101} (Xia Piep aggression)
  \item
    Row 2102: In \textbf{Row 2102} Xia stole corn to Piep. I will put
    \textbf{Tolerance} in \textbf{DyadResponse}, \textbf{Xia;Sf;Piep} in
    \textbf{SpecialBehvaiour} and \textbf{No;Piep} in \textbf{GotCorn}
    (Xia Piep Xia ate corn from both boxes)
  \item
    Row 2103: In \textbf{Row 2103} Xia stole corn to Piep. I will put
    \textbf{Tolerance} in \textbf{DyadResponse}, \textbf{Xia;Sf;Piep} in
    \textbf{SpecialBehvaiour} and \textbf{No;Piep} in \textbf{GotCorn}
    (Xia Piep Xia stolen some pieces from piep box)
  \item
    Row 2104: In \textbf{Row 2104} Xia stole corn to Piep. I will put
    \textbf{Tolerance} in \textbf{DyadResponse}, \textbf{Xia;Sf;Piep} in
    \textbf{SpecialBehvaiour} and \textbf{No;Piep} in \textbf{GotCorn}
    (Xia Piep Xia immediately go for piep box and eat her corn then eat
    his.)
  \item
    Row 2105: In \textbf{Row 2105} Xia stole corn to Piep. I will put
    \textbf{Tolerance} in \textbf{DyadResponse}, \textbf{Xia;Sf;Piep} iN
    \textbf{SpecialBehvaiour} and \textbf{No;Piep} in \textbf{GotCorn}
    (Xia Piep piep got 2 pieces out of 3. Xia took one corn)
  \item
    Row 2170: In \textbf{Row 2170} I will replace \textbf{DyadResponse}
    by \textbf{Female aggress male} (Piep Xia fem ag male)
  \item
    Row 2184: As both individual did not approach and \textbf{Not
    approaching} is already in DyadResponse, I will just remove
    \textbf{Other} from \textbf{DyadResponse}Xia Piep Neither ID
    approach (Xia Piep Neither ID approach)
  \item
    Row 2729: As there was a mistake in \textbf{Row 2729} I will delete
    this line: \textbf{Delete Row 2729} (Xin Ouli opened for oerw)
  \item
    Finally I will delete all the occurences of \textbf{Other} in
    DyadResponse
  \end{itemize}
\item
  First I will remove the lines with the combinations \textbf{Other \&
  No Response}
\end{itemize}

\begin{verbatim}
## Number of combinations with 'Other' in DyadResponse and 'No Response' in OtherResponse:  5
\end{verbatim}

\begin{verbatim}
## Row numbers:  39, 700, 815, 826, 1659
\end{verbatim}

\begin{verbatim}
## Number of rows that have been changed:  5
\end{verbatim}

\begin{verbatim}
## Number of occurrences of 'Other' in DyadResponse left:  13
\end{verbatim}

\begin{itemize}
\tightlist
\item
  Then I will treat the remaining 13 lines according to the rules I
  explained
\end{itemize}

\begin{verbatim}
## Number of rows changed:  0
\end{verbatim}

\begin{itemize}
\tightlist
\item
  We now have 0 remaing rows with \textbf{Others} in
  \textbf{DyadResponse} and have removed the cases where we had
  different outputs than \textbf{Other Resposne} in the \textbf{Other}
  column
\end{itemize}

\hypertarget{c-decision-making-for-dyad-response-hierarchy}{%
\subparagraph{3.9.3.c Decision making for Dyad Response
Hierarchy}\label{c-decision-making-for-dyad-response-hierarchy}}

\begin{itemize}
\item
  I will create a \textbf{Hierarchy for DyadResponse} in order to treat
  cases where multiple behaviours were produced within a trial in order
  to reduce the amount of responses and clear any discrepancies that
  could be found
\item
  \textbf{Aggression \& Tolerance}: I will keep \textbf{Agression} as we
  defined tolerance as the absence of any sign of aggression between
  individuals of a Dyad. Each Time we wrote tolerance it means that the
  monkeys touched the boxes at the same time. I will create a variable
  called \textbf{SimulatenousTouch} to record every time the box did the
  action at the same time. Then I will in every case of aggression and
  tolerance within a trial keep Aggression: \textbf{Aggression
  \textgreater{} Tolerance}. (Note that aggression is not yet a variable
  in itself but instead we have Male aggress female \& Female aggress
  male as occurences of aggression)
\end{itemize}

\begin{verbatim}
## Number of cases with Tolerance & Male aggress female:  31
\end{verbatim}

\begin{verbatim}
## Number of cases with Tolerance & Female aggress male:  24
\end{verbatim}

\begin{verbatim}
## Number of rows changed:  55
\end{verbatim}

\begin{verbatim}
## Number of lines remaining with both Aggression and Tolerance:  0
\end{verbatim}

\begin{itemize}
\tightlist
\item
  \textbf{Tolerance \& Loosing interest}: I will keep \textbf{Tolerance}
  as it means that the dyad did touch the boxes: \textbf{Tolerance
  \textgreater{} Loosing interest}
\end{itemize}

\begin{verbatim}
## Number of cases with Tolerance & Losing interest:  23
\end{verbatim}

\begin{verbatim}
## Number of rows changed:  23
\end{verbatim}

\begin{verbatim}
## Number of lines remaining with both Tolerance and Losing interest:  0
\end{verbatim}

\begin{itemize}
\tightlist
\item
  \textbf{Tolerance \& Intrusion}: I will keep \textbf{Tolerance}.
  Before replacing these occurrences I will created a variable called
  \textbf{Intrusion} the value will be Yes if there was Intrusion in
  DyadResponse (alone or with another response) and No if there wasn't
  any Intrusion: \textbf{Tolerance \textgreater{} Intrusion}
\end{itemize}

\begin{verbatim}
## Number of cases with Tolerance & Intrusion:  53
\end{verbatim}

\begin{verbatim}
## Number of rows changed:  53
\end{verbatim}

\begin{verbatim}
## Number of lines remaining with both Tolerance and Intrusion:  0
\end{verbatim}

\begin{itemize}
\tightlist
\item
  \textbf{Not approaching \& Loosing interest}: I will replace the
  occurrences of \textbf{Loosing interest} with \textbf{Not approaching}
  as individuals who did not pay/ lost interest did not come to the box
\end{itemize}

\begin{verbatim}
## Initial duplicates of 'Not Approaching':  571 
## Final duplicates of 'Not Approaching':  634 
## Final number of 'Not approaching' occurrences after replacing 'Losing interest':  634 
## Number of rows with 'Losing interest' remaining: 0
\end{verbatim}

\begin{itemize}
\tightlist
\item
  Replacement of Loosing interest with not approaching:

  \begin{enumerate}
  \def\labelenumi{\arabic{enumi}.}
  \tightlist
  \item
    When there is Loosing interest and Not approaching: keep not
    approaching
  \item
    Replace remaining occurences of Loosing interest (along or combined
    with another response, with not approaching)
  \end{enumerate}
\end{itemize}

\begin{verbatim}
## Initial duplicates of 'Not Approaching':  634 
## Final duplicates of 'Not Approaching':  634 
## Final number of 'Not approaching' occurrences after replacing 'Losing interest':  634 
## Number of rows with 'Losing interest' remaining: 0
\end{verbatim}

\begin{itemize}
\tightlist
\item
  \textbf{Replacement of intrusion by not approaching}
\item
  \textbf{Intrusion \& Not approaching}: I will keep Intrusion:
  \textbf{Intrusion \textgreater{} Not approaching}
\end{itemize}

\begin{verbatim}
## Initial duplicates of 'Not Approaching; Not Approaching':  0 
## Final duplicates of 'Not Approaching; Not Approaching':  0 
## Number of cases with Intrusion & Not Approaching:  34 
## Number of lines remaining with both Intrusion and Not Approaching:  0
\end{verbatim}

\begin{itemize}
\tightlist
\item
  \textbf{Not approaching \& Distracted}: I will keep \textbf{Not
  approaching}: \textbf{Not approaching \textgreater{} Distracted}
\end{itemize}

\begin{verbatim}
## Initial duplicates of 'Not Approaching; Not Approaching':  0 
## Final duplicates of 'Not Approaching; Not Approaching':  0 
## Number of changes made to replace 'Distracted' with 'Not approaching':  17 
## Number of changes in 'Not approaching' entries:  8
\end{verbatim}

\begin{itemize}
\tightlist
\item
  \textbf{Male Aggress Female / Female Aggress Female} and \textbf{Not
  Approaching}: I will keep both Aggression \textgreater{} Not
  approaching
\end{itemize}

\begin{verbatim}
## Number of cases with Male Aggress Female & Not Approaching:  2
\end{verbatim}

\begin{verbatim}
## Number of cases with Female Aggress Male & Not Approaching:  3
\end{verbatim}

\begin{verbatim}
## Total number of cases with Aggression & Not Approaching:  5
\end{verbatim}

\begin{verbatim}
## Number of changes made to remove 'Not Approaching' with 'Male aggress female':  2
\end{verbatim}

\begin{verbatim}
## Number of changes made to remove 'Not Approaching' with 'Female aggress male':  3
\end{verbatim}

\begin{verbatim}
## Number of remaining cases with Male Aggress Female & Not Approaching:  0
\end{verbatim}

\begin{verbatim}
## Number of remaining cases with Female Aggress Male & Not Approaching:  0
\end{verbatim}

\begin{itemize}
\item
  \begin{itemize}
  \tightlist
  \item
    \textbf{Male Aggress Female / Female Aggress Female} and
    \textbf{Tolerance}: I will keep both Tolerance \textgreater{} Male
    Aggress female/ Female Aggress Male
  \end{itemize}
\end{itemize}

\begin{verbatim}
## Number of cases with Male Aggress Female & Tolerance:  31
\end{verbatim}

\begin{verbatim}
## Number of cases with Female Aggress Male & Tolerance:  24
\end{verbatim}

\begin{verbatim}
## Number of changes made to remove 'Male aggress female' with 'Tolerance':  31
\end{verbatim}

\begin{verbatim}
## Number of changes made to remove 'Female aggress male' with 'Tolerance':  24
\end{verbatim}

\begin{verbatim}
## Number of remaining cases with Male Aggress Female & Tolerance:  0
\end{verbatim}

\begin{verbatim}
## Number of remaining cases with Female Aggress Male & Tolerance:  0
\end{verbatim}

\begin{itemize}
\tightlist
\item
  \textbf{Male aggress Female} and \textbf{Female aggress male}: create
  a variable called \textbf{Aggression} and put \textbf{Yes} for every
  occurrences of Female and Male aggression and \textbf{No} when there
  wasn't
\end{itemize}

\begin{verbatim}
## Number of cases with Male aggress female:  77
\end{verbatim}

\begin{verbatim}
## Number of cases with Female aggress male:  22
\end{verbatim}

\begin{verbatim}
## Total number of aggression cases:  99
\end{verbatim}

\begin{itemize}
\tightlist
\item
  \textbf{Tolerance \& Not approaching}: I will keep \textbf{Tolerance}
  as it means that one of the individuals took more than 30 seconds to
  come at the box and then at the end came. In these cases I will create
  a variable called \textbf{Hesistant} to count the frequency of times
  where the monkey took more time to approach than in the usual trials.
  Once the variable is created based on all the occurence of tolerance
  and not approaching, I will replace all the occurences of not
  approaching and tolerance with tolerance: \textbf{Tolerance
  \textgreater{} No approaching}
\end{itemize}

\begin{verbatim}
## Number of cases with Tolerance & Not Approaching:  30
\end{verbatim}

\begin{verbatim}
## Number of rows changed:  30
\end{verbatim}

\begin{verbatim}
## Number of lines remaining with both Tolerance and Not Approaching:  0
\end{verbatim}

\begin{itemize}
\tightlist
\item
  Treatment of mistake of not apporaching not approaching
\end{itemize}

\begin{verbatim}
## Count of 'Not approaching Not approaching':  58
\end{verbatim}

\begin{verbatim}
## Count of 'Not approaching Not approaching':  0
\end{verbatim}

\begin{verbatim}
## Count of 'Not approaching Not approaching':  0 
## Remaining duplicates of 'Not approaching Not approaching': 0
\end{verbatim}

\begin{longtable}[]{@{}llr@{}}
\caption{Table of Unique Keywords and Their Frequencies in
DyadResponse}\tabularnewline
\toprule
& Keyword & Frequency \\
\midrule
\endfirsthead
\toprule
& Keyword & Frequency \\
\midrule
\endhead
10 & Tolerance & 1882 \\
8 & Not approaching & 572 \\
4 & Intrusion & 83 \\
5 & Male aggress female & 74 \\
12 & Tolerance Intrusion & 49 \\
13 & Tolerance Not approaching & 29 \\
1 & Female aggress male & 18 \\
3 & Female aggress male Not approaching & 3 \\
11 & Tolerance Intrusion & 3 \\
7 & Male aggress female Not approaching & 2 \\
2 & Female aggress male Intrusion & 1 \\
6 & Male aggress female Intrusion & 1 \\
9 & Not approaching Intrusion & 1 \\
14 & Tolerance Not approaching Intrusion & 1 \\
\bottomrule
\end{longtable}

\begin{verbatim}
## Warning in RColorBrewer::brewer.pal(n, pal): n too large, allowed maximum for palette Set3 is 12
## Returning the palette you asked for with that many colors
\end{verbatim}

\includegraphics{BEX2223_files/figure-latex/check and state of dyadresponse-1.pdf}

\begin{longtable}[]{@{}lr@{}}
\caption{Table of Unique DyadResponse Entries with More Than 10
Occurrences}\tabularnewline
\toprule
DyadResponse & n \\
\midrule
\endfirsthead
\toprule
DyadResponse & n \\
\midrule
\endhead
Tolerance & 1882 \\
Not approaching & 572 \\
Intrusion & 83 \\
Male aggress female & 74 \\
Tolerance Intrusion & 49 \\
Tolerance Not approaching & 29 \\
Female aggress male & 18 \\
\bottomrule
\end{longtable}

\includegraphics{BEX2223_files/figure-latex/check and state of dyadresponse-2.pdf}

\hypertarget{visualizing-the-data-general-overview}{%
\section{4.Visualizing the data (General
Overview)}\label{visualizing-the-data-general-overview}}

\hypertarget{reordering-the-variables-in-bex}{%
\subsection{4.1 Reordering the variables in
Bex}\label{reordering-the-variables-in-bex}}

\begin{itemize}
\item
  Now that all the variables have been treated and cleaned I want to
  modify thew variables in Bex to keep only the ones I need and to put
  them in the right order
\item
  First lets print the name of all the variables before reordering them
  and keeping only the variables of interest
\end{itemize}

\begin{verbatim}
## tibble [2,719 x 37] (S3: tbl_df/tbl/data.frame)
##  $ Time                  : chr [1:2719] "08:18:23" "08:37:12" "08:37:58" "08:16:30" ...
##  $ Date                  : chr [1:2719] "2022-09-29" "2022-09-29" "2022-09-29" "2022-10-04" ...
##  $ Group                 : chr [1:2719] "Ankhase" "Ankhase" "Ankhase" "Ankhase" ...
##  $ MaleID                : chr [1:2719] "Buk" "Buk" "Buk" "Buk" ...
##  $ FemaleID              : chr [1:2719] "Ndaw" "Ndaw" "Ndaw" "Ndaw" ...
##  $ FemaleCorn            : chr [1:2719] "0" "0" "0" "3" ...
##  $ DyadDistance          : chr [1:2719] "5" "5" "5" "5" ...
##  $ DyadResponse          : chr [1:2719] "Tolerance" "Tolerance" "Intrusion" "Intrusion" ...
##  $ OtherResponse         : chr [1:2719] "No Response" "No Response" "No Response" "No Response" ...
##  $ Audience              : chr [1:2719] "Ginq; Gubh" "Ginq; Gubh" "Ginq; Gubh" "Ghid; Gil; Ginq; Gom" ...
##  $ IDIndividual1         : chr [1:2719] "No individual" "No individual" "Buk; Ndaw" "Buk" ...
##  $ IntruderID            : chr [1:2719] "No Intrusion" "No Intrusion" "Ginq; Gubh" "Ginq" ...
##  $ Remarks               : chr [1:2719] "No Remarks" "No Remarks" "Treated" "Treated" ...
##  $ MaleCorn              : chr [1:2719] "0" "0" "0" "7" ...
##  $ Intrusion             : chr [1:2719] "0" "0" "1" "0" ...
##  $ AmountAudience        : chr [1:2719] "2" "2" "2" "4" ...
##  $ Context               : chr [1:2719] "NoContext" "NoContext" "NoContext" "NoContext" ...
##  $ SpecialBehaviour      : chr [1:2719] "NoSpecialBehaviour" "NoSpecialBehaviour" "NoSpecialBehaviour" "NoSpecialBehaviour" ...
##  $ GotCorn               : chr [1:2719] "Yes" "Yes" "Yes" "Yes" ...
##  $ Period                : chr [1:2719] "6 to 8" "6 to 8" "6 to 8" "6 to 8" ...
##  $ Hour                  : chr [1:2719] "08:00:00" "08:00:00" "08:00:00" "08:00:00" ...
##  $ Day                   : chr [1:2719] "16" "16" "16" "21" ...
##  $ Month                 : chr [1:2719] "2022-09" "2022-09" "2022-09" "2022-10" ...
##  $ Male                  : chr [1:2719] "Buk" "Buk" "Buk" "Buk" ...
##  $ Female                : chr [1:2719] "Ndaw" "Ndaw" "Ndaw" "Ndaw" ...
##  $ Dyad                  : chr [1:2719] "Buk Ndaw" "Buk Ndaw" "Buk Ndaw" "Buk Ndaw" ...
##  $ DaysSinceStart        : chr [1:2719] "16" "16" "16" "21" ...
##  $ ExperimentDay         : chr [1:2719] "6" "6" "6" "7" ...
##  $ ExperimentDay_Verified: chr [1:2719] "6" "6" "6" "7" ...
##  $ Trial                 : chr [1:2719] "1" "2" "3" "4" ...
##  $ DyadDay               : chr [1:2719] "1" "1" "1" "2" ...
##  $ TrialDay              : chr [1:2719] "1" "2" "3" "1" ...
##  $ PlacementMale         : chr [1:2719] "0" "0" "0" "7" ...
##  $ PlacementFemale       : chr [1:2719] "0" "0" "0" "3" ...
##  $ Proximity             : chr [1:2719] "4-5" "4-5" "4-5" "4-5" ...
##  $ DyadResponse_sorted   : chr [1:2719] "Tolerance" "Tolerance" "Intrusion;Not approaching" "Intrusion;Losing interest" ...
##  $ MultipleResponses     : chr [1:2719] "Single Response" "Single Response" ">1 Response" ">1 Response" ...
\end{verbatim}

\begin{verbatim}
##                   Time                   Date                  Group 
##                      0                      0                      0 
##                 MaleID               FemaleID             FemaleCorn 
##                      0                      0                      0 
##           DyadDistance           DyadResponse          OtherResponse 
##                      0                      0                      0 
##               Audience          IDIndividual1             IntruderID 
##                      0                      0                      0 
##                Remarks               MaleCorn              Intrusion 
##                      0                      0                      0 
##         AmountAudience                Context       SpecialBehaviour 
##                      0                      0                      0 
##                GotCorn                 Period                   Hour 
##                      0                      0                      0 
##                    Day                  Month                   Male 
##                      0                      0                      0 
##                 Female                   Dyad         DaysSinceStart 
##                      0                      0                      0 
##          ExperimentDay ExperimentDay_Verified                  Trial 
##                      0                      0                      0 
##                DyadDay               TrialDay          PlacementMale 
##                      0                      0                      0 
##        PlacementFemale              Proximity    DyadResponse_sorted 
##                      0                      0                      0 
##      MultipleResponses 
##                      0
\end{verbatim}

\begin{itemize}
\item
  Before reordering these variables I may have to re update them to make
  sure all the changes that occurred between the creation of this
  variable and now, did not create any errors or mistakes
\item
  Finally I will keep the following variables in the next order: Date,
  Month, Day, Time, Period, Trial, Male, Female, Dyad, DyadDistance,
  Proximity, DyadResponse, Intrusion, IntruderID SpecialBehaviour,
  Audience, AmountAudience, Context, Special Behaviour
\end{itemize}

\begin{verbatim}
##  [1] "DaysSinceStart"   "ExperimentDay"    "Date"             "Month"           
##  [5] "DyadDay"          "TrialDay"         "Trial"            "Time"            
##  [9] "Hour"             "Period"           "Male"             "Female"          
## [13] "Dyad"             "DyadDistance"     "Proximity"        "DyadResponse"    
## [17] "IDIndividual1"    "Intrusion"        "IntruderID"       "SpecialBehaviour"
## [21] "Audience"         "AmountAudience"   "Context"
\end{verbatim}

\hypertarget{exploratory-graph-to-organise}{%
\subsection{4.2 Exploratory Graph (To
organise)}\label{exploratory-graph-to-organise}}

\#\#\#\#Dyad, Distance \& Date

\begin{itemize}
\item
  Trials of grpahs, I will have to check all of them
\item
  My goal here is too see if each dyad have an general evolution of
  their dyad distance trough time and how many varaition do they have
\end{itemize}

\begin{verbatim}
## `geom_smooth()` using formula = 'y ~ x'
\end{verbatim}

\begin{verbatim}
## Warning: Removed 65 rows containing non-finite outside the scale range
## (`stat_smooth()`).
\end{verbatim}

\includegraphics{BEX2223_files/figure-latex/Dyad Distance \& Date-1.pdf}

\begin{verbatim}
## Warning: Using `size` aesthetic for lines was deprecated in ggplot2 3.4.0.
## i Please use `linewidth` instead.
## This warning is displayed once every 8 hours.
## Call `lifecycle::last_lifecycle_warnings()` to see where this warning was
## generated.
\end{verbatim}

\begin{verbatim}
## `geom_smooth()` using formula = 'y ~ x'
\end{verbatim}

\includegraphics{BEX2223_files/figure-latex/Other graphs-1.pdf}

\includegraphics{BEX2223_files/figure-latex/daily graph-1.pdf}

\includegraphics{BEX2223_files/figure-latex/last test before sleep-1.pdf}

\includegraphics{BEX2223_files/figure-latex/daily graph2-1.pdf}

\begin{itemize}
\tightlist
\item
  Check amount of audience link with aggression occurences per sex
\end{itemize}

\includegraphics{BEX2223_files/figure-latex/tests on agression audience and time-1.pdf}
\includegraphics{BEX2223_files/figure-latex/tests on agression audience and time-2.pdf}

\begin{verbatim}
## Warning: There was 1 warning in `mutate()`.
## i In argument: `KinPresent = ifelse(...)`.
## Caused by warning in `grepl()`:
## ! l'argument pattern a une longueur > 1 et seul le premier élément est utilisé
\end{verbatim}

\includegraphics{BEX2223_files/figure-latex/tests on agression audience and time-3.pdf}
\includegraphics{BEX2223_files/figure-latex/tests on agression audience and time-4.pdf}
\includegraphics{BEX2223_files/figure-latex/tests on agression audience and time-5.pdf}

\hypertarget{preparation-of-dataset-for-importation-of-life-history-elo-rating-focal-and-ad-libitum-data}{%
\subsection{4.3 Preparation of dataset for importation of life history,
elo rating \& focal and ad libitum
data}\label{preparation-of-dataset-for-importation-of-life-history-elo-rating-focal-and-ad-libitum-data}}

\hypertarget{intermediate-check-of-audience-and-extraction-of-names}{%
\subsubsection{4.3.1 Intermediate check of audience and extraction of
names}\label{intermediate-check-of-audience-and-extraction-of-names}}

\begin{itemize}
\item
  Because I will look at the audience and individual factors on the
  individuals of the dyads i must make a lsit of all the individudals
  for which I will need to extract information.
\item
  First lets extract all of the names of individuals in audience
\item
  Step 1: Clean and Process the Audience Data: We first clean the
  Audience column, ensuring that any ``NA'' values are replaced with
  ``No audience'' and split the combined names. Whitespace around names
  is also removed.
\item
  Step 2: Recalculate AmountAudience: This step ensures the
  AmountAudience variable is correctly updated after splitting the
  Audience column.
\item
  Step 3: Summarize Changes: We count how many times the value ``NA''
  was replaced with ``No audience'' and display this count using cat().
\item
  Step 4: Categorize Audience Members: We categorize audience members
  into ``Male'', ``Female'', and other relevant categories based on the
  length of their IDs. Specific identifiers like ``unkam'', ``unkaf'',
  etc., are given appropriate labels.
\end{itemize}

Step 5: Display Tables by Category: We split the audience count table
into sections for ``Male'', ``Female'', and other categories, and then
display these tables separately. The use of knitr::kable() provides a
professional table format that is more suitable for RMarkdown.

\begin{verbatim}
## 
## **Table: Top 5 Male Audience Members**
\end{verbatim}

\begin{longtable}[]{@{}lc@{}}
\caption{Top 5 Male Audience Members}\tabularnewline
\toprule
Audience Member & Count \\
\midrule
\endfirsthead
\toprule
Audience Member & Count \\
\midrule
\endhead
Pix & 177 \\
Oup & 165 \\
Nko & 123 \\
Gom & 103 \\
Ome & 103 \\
\bottomrule
\end{longtable}

\begin{verbatim}
## 
## **Table: Top 5 Female Audience Members**
\end{verbatim}

\begin{longtable}[]{@{}lc@{}}
\caption{Top 5 Female Audience Members}\tabularnewline
\toprule
Audience Member & Count \\
\midrule
\endfirsthead
\toprule
Audience Member & Count \\
\midrule
\endhead
Sirk & 241 \\
Oort & 199 \\
Oerw & 158 \\
Piep & 154 \\
Ginq & 145 \\
\bottomrule
\end{longtable}

\begin{verbatim}
## 
## **List: All Male Audience Members**
\end{verbatim}

\begin{verbatim}
## Pix, Oup, Nko, Gom, Ome, Sey, Nda, Ndl, Gha, Xia, Sig, Xin, Rid, Gib, Gub, Mui, Kom, Dix, Nak, Buk, Aan, Non, Aal, Sho, Guz, Pie, Hee, Pom, Syl, Nuk, App, Gri, Ott, Ree, Gil, Kno, Xar, Goe, Ask, UnkAM, Nuu, Ros, Vla, Xop, Bet, Roc, Hem, Nge, Eis, Umb, Ren, Rim, War, Atj, Gua, Vul, Bob, Nca, Tot, Her, Ram, Dal, Tch, Ris, Vry, Dak, Uls, Flu, Gab, Win
\end{verbatim}

\begin{verbatim}
## 
## **List: All Female Audience Members**
\end{verbatim}

\begin{verbatim}
## Sirk, Oort, Oerw, Piep, Ginq, Obse, Ghid, Reen, Gubh, Ndaw, Sitr, Godu, Skem, Ndum, Naal, Ncok, Miel, Gobe, Nkos, Ouli, Ndon, Ndin, Puol, Giji, Lewe, Enge, Papp, Eina, Hond, Rimp, Heer, Sari, Gree, Olyf, Bela, Aapi, Guba, Popp, Oase, Xeni, Nurk, Haai, Rivi, Gran, Misk, Nooi, Prat, Gris, Pikk, Pann, Prai, Riss, Riva, Ekse, Rede, Griv, Regi, UnkJ, Xati, Rooi, Gaya, Prim, Rafa, Udup, Xala, Palm, Xinp, Guat, Reno, Beir, Gese, Grim, UnkA, Gale, Pret, Prag, Prui, Raba, Rioj, UnkAF, Utic
\end{verbatim}

\includegraphics{BEX2223_files/figure-latex/Audience first insghts ID-1.pdf}
\includegraphics{BEX2223_files/figure-latex/Audience first insghts ID-2.pdf}

\begin{verbatim}
## 
## ### Top 10 Audience Members for Dyad: Buk Ndaw 
## 
## 
## |Audience Member | Count|
## |:---------------|-----:|
## |Ginq            |    90|
## |Ghid            |    62|
## |No audience     |    59|
## |Nda             |    47|
## |Ndl             |    42|
## |Sho             |    40|
## |Gha             |    37|
## |Gom             |    36|
## |Gubh            |    35|
## |Godu            |    29|
## 
## ### Top 10 Audience Members for Dyad: Kom Oort 
## 
## 
## |Audience Member | Count|
## |:---------------|-----:|
## |No audience     |    95|
## |Sirk            |    51|
## |Xia             |    50|
## |Piep            |    47|
## |Reen            |    35|
## |Pix             |    29|
## |Sey             |    29|
## |Ome             |    25|
## |Oup             |    22|
## |Xin             |    21|
## 
## ### Top 10 Audience Members for Dyad: Nge Oerw 
## 
## 
## |Audience Member | Count|
## |:---------------|-----:|
## |No audience     |    55|
## |Sirk            |    54|
## |Oup             |    40|
## |Pix             |    28|
## |Sey             |    19|
## |Sig             |    18|
## |Obse            |    16|
## |Sitr            |    13|
## |Ouli            |    12|
## |Aan             |    11|
## 
## ### Top 10 Audience Members for Dyad: Pom Xian 
## 
## 
## |Audience Member | Count|
## |:---------------|-----:|
## |No audience     |   115|
## |Gri             |    31|
## |Gree            |    26|
## |Xar             |    23|
## |Xeni            |    21|
## |Gran            |    18|
## |Xop             |    17|
## |Prat            |    16|
## |Gris            |    15|
## |Miel            |    15|
## |Roc             |    15|
## 
## ### Top 10 Audience Members for Dyad: Sey Sirk 
## 
## 
## |Audience Member | Count|
## |:---------------|-----:|
## |No audience     |   250|
## |Piep            |    63|
## |Oort            |    51|
## |Oerw            |    48|
## |Pix             |    48|
## |Oup             |    44|
## |Ome             |    39|
## |Reen            |    37|
## |Obse            |    34|
## |Sitr            |    29|
## 
## ### Top 10 Audience Members for Dyad: Sho Ginq 
## 
## 
## |Audience Member | Count|
## |:---------------|-----:|
## |Ndaw            |    90|
## |Gom             |    67|
## |Gubh            |    61|
## |No audience     |    59|
## |Gha             |    54|
## |Ghid            |    49|
## |Buk             |    48|
## |Godu            |    46|
## |Gib             |    42|
## |Ndum            |    37|
## 
## ### Top 10 Audience Members for Dyad: Xia Piep 
## 
## 
## |Audience Member | Count|
## |:---------------|-----:|
## |No audience     |   178|
## |Sirk            |   124|
## |Oort            |    90|
## |Pix             |    56|
## |Oerw            |    51|
## |Xin             |    37|
## |Naal            |    32|
## |Sitr            |    32|
## |Obse            |    31|
## |Dix             |    29|
## 
## ### Top 10 Audience Members for Dyad: Xin Ouli 
## 
## 
## |Audience Member | Count|
## |:---------------|-----:|
## |Oerw            |    45|
## |Oort            |    39|
## |Sey             |    38|
## |No audience     |    34|
## |Oup             |    30|
## |Piep            |    29|
## |Aal             |    17|
## |Xia             |    17|
## |Obse            |    16|
## |Ott             |    16|
## |Pix             |    16|
\end{verbatim}

\hypertarget{intermediate-check-of-intrusion-extraction-of-names}{%
\subsubsection{4.3.2 Intermediate check of Intrusion, extraction of
names}\label{intermediate-check-of-intrusion-extraction-of-names}}

\includegraphics{BEX2223_files/figure-latex/Intruder IDs-1.pdf}
\includegraphics{BEX2223_files/figure-latex/Intruder IDs-2.pdf}
\includegraphics{BEX2223_files/figure-latex/Intruder IDs-3.pdf}

\begin{verbatim}
## [1] "Ordered Unique Intruder Names:"
\end{verbatim}

\begin{verbatim}
##  [1] "Buk"  "Ghid" "Ginq" "Godu" "Gran" "Gree" "Gri"  "Grif" "Gris" "Guat"
## [11] "Gub"  "Gubh" "Guz"  "Hee"  "Kno"  "Kom"  "Nak"  "Nda"  "Nge"  "Non" 
## [21] "Obse" "Oerw" "Oort" "Ouli" "Oup"  "Piep" "Pix"  "Sey"  "Sho"  "Sirk"
## [31] "Xia"  "Xin"  "Xop"
\end{verbatim}

\begin{verbatim}
## [1] "Top 10 Most Frequent Intruder Names:"
\end{verbatim}

\begin{verbatim}
##    Name Count Gender
## 1  Oerw    23 Female
## 2   Sey    19   Male
## 3  Oort    15 Female
## 4  Obse    10 Female
## 5  Ghid     9 Female
## 6   Buk     5   Male
## 7  Ginq     5 Female
## 8  Gubh     5 Female
## 9   Gri     4   Male
## 10 Piep     4 Female
\end{verbatim}

\begin{verbatim}
## [1] "Male Names (3 letters):"
\end{verbatim}

\begin{verbatim}
##  [1] "Guz" "Nda" "Sho" "Sey" "Pix" "Xia" "Oup" "Nak" "Gri" "Xop" "Nge" "Kno"
## [13] "Kom" "Buk" "Gub" "Xin" "Hee" "Non"
\end{verbatim}

\begin{verbatim}
## [1] "Female Names (4 letters):"
\end{verbatim}

\begin{verbatim}
##  [1] "Ginq" "Gubh" "Ghid" "Oerw" "Obse" "Piep" "Ouli" "Gree" "Gran" "Gris"
## [11] "Grif" "Guat" "Sirk" "Oort" "Godu"
\end{verbatim}

\begin{itemize}
\tightlist
\item
  After comparaison every individual that appearead in intrusion was
  also in audience except for \textbf{Grif} that i may add to the lsit
  of invididuals of audience before doing elo rating calculations
\end{itemize}

\hypertarget{intermediate-check-of-not-approaching-behaviour}{%
\subsubsection{4.3.3 Intermediate check of Not Approaching
behaviour\$}\label{intermediate-check-of-not-approaching-behaviour}}

\includegraphics{BEX2223_files/figure-latex/Intermediate check of not approaching-1.pdf}

\hypertarget{intermediate-check-of-intrusion}{%
\subsubsection{4.3.4 Intermediate check of
Intrusion}\label{intermediate-check-of-intrusion}}

\includegraphics{BEX2223_files/figure-latex/Intermediate check of intrusion-1.pdf}

\begin{verbatim}
## `summarise()` has grouped output by 'Dyad'. You can override using the
## `.groups` argument.
\end{verbatim}

\includegraphics{BEX2223_files/figure-latex/Intermediate check of intrusion-2.pdf}

\hypertarget{evolution-of-dyadresponse-per-dyad}{%
\subsubsection{4.3.5 Evolution of DyadResponse per
Dyad}\label{evolution-of-dyadresponse-per-dyad}}

Behaviours tests

\begin{verbatim}
## `summarise()` has grouped output by 'Dyad'. You can override using the
## `.groups` argument.
\end{verbatim}

\includegraphics{BEX2223_files/figure-latex/behaviours graphs tests-1.pdf}
\includegraphics{BEX2223_files/figure-latex/behaviours graphs tests-2.pdf}

\hypertarget{checkpoint---export-data-as-rds-and-xlslx}{%
\section{5.Checkpoint - Export Data as Rds and
Xlslx}\label{checkpoint---export-data-as-rds-and-xlslx}}

\hypertarget{importing-data-for-new-varaibles-age-elo-rating-dsi}{%
\section{6 Importing data for new varaibles (Age, Elo Rating,
DSI)}\label{importing-data-for-new-varaibles-age-elo-rating-dsi}}

\begin{itemize}
\tightlist
\item
  I want to investigate how intra-dyadic differences can explain
  inter-dyadic differences on tolerances levels during the box
  experiment
\item
  For that I will calculate the age of individuals , their rank using
  elo rating, and their social bond using the dyadic composite social
  index
\item
  I will use data files from Inkawu Vervet Project (IVP) that contain
  longterm data as on age, agonistic and affiliative ineractions as for
  proximity data amon other.
\end{itemize}

\hypertarget{assessing-age-of-individuals}{%
\subsection{6.1. Assessing Age of
Individuals}\label{assessing-age-of-individuals}}

\begin{itemize}
\tightlist
\item
  Female DOB: Use the direct DOB from LH, except for Ouli.
\item
  Male DOB: Use FirstRecorded - 4 years unless they have a recorded DOB.
\item
  Age Calculation: We'll calculate the ages for both using the DOBs or
  estimated DOBs.
\end{itemize}

\hypertarget{age-of-females}{%
\subsubsection{6.1.2 Age of Females}\label{age-of-females}}

\begin{verbatim}
##   Code        DOB FirstRecorded AdjustedDOB       Age
## 1 Ouli       <NA>    2010-11-09  2006-11-09 18.149586
## 2 Xian 2012-11-05    2012-11-05  2012-11-05 12.159045
## 3 Piep 2012-01-01    2013-01-18  2012-01-01 13.005058
## 4 Ginq 2014-10-18    2014-10-18  2014-10-18 10.209655
## 5 Oort 2015-11-20    2015-11-20  2015-11-20  9.119968
## 6 Ndaw 2016-02-08    2016-02-08  2016-02-08  8.900936
## 7 Sirk 2017-10-21    2017-10-21  2017-10-21  7.200695
## 8 Oerw 2018-11-09    2018-11-09  2018-11-09  6.149339
\end{verbatim}

\includegraphics{BEX2223_files/figure-latex/Female Age-1.pdf}

\hypertarget{age-of-male}{%
\subsubsection{6.1.3 Age of Male}\label{age-of-male}}

\begin{itemize}
\item
  Since most males dispsersed from unkonwn groups we dont have their
  date of birth. But male usually dispers around 5 years old.
\item
  Male DOB: Use FirstRecorded - 5 years, except for a few individuals
  who have DOB recorded (like Xia and Xin)
\item
  Use DOB directly if it exists.
\item
  Subtract 5 years from FirstRecorded if DOB is missing.
\item
  Calculate age based on AdjustedDOB.
\item
  Output a table with the relevant columns and create a graph * similar
  to the one you had for the females.
\item
  \textbf{NOTE FOR UPDATE}, \textbf{the individual with the code Buk,
  short for BukuBuku died on the 12 of september 2024}
\end{itemize}

\begin{verbatim}
##   Code        DOB FirstRecorded AdjustedDOB       Age
## 1  Sey 2014-01-01    2014-12-31  2014-01-01 11.003648
## 2  Xia 2016-11-14    2016-11-14  2016-11-14  8.134322
## 3  Nge 2016-11-18    2016-11-18  2016-11-18  8.123370
## 4  Kom       <NA>    2017-09-04  2012-09-04 12.328795
## 5  Pom 2017-10-19    2017-10-19  2017-10-19  7.206171
## 6  Xin 2017-01-01    2017-10-20  2017-01-01  8.002902
## 7  Sho       <NA>    2020-10-13  2015-10-13  9.224009
## 8  Buk       <NA>    2021-05-11  2016-05-11  8.646310
\end{verbatim}

\includegraphics{BEX2223_files/figure-latex/Male Age-1.pdf}

\hypertarget{dyadic-age-difference}{%
\subsubsection{6.1.4 Dyadic Age
Difference}\label{dyadic-age-difference}}

\begin{itemize}
\tightlist
\item
  Age Comparison: Calculate the age for both males and females based on
  their estimated or recorded DOB.
\item
  Lets explore male femalée ages before calculating differences between
  dyads
\end{itemize}

\begin{verbatim}
##    Code        DOB FirstRecorded AdjustedDOB       Age Gender
## 1  Ouli       <NA>    2010-11-09  2006-11-09 18.149586 Female
## 2  Xian 2012-11-05    2012-11-05  2012-11-05 12.159045 Female
## 3  Piep 2012-01-01    2013-01-18  2012-01-01 13.005058 Female
## 4  Ginq 2014-10-18    2014-10-18  2014-10-18 10.209655 Female
## 5  Oort 2015-11-20    2015-11-20  2015-11-20  9.119968 Female
## 6  Ndaw 2016-02-08    2016-02-08  2016-02-08  8.900936 Female
## 7  Sirk 2017-10-21    2017-10-21  2017-10-21  7.200695 Female
## 8  Oerw 2018-11-09    2018-11-09  2018-11-09  6.149339 Female
## 9   Sey 2014-01-01    2014-12-31  2014-01-01 11.003648   Male
## 10  Xia 2016-11-14    2016-11-14  2016-11-14  8.134322   Male
## 11  Nge 2016-11-18    2016-11-18  2016-11-18  8.123370   Male
## 12  Kom       <NA>    2017-09-04  2012-09-04 12.328795   Male
## 13  Pom 2017-10-19    2017-10-19  2017-10-19  7.206171   Male
## 14  Xin 2017-01-01    2017-10-20  2017-01-01  8.002902   Male
## 15  Sho       <NA>    2020-10-13  2015-10-13  9.224009   Male
## 16  Buk       <NA>    2021-05-11  2016-05-11  8.646310   Male
\end{verbatim}

\includegraphics{BEX2223_files/figure-latex/Age in male and female-1.pdf}

\hypertarget{absolute-age-difference}{%
\subsubsection{6.1.5 Absolute Age
Difference}\label{absolute-age-difference}}

\begin{itemize}
\item
  We are going to calculate the raw differnce of agin between each male
  and female in the dyad displaying the absolute age difference ordered
\item
  Before that i will create a variable called DyadData with information
  the each dyad
\end{itemize}

\begin{verbatim}
## # A tibble: 8 x 6
##   Dyad     Male  Female MaleAge FemaleAge AgeDifference
##   <chr>    <chr> <chr>    <dbl>     <dbl>         <dbl>
## 1 Sey Sirk Sey   Sirk     11.0       7.20         3.80 
## 2 Xia Piep Xia   Piep      8.13     13.0          4.87 
## 3 Nge Oerw Nge   Oerw      8.12      6.15         1.97 
## 4 Sho Ginq Sho   Ginq      9.22     10.2          0.986
## 5 Xin Ouli Xin   Ouli      8.00     18.1         10.1  
## 6 Buk Ndaw Buk   Ndaw      8.65      8.90         0.255
## 7 Kom Oort Kom   Oort     12.3       9.12         3.21 
## 8 Pom Xian Pom   Xian      7.21     12.2          4.95
\end{verbatim}

\includegraphics{BEX2223_files/figure-latex/Absolute Age Difference-1.pdf}

\hypertarget{relative-age-difference}{%
\subsubsection{6.1.6 Relative Age
Difference}\label{relative-age-difference}}

\begin{itemize}
\tightlist
\item
  (Check if necessary) We are going to normalize the average age of to
  ID in a dyad
\end{itemize}

\includegraphics{BEX2223_files/figure-latex/Relative Age Difference-1.pdf}

\hypertarget{in-order-to-compare-the-age-difference-accross-dyads-we-are-going-to-calculate-the-z-scores-of-the-aboslute-age}{%
\subsubsection{6.1.7 In order to compare the age difference accross
dyads we are going to calculate the z-scores of the aboslute
age}\label{in-order-to-compare-the-age-difference-accross-dyads-we-are-going-to-calculate-the-z-scores-of-the-aboslute-age}}

\includegraphics{BEX2223_files/figure-latex/zscores age-1.pdf}

\hypertarget{age-direction}{%
\subsubsection{6.1.7 Age direction}\label{age-direction}}

\begin{itemize}
\tightlist
\item
  For further analysis we may consider if the male or female was older
\end{itemize}

\begin{verbatim}
## # A tibble: 8 x 4
##   Male  Female AgeDifference AgeDirection
##   <chr> <chr>          <dbl> <chr>       
## 1 Sey   Sirk           3.80  Male Older  
## 2 Xia   Piep           4.87  Female Older
## 3 Nge   Oerw           1.97  Male Older  
## 4 Sho   Ginq           0.986 Female Older
## 5 Xin   Ouli          10.1   Female Older
## 6 Buk   Ndaw           0.255 Female Older
## 7 Kom   Oort           3.21  Male Older  
## 8 Pom   Xian           4.95  Female Older
\end{verbatim}

\includegraphics{BEX2223_files/figure-latex/Age direction-1.pdf}

\hypertarget{age-hypothesis}{%
\subsubsection{6.1.8 Age Hypothesis}\label{age-hypothesis}}

\begin{itemize}
\tightlist
\item
  A. Age Hypothesis: \textbf{We expect that the higher the age
  difference in a dyad, the less likely they are to reach tolerance
  compared to dyad with a smaller difference. In addition we will check
  if there is an effect of Age Direction (Wether the male or the female
  is older)}
\end{itemize}

\hypertarget{creation-of-binomial-tolerance}{%
\paragraph{6.1.8.1 Creation of Binomial
tolerance}\label{creation-of-binomial-tolerance}}

\begin{itemize}
\tightlist
\item
  In order to check the effect of age on tolerance I will create a
  dichotomic variable displaying \textbf{1 if there is tolerance} and
  \textbf{0 if there is not tolerance}
\end{itemize}

\begin{verbatim}
## # A tibble: 8 x 6
##   Dyad     TotalTrials ToleranceCount NoToleranceCount ToleranceProportion
##   <chr>          <int>          <dbl>            <dbl>               <dbl>
## 1 Buk Ndaw         258            145              113               0.562
## 2 Kom Oort         367            305               62               0.831
## 3 Nge Oerw         182            121               61               0.665
## 4 Pom Xian         257            171               86               0.665
## 5 Sey Sirk         586            401              185               0.684
## 6 Sho Ginq         281            158              123               0.562
## 7 Xia Piep         603            473              130               0.784
## 8 Xin Ouli         185            108               77               0.584
## # i 1 more variable: NoToleranceProportion <dbl>
\end{verbatim}

\includegraphics{BEX2223_files/figure-latex/visualization of tolerance rates-1.pdf}
\includegraphics{BEX2223_files/figure-latex/visualization of tolerance rates-2.pdf}

\hypertarget{section-1}{%
\subsection{}\label{section-1}}

\begin{verbatim}
## # A tibble: 8 x 10
##   Dyad     Male  Female AgeDifference AgeDirection TotalTrials ToleranceCount
##   <chr>    <chr> <chr>          <dbl> <chr>              <int>          <dbl>
## 1 Sey Sirk Sey   Sirk           3.80  Male Older           586            401
## 2 Xia Piep Xia   Piep           4.87  Female Older         603            473
## 3 Nge Oerw Nge   Oerw           1.97  Male Older           182            121
## 4 Sho Ginq Sho   Ginq           0.986 Female Older         281            158
## 5 Xin Ouli Xin   Ouli          10.1   Female Older         185            108
## 6 Buk Ndaw Buk   Ndaw           0.255 Female Older         258            145
## 7 Kom Oort Kom   Oort           3.21  Male Older           367            305
## 8 Pom Xian Pom   Xian           4.95  Female Older         257            171
## # i 3 more variables: NoToleranceCount <dbl>, ToleranceProportion <dbl>,
## #   NoToleranceProportion <dbl>
\end{verbatim}

\begin{verbatim}
## tibble [8 x 10] (S3: tbl_df/tbl/data.frame)
##  $ Dyad                 : chr [1:8] "Sey Sirk" "Xia Piep" "Nge Oerw" "Sho Ginq" ...
##  $ Male                 : chr [1:8] "Sey" "Xia" "Nge" "Sho" ...
##  $ Female               : chr [1:8] "Sirk" "Piep" "Oerw" "Ginq" ...
##  $ AgeDifference        : num [1:8] 3.803 4.871 1.974 0.986 10.147 ...
##  $ AgeDirection         : chr [1:8] "Male Older" "Female Older" "Male Older" "Female Older" ...
##  $ TotalTrials          : int [1:8] 586 603 182 281 185 258 367 257
##  $ ToleranceCount       : num [1:8] 401 473 121 158 108 145 305 171
##  $ NoToleranceCount     : num [1:8] 185 130 61 123 77 113 62 86
##  $ ToleranceProportion  : num [1:8] 0.684 0.784 0.665 0.562 0.584 ...
##  $ NoToleranceProportion: num [1:8] 0.316 0.216 0.335 0.438 0.416 ...
\end{verbatim}

\hypertarget{age-tolerance}{%
\subsubsection{6.1.9 Age \& Tolerance}\label{age-tolerance}}

\begin{itemize}
\tightlist
\item
  I want to test wether age can predict tolerance and if there is an
  effect of age direction. In addition maybe track the effect of both
  Age difference and direction on tolerance
\end{itemize}

\hypertarget{section-2}{%
\paragraph{6.1.9.1}\label{section-2}}

\begin{itemize}
\tightlist
\item
  Model 1: Absolute age difference \textasciitilde{} tolerance
\end{itemize}

\begin{verbatim}
## `geom_smooth()` using formula = 'y ~ x'
\end{verbatim}

\includegraphics{BEX2223_files/figure-latex/Absolute age difference \& tolerance-1.pdf}

\begin{verbatim}
## 
##  Pearson's product-moment correlation
## 
## data:  dyad_summary$AgeDifference and dyad_summary$ToleranceProportion
## t = 0.19274, df = 6, p-value = 0.8535
## alternative hypothesis: true correlation is not equal to 0
## 95 percent confidence interval:
##  -0.6628703  0.7420961
## sample estimates:
##        cor 
## 0.07844459
\end{verbatim}

\begin{verbatim}
## 
## Call:
## lm(formula = ToleranceProportion ~ AgeDifference, data = dyad_summary)
## 
## Residuals:
##       Min        1Q    Median        3Q       Max 
## -0.099630 -0.096714 -0.001366  0.041322  0.165240 
## 
## Coefficients:
##               Estimate Std. Error t value Pr(>|t|)    
## (Intercept)   0.657686   0.062557  10.513 4.35e-05 ***
## AgeDifference 0.002536   0.013155   0.193    0.854    
## ---
## Signif. codes:  0 '***' 0.001 '**' 0.01 '*' 0.05 '.' 0.1 ' ' 1
## 
## Residual standard error: 0.1076 on 6 degrees of freedom
## Multiple R-squared:  0.006154,   Adjusted R-squared:  -0.1595 
## F-statistic: 0.03715 on 1 and 6 DF,  p-value: 0.8535
\end{verbatim}

\begin{itemize}
\item
  We are going to conduct a Spearman's rank correlation because we don't
  have a normal distribution and this methods considers the order of the
  data
\item
  Comparing with Previous Results Earlier Pearson's Correlation:
  Correlation Coefficient (r): Approximately 0.0784 P-value: 0.8535
  Interpretation: No significant linear relationship. Spearman's
  Correlation: Correlation Coefficient (rho): 0.4286 P-value: 0.2992
  Interpretation: Suggests a moderate monotonic relationship but not
  statistically significant. Why the Difference? Pearson's Correlation:
  Measures linear relationships. Sensitive to outliers and requires
  normally distributed variables. Spearman's Correlation: Measures
  monotonic relationships. Less sensitive to outliers and does not
  require normal distribution.
\end{itemize}

\begin{verbatim}
## 
##  Spearman's rank correlation rho
## 
## data:  dyad_summary$AgeDifference and dyad_summary$ToleranceProportion
## S = 48, p-value = 0.2992
## alternative hypothesis: true rho is not equal to 0
## sample estimates:
##       rho 
## 0.4285714
\end{verbatim}

\includegraphics{BEX2223_files/figure-latex/Age Spearman-1.pdf}

\begin{verbatim}
## `geom_smooth()` using formula = 'y ~ x'
\end{verbatim}

\includegraphics{BEX2223_files/figure-latex/Age Spearman-2.pdf}

\begin{itemize}
\item
  Using a Spearman correlation we found a spearmans correlation
  coefficient (rho) which is equal to 0,4285714 indicating a moderate
  positive association between ranks of age difference and tolerance
  proportion. It seems that as the age difference increases, the
  tolerance proportion increases
\item
  But looking at the p-value of 0,2992 we can see that the correlation
  is not statistically significant and there is a 29,92\% probability
  that the observed correlations occured by chance
\item
  X-axis: AgeDifference
\item
  Y-axis:
\item
  The correlation coefficient (0.0784) is very weak
\item
  Also the p.value, 0.8535 is not significant
\item
  It seems that they are not significant correlation between age
  difference and tolerance proportion
\end{itemize}

\hypertarget{section-3}{%
\paragraph{6.1.9.2}\label{section-3}}

\begin{itemize}
\tightlist
\item
  Model 2: Age direction \textasciitilde{} tolerance
\end{itemize}

\begin{verbatim}
## `geom_smooth()` using formula = 'y ~ x'
\end{verbatim}

\includegraphics{BEX2223_files/figure-latex/Age direction \& Tolerance-1.pdf}
\includegraphics{BEX2223_files/figure-latex/Age direction \& Tolerance-2.pdf}

\begin{verbatim}
## 
##  Welch Two Sample t-test
## 
## data:  ToleranceProportion by AgeDirection
## t = -1.4071, df = 4.5304, p-value = 0.2242
## alternative hypothesis: true difference in means between group Female Older and group Male Older is not equal to 0
## 95 percent confidence interval:
##  -0.27457658  0.08425425
## sample estimates:
## mean in group Female Older   mean in group Male Older 
##                  0.6315716                  0.7267327
\end{verbatim}

*p-value of 0.2242 suggests that there is no significant difference in
tolerance proportions between dyads where the male is older and those
where the female is older at the conventional alpha level of 0.05.

\hypertarget{section-4}{%
\paragraph{6.1.9.3}\label{section-4}}

\begin{itemize}
\item
  Model 3: Absolute age + Age direction \textasciitilde{} tolerance
\item
  First lets do a regression model
\end{itemize}

\begin{verbatim}
## 
## Call:
## lm(formula = ToleranceProportion ~ AgeDifference + AgeDirection, 
##     data = dyad_summary)
## 
## Residuals:
##        1        2        3        4        5        6        7        8 
## -0.04738  0.14899 -0.05564 -0.04934 -0.08396 -0.04513  0.10302  0.02944 
## 
## Coefficients:
##                        Estimate Std. Error t value Pr(>|t|)    
## (Intercept)            0.605582   0.069737   8.684 0.000335 ***
## AgeDifference          0.006127   0.012566   0.488 0.646498    
## AgeDirectionMale Older 0.102800   0.075076   1.369 0.229212    
## ---
## Signif. codes:  0 '***' 0.001 '**' 0.01 '*' 0.05 '.' 0.1 ' ' 1
## 
## Residual standard error: 0.1005 on 5 degrees of freedom
## Multiple R-squared:  0.2772, Adjusted R-squared:  -0.01193 
## F-statistic: 0.9588 on 2 and 5 DF,  p-value: 0.4442
\end{verbatim}

\includegraphics{BEX2223_files/figure-latex/ABsolute age + Age direction \& Tolerance-1.pdf}

\begin{verbatim}
## `geom_smooth()` using formula = 'y ~ x'
\end{verbatim}

\includegraphics{BEX2223_files/figure-latex/ABsolute age + Age direction \& Tolerance-2.pdf}

\begin{verbatim}
## 
## Call:
## lm(formula = ToleranceProportion ~ AgeDifference * AgeDirection, 
##     data = dyad_summary)
## 
## Residuals:
##        1        2        3        4        5        6        7        8 
## -0.06620  0.14940 -0.03185 -0.05146 -0.08012 -0.04772  0.09805  0.02991 
## 
## Coefficients:
##                                      Estimate Std. Error t value Pr(>|t|)   
## (Intercept)                          0.608345   0.077844   7.815  0.00145 **
## AgeDifference                        0.005475   0.014107   0.388  0.71769   
## AgeDirectionMale Older               0.030261   0.272127   0.111  0.91681   
## AgeDifference:AgeDirectionMale Older 0.023947   0.085541   0.280  0.79340   
## ---
## Signif. codes:  0 '***' 0.001 '**' 0.01 '*' 0.05 '.' 0.1 ' ' 1
## 
## Residual standard error: 0.1113 on 4 degrees of freedom
## Multiple R-squared:  0.2911, Adjusted R-squared:  -0.2406 
## F-statistic: 0.5475 on 3 and 4 DF,  p-value: 0.676
\end{verbatim}

\begin{verbatim}
## `geom_smooth()` using formula = 'y ~ x'
\end{verbatim}

\includegraphics{BEX2223_files/figure-latex/ABsolute age + Age direction \& Tolerance-3.pdf}

\begin{itemize}
\tightlist
\item
  Check of model assumptiuons: Residuals vs.~Fitted: Check for linearity
  and homoscedasticity. Normal Q-Q Plot: Assess normality of residuals.
  Scale-Location Plot: Check for homoscedasticity. Residuals
  vs.~Leverage: Identify influential observations.
\end{itemize}

\hypertarget{age-conclusion}{%
\paragraph{6.1.9.4 Age Conclusion}\label{age-conclusion}}

\begin{itemize}
\tightlist
\item
  It seems that both age difference and age direction within dyads re
  not significant and do not predict tolerance rates differences
\item
  This may come from the very small sample but we will try to explain
  tolerance with other factors as rank and intial social bond
\end{itemize}

\hypertarget{age-class-for-elo-rating-calculations}{%
\paragraph{6.1.9.5 Age Class for Elo Rating
Calculations}\label{age-class-for-elo-rating-calculations}}

\hypertarget{elo-rating}{%
\subsection{6.2 Elo Rating}\label{elo-rating}}

\begin{itemize}
\item
  In order to calculate the Elo Rating, I will have to create different
  files

  \begin{enumerate}
  \def\labelenumi{\arabic{enumi}.}
  \tightlist
  \item
    \textbf{FinalAgonistic} using the files
    \textbf{Agonistic2016-2023.csv, Agonistic.csv and Focal.csv}
  \item
    \textbf{WinnerLoser} using the newly created
    \textbf{FinalAgonistic.csv} and \textbf{IVP Life
    history\_180424.csv}
  \item
    \textbf{Elo rating} using presence matrices from the files
    \textbf{AK2020-2024.csv, BD2020-2024.csv, KB2020-2024.csv,
    NH2020-2024.csv} and ** WinnerLoser.csv**
  \end{enumerate}
\item
  As males and females hierarchy are distinct I will have to calculate
  their hierarchies separately
\item
  Speciffically for these codes I may have seqcheck() and elo.seq()
  functions errors as: -First interaction occurred before presence
  (approx) meaning that there's already an interaction recorded before
  the male was added into that group in the life history

  \begin{itemize}
  \tightlist
  \item
    In these cases I will have to either change the presence file
    manually or get rid of interactions before his first presence date.
  \end{itemize}
\end{itemize}

\hypertarget{creation-of-final-agonistic}{%
\subsubsection{6.2.1 Creation of Final
Agonistic}\label{creation-of-final-agonistic}}

\begin{itemize}
\item
  I will now create FinalAgonistic.csv using Agonistic data from 2016 to
  2023 and maybe the latest agonistic file until May 2023. Also I Focal
  data from June 2022 until the latest focals (date to check)
\item
  \textbf{FinalAgonistic.csv} will combine the above input files after
  filtering one-on-one interactions (excluding support interactions). It
  will involve cleaning and merging the raw data from agonistic
  interactions and focals to prepare it for the winner-loser
  calculations.
\item
  I will use the information of each first experiment day to know when
  to stop the elo calculations. For each dyad I will take the beggining
  of the month of the experiment.
\item
  \textbf{BD1 \textgreater{} SEPT 2022}

  \begin{enumerate}
  \def\labelenumi{\alph{enumi}.}
  \tightlist
  \item
    Sey Sirk : 14.09.2022
  \item
    Xia Piep : 16.09.2022
  \item
    Nge Oerw : 22.09.2022
  \item
    Xin Ouli : 27.09.2022
  \end{enumerate}
\item
  \textbf{AK \textgreater{} SEPT 2022}

  \begin{enumerate}
  \def\labelenumi{\alph{enumi}.}
  \setcounter{enumi}{4}
  \tightlist
  \item
    Sho Ginq : 27.09.2022
  \item
    Buk Ndaw : 29.09.2022
  \end{enumerate}
\item
  \textbf{BD2 \textgreater{} DEC 2022}

  \begin{enumerate}
  \def\labelenumi{\alph{enumi}.}
  \setcounter{enumi}{7}
  \tightlist
  \item
    Kom Oort : 12.12.2022
  \end{enumerate}
\item
  \textbf{NH \textgreater{} MAR 2023}

  \begin{enumerate}
  \def\labelenumi{\roman{enumi}.}
  \tightlist
  \item
    Pom Xian : 10.03.2023
  \end{enumerate}
\item
  Also, because the lastest dyad starts it's first trial after the
  \textbf{10.03.2023} I will remove all the elo data after this date
\item
  Note that male and female hierarchy are not comparable on the same
  scale in vervet monkeys so I will have to conduct 8 Elo Ration
  calcuations
\item
  \begin{enumerate}
  \def\labelenumi{\arabic{enumi}.}
  \tightlist
  \item
    BD1, Will calculate elo for all females in BD group so we can
    extract the ones of \textbf{Sirk, Piep, Oerw, Ouli} while for males
    we will look at the rank of \textbf{Sey, Xia, Nge, Xin}I will
    calculate their elo using the presence data and winner looser data
    until the \textbf{13.10.2022}
  \item
    BD2 will calculate again elo for males and females using the data
    untill the \textbf{11.12.2022} to extract the elo of \textbf{Kom and
    Oort} before their first trial day
  \item
    For AK, the elo will be calculated in once using the data untill the
    \textbf{26.09.2022} to then extract rank for females \textbf{Ginq,
    Ndaw} and males \textbf{Buk, Sho}
  \item
    Last Elo calculations for males and females in NH will use data
    until the \textbf{09.03.23} to extract rank of female \textbf{Xian}
    and Male\textbf{Pom} \#\#\# 6.2 Female Elo Ratings
  \end{enumerate}
\end{itemize}

\hypertarget{elo-ratings}{%
\subsubsection{6.3Elo Ratings}\label{elo-ratings}}

\hypertarget{select-final-dataset-export-for-manipulation}{%
\section{SELECT FINAL DATASET \& EXPORT FOR
MANIPULATION}\label{select-final-dataset-export-for-manipulation}}

\begin{itemize}
\tightlist
\item
  Here are the New Variables to Add -MAge and FAge: Male and female ages
  in years. -AgeDiff: Absolute difference between MAge and FAge.
  -IEloDiff: Absolute difference between male and female Elo
  quartiles.\\
  -Binomial Variables (Tol, Agg, NotApp): Based on your criteria.
  -Existing variables in BexClean: ``Date'', ``Month'', ``DyadDay'',
  ``Trial'', ``Time'', ``Period'', ``Male'', ``Female'', ``Dyad'',
  ``DyadDistance'', ``Proximity'', ``DyadResponse'',
  ``SpecialBehaviour'', ``Context'', ``ToleranceBinomial''
\end{itemize}

\begin{verbatim}
## # A tibble: 8 x 10
##   Dyad     Male  Female AgeDifference AgeDirection TotalTrials ToleranceCount
##   <chr>    <chr> <chr>          <dbl> <chr>              <int>          <dbl>
## 1 Sey Sirk Sey   Sirk           3.80  Male Older           586            401
## 2 Xia Piep Xia   Piep           4.87  Female Older         603            473
## 3 Nge Oerw Nge   Oerw           1.97  Male Older           182            121
## 4 Sho Ginq Sho   Ginq           0.986 Female Older         281            158
## 5 Xin Ouli Xin   Ouli          10.1   Female Older         185            108
## 6 Buk Ndaw Buk   Ndaw           0.255 Female Older         258            145
## 7 Kom Oort Kom   Oort           3.21  Male Older           367            305
## 8 Pom Xian Pom   Xian           4.95  Female Older         257            171
## # i 3 more variables: NoToleranceCount <dbl>, ToleranceProportion <dbl>,
## #   NoToleranceProportion <dbl>
\end{verbatim}

\begin{itemize}
\tightlist
\item
  Dyad,IEloDiff, IZDSi, AgeDiff, Date, DyadDay, DyadTrial, Resp, Tol,
  Agg, NoApp, EloSex
\end{itemize}

*MAge, FAge,MElo, FElo,

\begin{itemize}
\tightlist
\item
  IDsi mus be calculated using the same amount of time before 1st trial
  (ex. 60 days before 1st date of trial/dyad)
\item
  Elo Diff, must have the same requirement as IDSI for amount of data
  taken in account
\item
  Tol, Agg and NoApp will be binomial variables for Tolerance,
  Aggression and Not approaching
\end{itemize}

\hypertarget{research-question}{%
\section{RESEARCH QUESTION}\label{research-question}}

\begin{itemize}
\item
  Variations in Age, Rank, IDSI, and IEloDiff and time spent together
  (amount of trials on a set period/in general) can explain differneces
  in tolerance (aggression/not approaching?!) rates in a forced
  proximity box experiment conducted in wild vervet monkeys
\item
  The effects may be moderated by /\textgreater{} Seasonality, Male
  tenure, Female pregnancy
\item
  \ldots{}
\item
  Lower variations in age and rank and higher inital social bonds and
  overall amount of time spent otgether leads to higher rates of
  tolerance
\end{itemize}

\hypertarget{statistical-test}{%
\section{STATISTICAL TEST}\label{statistical-test}}

\begin{itemize}
\tightlist
\item
  glmer
\item
  \{lmerTest\} / pTol \textasciitilde{} AmountTrial (totalNtrial vs
  fixed amount each dyad) + AgeDiff + IEloDiff + IDsi +(Day\textbar{}
  Dyad)
\end{itemize}

\hypertarget{remarks}{%
\section{REMARKS}\label{remarks}}

\begin{itemize}
\tightlist
\item
  IDSI and IEloDiff will be fixed (Same value in all the data set/dyad)
  because we selected only the initial differences in rank and social
  bonds
\item
  If Tolerance is a proportion it is a bound value btw 1 or 0, beta
  regression ca be used for bound value or nbinomal model
\end{itemize}

\hypertarget{tasks}{%
\section{TASKS}\label{tasks}}

\begin{itemize}
\item
  \begin{quote}
  Check the evolution of the proportion of tolerance per Dyad
  \end{quote}
\item
  \begin{quote}
  Create new dataset with set varialbles
  \end{quote}
\item
  \begin{quote}
  \end{quote}
\item
  \begin{quote}
  Create Dataset and export as CSV
  \end{quote}
\item
  \begin{quote}
  Pom Elo from a month after experiment maybe since we started
  experiment as soon as he arrived in Noha
  \end{quote}
\item
  \begin{quote}
  ElO
  \end{quote}

  \begin{itemize}
  \tightlist
  \item
    Rank / Similar, MaleHigh, FemaleHigh but mention that diff,
    hiereachire for male and female elo but because we suppose
    interaction higer sex and rank
  \item
    Diff between quartile values \textgreater{} take absolute difference
    and the 3 categories to see if inclunece of higher sex /rank
  \end{itemize}
\end{itemize}

\hypertarget{questions}{%
\section{QUESTIONS}\label{questions}}

\begin{itemize}
\tightlist
\item
  Seasonality?
\item
  Male Tenure (Years 7 3 cat?) / Female Pregnancy (17= if bb for 3 month
  = yes if not 0)
\item
  Level of habituation, food availability (proxmiity with seasonality),
  temperature, group size, amount of males in group, previous
  interactions\ldots.
\end{itemize}

Paper to read: Factors affecting tolerance persistence after grooming
interactions in wild female vervet monkeys, Chlorocebus pygerythrus

therotical background Background: How do individual fogr relationship ,
how does life history diff impact how these relationships

Goal /Cooperation

Evolution of tolerance, creation of relationship, fwhat foactors change
that, food eovolution of social bonds, goal, group living male female
realtionhsip, friendhsips

why are picky

Tolerance

\hypertarget{to-sort-not-finished}{%
\section{TO SORT; NOT FINISHED}\label{to-sort-not-finished}}

\begin{itemize}
\tightlist
\item
  I want to use the date to know \textbf{how many sessions} have been
  done with each dyads in my experiment. * I will create a variable
  called \textbf{Session} where \textbf{1 session = 1 day} * The data
  has values from the \textbf{14th of September 2022} until the
  \textbf{13th of September 2023} * I will also create a variable called
  \textbf{Trial} to know how many trials have been done with each dyad
  where \textbf{1 row = 1 trial}
\end{itemize}

\begin{verbatim}
* I may consider, in parallel of my hypothesis, to separate the data in *4 seasons* to make a preliminary check of a potential effect of seasonality. Nevertheless the fact that we did not use anywithout      tools to mesure the weather and the idea to make a categorization in 4 seasons without considering the actua quite arbitrary. I may do it but with no intention to include this in my scientific report.
l temperature, food quantitiy and other elements related to seasonailty make this categorizationn a categorization where 12 months of          data will be separated in 4 categories
\end{verbatim}

\begin{itemize}
\tightlist
\item
  But before I may want to make a few changes already by merging
  \textbf{Male corn} and \textbf{Male placement corn} into '' Male
  corn'' and maybe replacing all of the NA's in ``Other response'' by
  response
\end{itemize}

\#Lines to check unique values in MaleFemaleID to see if they are any
problems with it \# Unique values in MaleID unique\_male\_ids \textless-
unique(BexClean\$MaleID)

\hypertarget{unique-values-in-femaleid}{%
\section{Unique values in FemaleID}\label{unique-values-in-femaleid}}

unique\_female\_ids \textless- unique(Bex\$FemaleID)

\hypertarget{sections-below-are-here-for-the-organization-of-my-paper-and-will-be-worked-on-once-the-data-cleaning-and-exploration-is-done}{%
\section{Sections below are here for the organization of my paper and
will be worked on once the data cleaning and exploration is
done}\label{sections-below-are-here-for-the-organization-of-my-paper-and-will-be-worked-on-once-the-data-cleaning-and-exploration-is-done}}

\hypertarget{describing-the-data}{%
\subsection{5. Describing the data}\label{describing-the-data}}

\hypertarget{research-question-hypothesis}{%
\subsection{5.Research question \&
Hypothesis}\label{research-question-hypothesis}}

\hypertarget{research-question-1}{%
\subsubsection{Research question}\label{research-question-1}}

\begin{itemize}
\item
  What factors influence the rate at which individuals (vervets) learn
  to tolerate each other in a controlled box experiment?
\item
  Ex: The rate at which individuals (vervets) learn to tolerate each
  other in a box experiment is influenced by social factors (audience,
  social network, behavior of the partner) and idioyncratic factors
  (age, rank)
\end{itemize}

\hypertarget{hypothesis}{%
\subsubsection{Hypothesis}\label{hypothesis}}

\begin{itemize}
\item
  \begin{enumerate}
  \def\labelenumi{\arabic{enumi}.}
  \tightlist
  \item
    Hypothesis about the Presence of High-Ranking Individuals:
  \end{enumerate}
\end{itemize}

The presence of a higher number of high-ranking individuals in the
audience will negatively correlate with the level of tolerance achieved
among vervets in the box experiment. This is expected to result in
higher frequencies of aggressive behaviors, intrusions, and loss of
interest, particularly from lower-ranking individuals.

\begin{itemize}
\item
  \begin{enumerate}
  \def\labelenumi{\arabic{enumi}.}
  \setcounter{enumi}{1}
  \tightlist
  \item
    Hypothesis about Partner Agonistic Behaviors:
  \end{enumerate}
\end{itemize}

Vervets tolerance levels in the box experiment will be influenced by
their partner's display of agonistic behaviors. Specifically, partners
who exhibit more frequent agonistic behaviors towards their partner will
lead to decrease in their motivation to participate in future trials.

\begin{itemize}
\item
  \begin{enumerate}
  \def\labelenumi{\arabic{enumi}.}
  \setcounter{enumi}{2}
  \tightlist
  \item
    Hypothesis about the Establishment of an Optimal Distance:
  \end{enumerate}
\end{itemize}

During the box experiment, vervet dyads will establish an ``optimal''
distance for interaction, characterized by a higher frequency of
tolerance compared to other distances. This optimal distance is expected
to signify that the individuals tolerate each other more effectively at
this specific proximity .

\begin{itemize}
\item
  \begin{enumerate}
  \def\labelenumi{\arabic{enumi}.}
  \setcounter{enumi}{3}
  \tightlist
  \item
    Hypothesis about Age and Rank:
  \end{enumerate}
\end{itemize}

The age and rank of individual vervets within the group will influence
the success of the trials in the box experiment. Specifically, older and
higher-ranking individuals are expected to exhibit lower rates of
success compared to dyads consisting of younger and lower-ranked
individuals. This decrease in success is anticipated to be associated
with a higher frequency of aggressive behaviors displayed by older and
higher-ranking individuals towards their partners. (I'm not sure this
hypothesis makes sens, I have the feeling age and rank must have an
influence but I don't know how to put it, I will think about it)

\begin{itemize}
\item
  \begin{enumerate}
  \def\labelenumi{\arabic{enumi}.}
  \setcounter{enumi}{4}
  \tightlist
  \item
    Hyptohesis about seasonality
  \end{enumerate}
\end{itemize}

Seasonality is expected to impact the motivation of vervet dyads to
participate in the box experiment. We hypothesize that dyads will have
lower motivation, as indicated by a reduced number of trials, during the
summer months compared to the winter months. This difference in
motivation is likely influenced by temperature and food availability. To
test this hypothesis, we will categorize the data into four seasonal
periods, each spanning four months, and analyze whether there is a
significant effect of seasonality on the motivation to engage in the
trials.

\hypertarget{statistical-tests-and-analisis-of-the-data}{%
\subsection{6.Statistical tests and analisis of the
data}\label{statistical-tests-and-analisis-of-the-data}}

\hypertarget{statistical-tests}{%
\subsubsection{Statistical tests}\label{statistical-tests}}

\begin{itemize}
\tightlist
\item
  \textbf{Hypothesis 1}: Influence of High-Ranking Individuals
\end{itemize}

Variables Needed:

\textbf{DyadResponse} (specifically, ``aggression'' responses)
\textbf{Amountaudience} (to measure the number of individuals in the
audience) \textbf{Audience\ldots15} (to identify the names of
individuals in the audience for calculating dominance ranks) \textbf{Elo
rating} of the individuals based on the ab libitum data collected in IVP
(which I have to calculate asap)

Statistical Analysis:\textbf{Logistic Regression}, as it could analyze
the influence of high-ranking individuals on the occurrence of
aggression in dyad responses. This will help determine whether the
presence of high-ranking individuals affects the likelihood of
aggression.

\begin{itemize}
\tightlist
\item
  \textbf{Hypothesis 2}: Impact of Partner's Agonistic Behaviors
\end{itemize}

Variables Needed:

\begin{itemize}
\tightlist
\item
  \textbf{DyadResponse} (specifically, ``aggression'' responses)
\item
  \textbf{MaleagressF} (male's aggression towards female)
\item
  \textbf{FemaleaggressM} (female's aggression towards male)
\end{itemize}

Statistical Analysis: \textbf{Logistic Regression} as it could be used
to assess how the occurrence of aggression in dyad responses is
influenced by the partner's gender-specific agonistic behaviors.

\begin{itemize}
\tightlist
\item
  \textbf{Hypothesis 3}: Identification of an Optimal Interaction
  Distance
\end{itemize}

Variables Needed:

\begin{itemize}
\tightlist
\item
  \textbf{DyadDistance} (distance between boxes)
\item
  \textbf{Tolerance} (as a binary outcome)
\end{itemize}

Statistical Analysis: \textbf{generalized Linear Model (GLM)} to
investigate whether there is an optimal distance that leads to a higher
likelihood of tolerance (Tolerance = 1).

\begin{itemize}
\tightlist
\item
  \textbf{Hypothesis 4}: Role of Age and Rank
\end{itemize}

Variables Needed:

\begin{itemize}
\tightlist
\item
  \textbf{Tolerance} (as a binary outcome)
\item
  \textbf{Male and Female} (to identify individuals' ages and ranks)
\item
  \textbf{Dyad} (to link individuals to dyads)
\item
  \textbf{Birthdate} to calculate the age of each individual
\end{itemize}

Statistical Analysis: \textbf{Logistic Regression} Logistic regression
can be employed to determine whether the age and rank of individual
vervets within dyads have an impact on the likelihood of tolerance
(Tolerance = 1).

\begin{itemize}
\tightlist
\item
  \textbf{Hypothesis 5}: Influence of Seasonality
\end{itemize}

Variables Needed:

\begin{itemize}
\tightlist
\item
  \textbf{Date} (to categorize data into seasons)
\item
  \textbf{Trial} (to count the number of trials in each season) and the
  data for at least 365 days so i can separate the data in 4 (1 year = 4
  seasons = 12*4 month) to see if they may be an effect of seasonality
  on the motivation (amount of trials) of the dyads
\end{itemize}

Statistical Analysis:

ANOVA or Kruskal-Wallis Test: Depending on the distribution of your
trial data, you can use either ANOVA (if the data are normally
distributed) or the Kruskal-Wallis test (for non-normally distributed
data) to assess the impact of seasonality on the number of trials. If
significant differences are found, you can follow up with post-hoc tests
to identify which seasons differ from each other. Please note that the
effectiveness of these analyses may depend on the distribution of your
data and specific research objectives. You may also consider conducting
exploratory data analysis (e.g., visualization) to gain a better
understanding of your dataset before performing these analyses.
Additionally, if you have specific questions about data preprocessing or
variable transformations, feel free to ask for further guidance.
--\textgreater{} I took this from ChatGPT, I have to look more into it

\textbf{REMARKS}: So here are a few updates I made in the document. I
also planned to send my cleaned data to Radu (the statistician of UNINE)
as he was keen to help me find the right test. Of course I will also
look again in Bshary's and Charlotte's work with the boxes and improve
these suggestions that are quite simple for now

Also I still have to clean the last grpahs about male/female aggression
as I didn't finish that yet. I juste wanted to share my hypothesis and
ideas for statistics so I can soon go into the ``serious'' work

Anyway, thank you in advance for your help \textless3

Michael

\hypertarget{plotting-the-results-of-the-analysis}{%
\subsection{7. Plotting the results of the
analysis}\label{plotting-the-results-of-the-analysis}}

\hypertarget{interpretation-of-the-results}{%
\subsection{9. Interpretation of the
results}\label{interpretation-of-the-results}}

\hypertarget{comeback-on-the-research-question-and-hypothesis}{%
\subsection{10. Comeback on the research question and
hypothesis}\label{comeback-on-the-research-question-and-hypothesis}}

\hypertarget{bibliography}{%
\subsection{11. Bibliography}\label{bibliography}}

\hypertarget{organization-for-my-paper}{%
\subsection{12. Organization for my
paper}\label{organization-for-my-paper}}

\begin{itemize}
\tightlist
\item
  Introduction

  \begin{itemize}
  \tightlist
  \item
    Tolerance humans, primates
  \item
    Apes vs monkeys / Captivity vs Wild
  \item
    IVP: Wild habituated vervets, experiments possible
  \item
    Paper Bshary, Canteloup\ldots{} Prolongation study
  \item
    Relevance idea/topic research
  \item
    Research question \& hypothesis
  \end{itemize}
\end{itemize}

But: intro need triangle shape: broad to narrow end wiht research
question\textgreater{} tolerance importance \textgreater{} animal reign,
actual knowledge/ direction knowledge we need \textgreater{} show how my
experiment goes in that way How to adress the gap, answer with research
question

Then explain why choosing vervet monkeys, (IVP in methods), sociality,
experiments made

\begin{itemize}
\item
  Methods

  \begin{itemize}
  \item
    IVP, research area, (goal, house, type people)
  \item
    Population: groups, dyads, male/female, ranks..
  \item
    Box material: boxes, remotes, batteries, camera, tripod, corn (no
    marmelade ;), (water spray, security reason, non agressive way to
    select individuals and not engage with mokeys when reachrging boxes
    with corn), pattern, previous distances, tablets, box experiment
    form
  \item
    Tablets
  \item
    (No observers mentionned)
  \item
    Habituation boxes \textgreater{} individuals trained to recognice
    boxes, they have differernt levels of habituation
  \item
    Patterns \textgreater{} appendix, mention similar to habituation,
    use to recognize box but efficieny depeds of experience)
  \item
    Selection dyads \textgreater{} assigment from elo rating (different
    rank), if above average bond no dyad made, if not possible,
    availibilty of monkey also factor !! Non random can be a problem,
    think about why and how you selected data We created variations in
    dyads made by different sex, rank and not above average bonde
    (calculate bondeness)
  \item
    Amount corn, do you want to mention it\textgreater{} maybe important
    Calculate corn during and placement cf paper on corn /food
    motivation
  \item
    Corn (daily intake vervet \% made from corn, cf site we saw, cf
    screenshot, comapre paper previousely made an all)
  \item
    1st dyad trial (BD) \textgreater{} appendix
  \item
    Videos \textgreater{} details appendix
  \item
    Finding dyads \textgreater{} appendix
  \item
    Placement to attract them \textgreater{} meniton if statiscial made
    on placement corn
  \item
    Trials (1 session = max 15 trials/in total) (session could be broken
    in different sub sessions to reach 15 trials max)
  \item
    If agression \textgreater{} 1m / If 2x tolerance \textless{} 1m ,
    also if not approaching \textgreater{} 1m ( if no tolerance increase
    distance except if intrusion) (borgeaud \textgreater{} expectation
    fo aggression)
  \item
    Time of the day \textgreater{} appendix
  \item
    Territory? \textgreater{} appendix
  \item
    Amount sessions p day/week, how we chose the moment to follow them
    \textgreater appendix
  \item
    Problems/ unplanned events: weather, BGE's, not finding the monkeys
    (group, dyad or individual), dispersal of males, river crossing,
    inacessibility (experiments or boxes), low vision (experiments or
    monkeys),\textgreater{} appendix
  \item
    (Where do i mention the confounding variables?) \textgreater{} look
    in litterature, if something that could affect and already reported
    in papers check, oterhwise exclude ``normal life'' factors for both
    monekys and Experimenter
  \item
    Types of experimental plan
  \item
    Statistical tests (for each hypothesis)
  \end{itemize}
\item
  Analysis
\item
  Results
\item
  Interpretation
\item
  Conclusion
\end{itemize}

\hypertarget{key-cognitive-skilly-that-may-be-involved-in-the-box-experiment}{%
\section{Key cognitive skilly that may be involved in the box
experiment}\label{key-cognitive-skilly-that-may-be-involved-in-the-box-experiment}}

\begin{itemize}
\tightlist
\item
  Problem solving
\item
  Memory
\item
  Conflict resolution
\item
  Cooperation
\end{itemize}

\hypertarget{key-idea-to-develop-insghts}{%
\section{Key idea to develop \&
insghts}\label{key-idea-to-develop-insghts}}

\begin{itemize}
\item
  Inter sexual food competition/or tolerance: which type of adaptation
\item
  Tolerance
\item
  Adaptation to a new context (box experiment setting and repeated
  encounters with a specific individual)
\item
  Evolution of tolerance in competitive context
\item
  Social cognition \textgreater{} meaning influence of the group or
  audience on the choice of an individual of a Dyad
\item
  Primate social structure and dynamics
\item
  Foundation of tolerance, solving of competition related to food,
  mechanisms underlying male-female vervet relationships, dyadic
  interactions
\item
  Insights on the evolution of complex decision making in a social
  context and adaptive mechanisms related to food competition x
\end{itemize}

In summary, studying dyadic interactions, particularly between males and
females in vervet monkeys, not only enhances our understanding of
evolutionary origins and adaptive advantages of complex cognition but
also provides valuable insights into human social behavior, cooperation,
and cognitive foundations shared across species. These studies bridge
the gap between animal behavior research and cognitive science, offering
interdisciplinary perspectives on the complexities of social
interactions and their implications for both animal and human societies.

\hypertarget{glossary}{%
\section{Glossary}\label{glossary}}

\begin{itemize}
\tightlist
\item
  \textbf{Tolerance}: Tolerance: An individual has an encounter with a
  conspecific and can freely leave but remains in the encounter without
  acting aggressively toward the conspecific. (Pisor \& Surbeck, 2019)
\item
  \textbf{Agression}
\item
  \textbf{Session}
\item
  \textbf{Trial}
\item
  \textbf{Group}: In the Primate order, groups are individuals ``which
  remain {[}physically{]} together in or separate from a larger unit''
  and interact with each other more than with other individuals.6 This
  definition does not cover all uses of the word ``group'' in the social
  sciences (e.g., human identity groups who identify with a common name
  or symbol may or may not interact with one another more frequently
  than with other individuals). Because of this ambiguity, we use the
  word ``community'' when referring to humans to better capture the
  notion of spatial proximity, per Ref. 54. Members of the same group
  are referred to as ``same-group'' and those from another group
  ``extra-group.'' (Pisor \& Surbeck, 2019)
\end{itemize}

\hypertarget{bibliography-1}{%
\section{Bibliography}\label{bibliography-1}}

• Pisor, A. C., \& Surbeck, M. (2019). The evolution of intergroup
tolerance in nonhuman primates and humans. Evolutionary Anthropology:
Issues and ReViews. Advance online publication.
\url{https://doi.org/10.1002/evan.21793} (Pisor \& Surbeck, 2019)

\hypertarget{annex}{%
\section{Annex}\label{annex}}

\hypertarget{annex-1-view-of-the-dataset-when-imported---first-6-entries-of-each-variable}{%
\paragraph{Annex 1 : View of the dataset when imported - First 6 entries
of each
variable}\label{annex-1-view-of-the-dataset-when-imported---first-6-entries-of-each-variable}}

\begin{itemize}
\tightlist
\item
  We can see here the brief View of the \textbf{original dataset} names
  \textbf{BoxEx}when i initially imported it as seen in \textbf{section
  0: Opening data}
\end{itemize}

\begin{longtable}[]{@{}cccc@{}}
\caption{First Few Entries (continued below)}\tabularnewline
\toprule
Date & Time & Data & Group \\
\midrule
\endfirsthead
\toprule
Date & Time & Data & Group \\
\midrule
\endhead
2022-09-27 & 1899-12-31 09:47:50 & Box Experiment & Baie Dankie \\
2022-09-27 & 1899-12-31 09:50:07 & Box Experiment & Baie Dankie \\
2022-09-27 & 1899-12-31 09:53:11 & Box Experiment & Baie Dankie \\
2022-09-27 & 1899-12-31 09:54:28 & Box Experiment & Baie Dankie \\
2022-09-27 & 1899-12-31 09:55:19 & Box Experiment & Baie Dankie \\
2022-09-27 & 1899-12-31 09:56:56 & Box Experiment & Baie Dankie \\
\bottomrule
\end{longtable}

\begin{longtable}[]{@{}cccc@{}}
\caption{Table continues below}\tabularnewline
\toprule
GPSS & GPSE & MaleID & FemaleID \\
\midrule
\endfirsthead
\toprule
GPSS & GPSE & MaleID & FemaleID \\
\midrule
\endhead
-28.010549999999999 & 31.191050000000001 & Nge & Oerw \\
-28.010549999999999 & 31.191050000000001 & Nge & Oerw \\
-28.010549999999999 & 31.191050000000001 & Nge & Oerw \\
-28.010549999999999 & 31.191050000000001 & Nge & Oerw \\
-28.010549999999999 & 31.191050000000001 & Nge & Oerw \\
-28.010549999999999 & 31.191050000000001 & Nge & Oerw \\
\bottomrule
\end{longtable}

\begin{longtable}[]{@{}ccccc@{}}
\caption{Table continues below}\tabularnewline
\toprule
Male placement corn & MaleCorn & FemaleCorn & DyadDistance &
DyadResponse \\
\midrule
\endfirsthead
\toprule
Male placement corn & MaleCorn & FemaleCorn & DyadDistance &
DyadResponse \\
\midrule
\endhead
NA & 3 & NA & 2m & Tolerance \\
NA & 3 & NA & 2m & Tolerance \\
NA & 3 & NA & 1m & Tolerance \\
NA & 3 & NA & 1m & Tolerance \\
NA & 3 & NA & 0m & Tolerance \\
NA & 3 & NA & 0m & Tolerance \\
\bottomrule
\end{longtable}

\begin{longtable}[]{@{}cccc@{}}
\caption{Table continues below}\tabularnewline
\toprule
OtherResponse & Audience & IDIndividual1 & IntruderID \\
\midrule
\endfirsthead
\toprule
OtherResponse & Audience & IDIndividual1 & IntruderID \\
\midrule
\endhead
NA & Obse; Oup; Sirk & NA & NA \\
NA & Obse; Oup; Sirk & NA & NA \\
NA & Oup; Sirk & NA & NA \\
NA & Sirk & NA & NA \\
NA & Sey; Sirk & NA & NA \\
NA & Sey; Sirk & NA & NA \\
\bottomrule
\end{longtable}

\begin{longtable}[]{@{}
  >{\centering\arraybackslash}p{(\columnwidth - 0\tabcolsep) * \real{1.00}}@{}}
\caption{Table continues below}\tabularnewline
\toprule
\begin{minipage}[b]{\linewidth}\centering
Remarks
\end{minipage} \\
\midrule
\endfirsthead
\toprule
\begin{minipage}[b]{\linewidth}\centering
Remarks
\end{minipage} \\
\midrule
\endhead
NA \\
NA \\
Nge box did not open because of the battery. Oerw vocalized to MA when
he ap to the box to open it. \\
Sey came to the boxes once they were open \\
NA \\
NA \\
\bottomrule
\end{longtable}

\begin{longtable}[]{@{}cc@{}}
\toprule
Observers & DeviceId \\
\midrule
\endhead
Josefien; Michael; Ona; Zonke &
\{7A4E6639-7387-7648-88EC-7FD27A0F258A\} \\
Josefien; Michael; Ona; Zonke &
\{7A4E6639-7387-7648-88EC-7FD27A0F258A\} \\
Josefien; Michael; Ona; Zonke &
\{7A4E6639-7387-7648-88EC-7FD27A0F258A\} \\
Josefien; Michael; Ona; Zonke &
\{7A4E6639-7387-7648-88EC-7FD27A0F258A\} \\
Josefien; Michael; Ona; Zonke &
\{7A4E6639-7387-7648-88EC-7FD27A0F258A\} \\
Josefien; Michael; Ona; Zonke &
\{7A4E6639-7387-7648-88EC-7FD27A0F258A\} \\
\bottomrule
\end{longtable}

\end{document}
